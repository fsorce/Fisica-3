\chapter{Gas ideale}
\begin{definition}[Mole]
Una \textbf{mole di una sostanza} corrisponde a $6.02\cdot 10^{23}$ particelle di quella sostanza. La costante \`e detta \textbf{numero di Avogadro} e la indichiamo con $N_a \mathrm{mol}\ii$. 
\end{definition}

\begin{definition}[Condizioni standard]
Un gas \`e in \textbf{condizioni standard} (\textbf{STP}) se \`e alla temperatura di $0^\circ\mathrm C$ e alla pressione di $1\ \mathrm{atm}=101.3245\ \mathrm{kPa}$.
\end{definition}

\noindent Per i gas ideali valgono le seguenti leggi:
\begin{fact}[Legge di Boyle]
Se $T$ \`e costante
\[V\propto \frac1p\]
\end{fact}
\begin{fact}[Legge di Charles]
Se $p$ \`e costante
\[V\propto (1+\al T)\]
\end{fact}
\begin{fact}[Legge di Gay-Lussac]
Se $V$ \`e costante
\[p\propto T\]
\end{fact}
\begin{fact}[Legge di Avogadro]
Se $p$ e $T$ sono fissate, tutti i gas occupano lo stesso volume se consistono della stessa quantit\`a di materia, in particolare
\[V\propto n.\]
Una mole di gas in condizioni standard occupa un volume di $22.4\ell$ (litri).
\end{fact}

\noindent Combinando le leggi appena citate arriviamo alla \textbf{legge dei Gas perfetti}
\[\boxed{pV=nRT}\]
dove $p$ \`e la pressione, $V$ \`e il volume, $n$ \`e il numero di moli, $T$ \`e la temperatura e $R$ \`e la \textbf{costante fondamentale dei gas} e vale $8.314 \frac{\mathrm{J}}{\mathrm{K}\ \mathrm{mol}}$.

\begin{definition}[Costante di Boltzmann]
Definiamo la \textbf{costante di Boltzmann} $k_b$ in modo tale che 
\[R=N_a k_b.\]
\end{definition}