\documentclass[a4paper]{report}
\input{../../Templates/preamble.tex}
%PER CAMBIARE I MARGINI
\usepackage[margin=4cm]{geometry}

%----------- Setup stilistico ----------------
\definecolor{DarkRed}{HTML}{B6321C}
\hypersetup{
    colorlinks=true,
    linkcolor=DarkRed,
    filecolor=blue,
    citecolor = black,
    urlcolor=cyan,
}
\renewcommand\thefootnote{\textcolor{blue}{\arabic{footnote}}}
% ============================================


%---------- Comandi specifici ----------------


%--------- Comandi dattilografici ------------


% ============================================
\title{Fisica 3\\
\large Corso del prof. Sozzi Marco}

\author{Francesco Sorce}
\date{Università di Pisa\\
Dipartimento di Matematica\\
A.A. 2023/24}

\begin{document}
\maketitle

%\newpage
\tableofcontents
\newpage


\part{Termodinamica}
\chapter{Introduzione}
\noindent
La termodinamica \`e lo studio di sistemi dal punto di vista macroscopico.\\
Le massime fondamentali della termodinamica sono
\begin{itemize}
\item L'energia dell'universo \`e costante
\item L'entropia dell'universo tende ad aumentare.
\end{itemize}

\section{Prime definizioni}
\begin{definition}[Sistema termodinamico]
Un \textbf{sistema termodinamico} \`e un sistema omogeneo composto da ``molti" elementi.\\
Lo \textbf{stato} di un sistema termodinamico \`e univocamente determinato da un numero contenuto di parametri\footnote{Per esempio temperatura, pressione o volume.} detti \textbf{funzioni di stato}.\\
Il numero di funzioni di stato necessarie per specificare lo stato \`e detto \textbf{numero di gradi di libert\`a}.
\end{definition}

\begin{remark}
Le funzioni di stato di un sistema non dipendono da come esso \`e venuto ad esistere; se due procedimenti portano da un particolare stato ad un altro, le differenze nelle funzioni di stato dipendono univocamente dallo stato iniziale e quello finale.
\end{remark}

\begin{remark}[Sistema ambiente]
Spesso torna comodo considerare una coppia di sistemi, uno detto semplicemente sistema e l'altro \textbf{ambiente}.
\end{remark}

\begin{definition}[Variabili estensive e intensive]
Dato un sistema termodinamico, delle variabili ad esso inerenti si dicono \textbf{estensive} se sono proporzionali alla quantit\`a di materia contenuta nel sistema e \textbf{intensive} altrimenti.
\end{definition}

\begin{example}
Il volume e l'energia sono grandezze estensive mentre la pressione e la temperatura sono intensive.
\end{example}

\begin{definition}[Sistemi isolati, chiusi e aperti]
Un sistema termodinamico si dice 
\begin{itemize}
\item \textbf{isolato} se non ammette scambio con l'ambiente,
\item \textbf{chiuso} se non ammette scambio di materia con l'ambiente,
\item \textbf{aperto} se ammette scambi con l'ambiente.
\end{itemize}
\end{definition}

Per considerare pi\`u sistemi termodinamici dobbiamo considerarli come separati da una \textit{parete}.

\begin{definition}[Tipi di parete]
Una parete tra due sistemi \`e
\begin{itemize}
\item \textbf{adiabatica} se non permette scambi,
\item \textbf{diatermica} se non ammette scambi di materia,
\item \textbf{semipermeabile} se fa passare alcuni tipi di materia.
\item \textbf{permeabile}\footnote{una parete permeabile \`e come se non ci fosse} se permette ogni tipo di scambio.
\end{itemize}
\end{definition}

\begin{definition}[Equilibrio]
Un sistema \`e in \textbf{equilibrio} se le sue funzioni di stato restano ``costanti" (per molto tempo rispetto alla scala temporale rilevante).\\
Un sistema \`e in \textbf{equilibrio termico} se non ci sono differenze di temperatura\footnote{definiremo la temperatura in seguito.}.\\
Un sistema \`e in \textbf{equilibrio termodinamico} se \`e in equilibrio meccanico, termico e chimico.
\end{definition}

\begin{remark}
I sistemi tendono spontaneamente ed irreversibilmente all'equilibrio termodinamico.
\end{remark}

\begin{definition}[Equazione di stato]
Se quando un sistema \`e in equilibrio vale una equazione tra le funzioni di stato, queste si dicono \textbf{equazioni di stato}.
\end{definition}

\begin{definition}[Tipi di trasferimenti di energia]
Considerato un sistema termodinamico e l'ambiete definiamo le seguenti tipologie di scambi di energia:
\begin{itemize}
\item uno scambio di energia meccanica \`e detto \textbf{lavoro},
\item uno scambio di energia termica \`e detto \textbf{calore},
\item uno scambio di energia chimica \`e definito da \[\Delta E=\int \mu dn,\] dove $n$ \`e il numero di particelle coinvolte e $\mu$ \`e il \textbf{potenziale chimico}.\footnote{questa quantit\`a ha senso solo per sistemi aperti.}
\end{itemize}
Affermiamo per convenzione che uno scambio di energia ha segno \textit{positivo} se il sistema \underline{acquista energia dall'ambiente}.
\end{definition}

\begin{remark}
Il lavoro meccanico \`e dato da $W=\int \vec F\cdot \vec{d\ell}$. \`E un fatto generale che il lavoro ha la forma
\[\int(\text{intensiva})d(\text{estensiva}).\]
\end{remark}



\begin{definition}[Processi quasistatici]
Un sistema \`e \textbf{quasi in equilibrio} se \`e cos\`i vicino all'equilibrio che le equazioni di stato si possono considerare valide. Un \textbf{processo quasistatico} \`e descrivibile da una successione di variazioni infinitesime tra stati vicini all'equilibrio.\\
Se non sono presenti ``attriti", un processo quasistatico \`e detto \textbf{reversibile}.\\
Un processo \`e detto \textbf{totalmente reversibile} se \`e reversibile e la sua interazione con l'ambiente \`e reversibile.
\end{definition}

\begin{definition}[Termostato]
Un \textbf{termostato} \`e un sistema grande a sufficienza in modo che anche se vi si aggiunge calore esso non cambia di temperatura. \`E dunque una sorgente ideale di calore.
\end{definition}

\begin{definition}[Termometro]
Un \textbf{termometro} \`e un sistema piccola a sufficienza in modo che ogni scambio di calore \`e trascurabile.
\end{definition}




%\part{Relativit\`a speciale}

%\part{Basi di meccanica quantistica}



%\appendix
%\include{Riconoscimenti.tex}

\end{document}