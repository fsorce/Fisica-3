\documentclass[a4paper]{report}
\input{../../Templates/preamble.tex}
%PER CAMBIARE I MARGINI
\usepackage[margin=4cm]{geometry}

%----------- Setup stilistico ----------------
\definecolor{DarkRed}{HTML}{B6321C}
\hypersetup{
    colorlinks=true,
    linkcolor=DarkRed,
    filecolor=blue,
    citecolor = black,
    urlcolor=cyan,
}
\renewcommand\thefootnote{\textcolor{blue}{\arabic{footnote}}}
% ============================================


%---------- Comandi specifici ----------------
\newcommand{\arctanh}{\mathrm{arctanh}\,}

%--------- Comandi dattilografici ------------
\NewDocumentCommand{\ddb}{O{}mmm}{
    {\left.\frac{d^{#1}{#3}}{d{#2}^{#1}}\right|_{#4}}
}
\NewDocumentCommand{\ppb}{O{}mmm}{
    {\left.\frac{\partial^{#1}{#3}}{\partial{#2}^{#1}}\right|_{#4}}
}

\NewDocumentCommand{\pps}{O{}mm}{
    {{\frac{\partial^{#1}{#3}}{\partial{#2}^{#1}}}}
}

\NewDocumentCommand{\dds}{O{}mm}{
    {{\frac{d^{#1}{#3}}{d{#2}^{#1}}}}
}


% ============================================
\title{Fisica 3\\
\large Corso del prof. Sozzi Marco}

\author{Francesco Sorce}
\date{Università di Pisa\\
Dipartimento di Matematica\\
A.A. 2023/24}

\begin{document}
\maketitle

%\newpage
\tableofcontents
\newpage


\part{Termodinamica}
\chapter{Equilibrio e Calore}
\noindent
La termodinamica \`e lo studio di sistemi dal punto di vista macroscopico.\\
Le massime fondamentali della termodinamica sono
\begin{itemize}
\item L'energia dell'universo \`e costante
\item L'entropia dell'universo tende ad aumentare.
\end{itemize}

\section{Prime definizioni}
\begin{definition}[Sistema termodinamico]
Un \textbf{sistema termodinamico} \`e un sistema omogeneo composto da ``molti" elementi.\\
Lo \textbf{stato} di un sistema termodinamico \`e univocamente determinato da un numero contenuto di parametri\footnote{Per esempio temperatura, pressione o volume.} detti \textbf{funzioni di stato}.\\
Il numero di funzioni di stato necessarie per specificare lo stato \`e detto \textbf{numero di gradi di libert\`a}.
\end{definition}

\begin{remark}
Le funzioni di stato di un sistema non dipendono da come esso \`e venuto ad esistere; se due procedimenti portano da un particolare stato ad un altro, le differenze nelle funzioni di stato dipendono univocamente dallo stato iniziale e quello finale.
\end{remark}

\begin{remark}[Sistema ambiente]
Spesso torna comodo considerare una coppia di sistemi, uno detto semplicemente sistema e l'altro \textbf{ambiente}.
\end{remark}

\begin{definition}[Variabili estensive e intensive]
Dato un sistema termodinamico, delle variabili ad esso inerenti si dicono \textbf{estensive} se sono proporzionali alla quantit\`a di materia contenuta nel sistema e \textbf{intensive} altrimenti.
\end{definition}

\begin{example}
Il volume e l'energia sono grandezze estensive mentre la pressione e la temperatura sono intensive.
\end{example}

\begin{remark}
Il lavoro meccanico \`e dato da $W=\int \vec F\cdot \vec{d\ell}$. \`E un fatto generale che il lavoro ha la forma
\[\int(\text{intensiva})d(\text{estensiva}).\]
\end{remark}

\begin{definition}[Sistemi isolati, chiusi e aperti]
Un sistema termodinamico si dice 
\begin{itemize}
\item \textbf{isolato} se non ammette scambio con l'ambiente,
\item \textbf{chiuso} se non ammette scambio di materia con l'ambiente,
\item \textbf{aperto} se ammette scambi con l'ambiente.
\end{itemize}
\end{definition}

Per considerare pi\`u sistemi termodinamici dobbiamo considerarli come separati da una \textit{parete}.

\begin{definition}[Tipi di parete]
Una parete tra due sistemi \`e
\begin{itemize}
\item \textbf{adiabatica} se non permette scambi,
\item \textbf{diatermica} se non ammette scambi di materia,
\item \textbf{semipermeabile} se fa passare alcuni tipi di materia.
\item \textbf{permeabile}\footnote{una parete permeabile \`e come se non ci fosse} se permette ogni tipo di scambio.
\end{itemize}
\end{definition}

\section{Principio 0, Equilibrio e Temperatura empirica}
\subsection{Tipi di equilibrio}
\begin{definition}[Equilibrio]
Un sistema \`e in \textbf{equilibrio} se le sue funzioni di stato restano ``costanti" (per molto tempo rispetto alla scala temporale rilevante).\\
Un sistema \`e in \textbf{equilibrio termico} se non ci sono differenze di temperatura\footnote{definiremo la temperatura in seguito.}.\\
Un sistema \`e in \textbf{equilibrio termodinamico} se \`e in equilibrio meccanico, termico e chimico.
\end{definition}

\begin{remark}
I sistemi tendono spontaneamente ed irreversibilmente all'equilibrio termodinamico.
\end{remark}

\begin{fact}[0-esimo principio della termodinamica]
\textbf{Due sistemi in equilibrio termico con un terzo sono in equilibrio tra loro.}
\end{fact}

\begin{definition}[Equazione di stato]
Se quando un sistema \`e in equilibrio vale una equazione tra le funzioni di stato, queste si dicono \textbf{equazioni di stato}.
\end{definition}

\begin{remark}[Segno degli scambi di energia]
\underline{\textbf{NOTA BENE:}} Affermiamo per convenzione che uno scambio di energia ha segno \textit{positivo} se il sistema \textbf{\textit{acquista energia dall'ambiente}}.
\end{remark}



\subsection{Processi quasistatici}

\begin{definition}[Processi quasistatici]
Un sistema \`e \textbf{quasi in equilibrio} se \`e cos\`i vicino all'equilibrio che le equazioni di stato si possono considerare valide.\\ 
Un \textbf{processo quasistatico} \`e un processo tale per cui il sistema \`e quasi in equilibrio in ogni istante.\\
Se non sono presenti ``attriti", un processo quasistatico \`e detto \textbf{reversibile}.\\
Un processo \`e detto \textbf{totalmente reversibile} se \`e reversibile e la sua interazione con l'ambiente \`e reversibile.
\end{definition}

\begin{remark}
Un processo quasi statico va pensato come un processo molto lento; cos\`i lento da poter pensare al sistema come ``sempre in equilibrio".
\end{remark}



\subsection{Temperatura empirica}

\begin{proposition}[Temperatura empirica]\label{TemperaturaEmpirica}
Ogni sistema termodinamico ammette una funzione che \`e costante in stato di equilibrio. La costante \`e detta \textbf{temperatura empirica}.
\end{proposition}
\begin{proof}
Consideriamo tre sistemi, con funzioni di stato $(x_1, y_1),\ (x_2,y_2)$ e $(x_3,y_3)$ in equilibrio tra loro. Esistono dunque equazioni di stato della forma
\[\begin{cases}
x_3=f(x_1,y_1,y_3)\\
x_3=g(x_2,y_2,y_3)
\end{cases}\]
poich\'e i sistemi 1 e 2 sono in equilibrio, se eguagliamo le due equazioni sappiamo che ci\`o che otteniamo non dipende da $y_3$, quindi
\[\begin{cases}
f(x_1,y_1,y_3)=\phi_1(x_1,y_1)\zeta(y_3)+\eta(y_3)\\
g(x_2,y_2,y_3)=\phi_2(x_2,y_2)\zeta(y_3)+\eta(y_3)
\end{cases}\]
dunque se 1 e 2 sono in equilibrio si ha che 
\[\phi_1(x_1,y_1)=\phi_2(x_2,y_2),\]
ma i due membri dipendono da insiemi di variabili disgiunti, quindi esiste $\theta_0$ tale che entrambe queste espressioni eguagliano $\theta_0$ se sono in equilibrio. Il valore $\theta_0$ \`e detto la temperatura empirica dei sistemi, i quali sono in equilibrio solo se hanno la stessa temperatura empirica.
\end{proof}

\begin{definition}[Isoterme]
Dato un sistema termodinamico e un valore $\theta_0$ di temperatura empirica, chiamiamo \textbf{isoterma a livello $\theta_0$} l'insieme degli stati del sistema la cui temperatura \`e $\theta_0$.
\end{definition}


\subsection{Definizione di temperatura tramite gas}
\begin{fact}[Punto triplo]
\emph{Considerando come sistema termodinamico dell'acqua esiste una precisa combinazione di temperatura e pressione tale per cui essa risulta in trasizione tra gli stati solido liquido e gassoso simultaneamente.\\
Questo stato si chiama \textbf{punto triplo} e i valori in questione sono una temperatura di $0.01 ^\circ \mathrm{C}$ e una pressione di $0.006\ \mathrm{atm}$.}
\end{fact}
\medskip

\noindent
A bassa pressione i gas si comportano tutti allo stesso modo\footnote{rispettano l'equazione di stato $pV=f(\theta)$}.\\
Se fissiamo il volume e la quantit\`a di materia del gas possiamo definire $\theta$ in modo tale che $p=p_0(1+\al\theta)$, cio\`e poniamo 
\[\theta=\frac1\al\frac{p-p_0}{p_0}.\] 
Se imponiamo che l'acqua congeli per $\theta=0$ e evapori per $\theta=100$ allora ricaviamo $1/\al=273.15$. Notiamo inoltre\footnote{l'addizione di $\al\ii$ corrisponde alla traslazione che trasforma gradi Celsius in gradi Kelvin.}
\[\frac{p_2}{p_1}=\frac{\al\ii+\theta_2}{\al\ii +\theta_1}=\frac{\theta_2'}{\theta_1'}.\]
Possiamo dunque definire la temperatura (in Kelvin) come
\[T=\lim_{p^{(PT)}\to 0}273.16 \frac{p}{p^{(PT)}}\]
dove $p^{(PT)}$ \`e la pressione del gas nel termometro quando questo sistema \`e in equilibrio con il sistema di punto triplo con l'acqua. Il limite corrisponde a prendere gas sempre pi\`u rarefatti, cio\`e a lavorare nel limite dei gas perfetti dove vale la proporzionalit\`a sopra.
\medskip

\noindent Sfruttando questa definizione possiamo costruire un termometro a gas come in figura

[FIGURA TERMOMETRO A GAS]

\noindent Quando il gas \`e alla temperatura che vogliamo misurare, misuriamo la differenza di altezza tra il livello a contatto con il gas e il livello di controllo posto a pressione atmosferica. 
Questa differenza \`e proporzionale alla differenza di pressione e questo ci permette di ricavare la temperatura se la fissiamo per quando \`e nel punto critico.


\section{Primo principio e Definizione di calore}

\begin{fact}[Primo principio della termodinamica]
\textbf{L'energia interna di un sistema di conserva.}
\end{fact}

\begin{definition}[Calore]
Il \textbf{calore} \`e la differenza tra la variazione di energia interna e il lavoro compiuto su un sistema termodinamico, esplicitamente
\[\boxed{\Delta U=Q+W}\]
\end{definition}

\begin{remark}
Il calore e il lavoro non sono funzioni di stato, ma la loro somma s\`i.
\end{remark}

\begin{remark}[Primo principio in forma differenziale]
Scrivendo il primo principio in termini di infinitesimi troviamo
\[dU=\delta Q+\delta W,\]
in particolare per i gas ideali vale
\[dU=\delta Q-pdV.\]
\end{remark}


\begin{definition}[Caloria]
Una \textbf{caloria} \`e la quantit\`a di calore necessaria per far variare la temperatura di un grammo di acqua da $14.5^\circ\mathrm{C}$ a $15.5^\circ\mathrm{C}$.\\
In Joule si ha che
\[\boxed{1\ \mathrm{cal}=4.186\ \mathrm{J}}\]
\end{definition}


\begin{remark}
In una trasformazione adiabatica, il lavoro \`e dato dalla differenza di energia interna.
\end{remark}
\begin{example}[Coppia di sistemi dentro un contenitore adiabatico]
Consideriamo due sistemi $A$ e $B$ dentro un contenitore adiabatico. Per il primo principio
\[0=\Delta U=\Delta U_A+\Delta U_B=Q_A+Q_B+\under{=W}{W_A+W_B}.\]
I trasferimenti di calore possono avvenire solo tra $A$ e $B$, quindi $Q_A+Q_B=0$ e $W=0$. Quanto scritto \`e una ``legge di conservazione del calore" in questo tipo di sistema.
\end{example}
\include{02CicliSecondoPrincipio}
\chapter{Trasferimento di calore}

\section{Modalit\`a di trasferimento di calore}
Il trasperimento di calore, cio\`e di energia derivante da una differenza di temperatura, avviene in tre modi: conduzione, covezione ed irraggiamento.

\subsection{Conduzione}
Parliamo di \textbf{conduzione} quando il tresferimento di calore avviene per contatto ma senza scambio di materia (attraverso una parete diatermica).\medskip

\noindent Empiricamente riscontriamo
\begin{fact}[Legge di Fourier]
Vale la relazione
\[\frac1A\frac{\delta Q}{\Delta t}=-\kappa\frac{\Delta T}{\Delta X},\]
dove $T$ \`e la temperatura, $X$ \`e la distanza tra i punti tra cui stiamo calcolando la differenza di temperatura, $A$ \`e l'area ortogonale alla direzione lungo la quale si propaga il calore e $\kappa$ \`e una costante detta \textbf{conducibilit\`a termica}.
\end{fact}

\noindent L'unit\`a di misura della conducibilit\`a termica \`e
\[[\kappa]=\frac W{mK}\approx \begin{cases}
10^2 &\text{metalli}\\
0.1 &\text{gas}
\end{cases}.\]
\noindent Possiamo precisare la legge di Fourier introducendo la
\textbf{corrente di calore} $\vec J_Q$. La legge assume la forma
\[\vec J_Q=-k\vec \nabla T.\]
Concentrandosi su uno dei sistemi possiamo scrivere
\[\boxed{\delta Q= cm\delta T}\]
dove $m$ \`e la massa e $c$ \`e il \textbf{calore specifico}.\bigskip

\noindent Possiamo calcolare il calore totale che entra dentro una superficie per unit\`a di tempo come
\[\int_V c\pp tT\rho dV=\frac1{\Delta t}\int_{\del V} \delta Q=-\int_{\del V} \vec J_Q\cdot \vec d\Sigma=-\int_V \nabla \cdot \vec J_Q dV=\int_Vk\nabla^2 T dV.\]
Ricaviamo dunque
\[\boxed{\pp tT=\frac{\kappa}{\rho c}\nabla^2T}\]
Questa \`e la famosa \textit{equazione del calore}.

\subsection{Convezione}
Parliamo di \textbf{convezione} quando il trasferimento di calore avviene tramite lo spostamento di materia.\\ La formula rilevante in questo caso \`e
\[\frac1A\frac{\delta Q}{\Delta t}=h\Delta T,\]
dove $h$ \`e il \textbf{coefficiente convettivo}.

\subsection{Irraggiamento}
Parliamo di \textbf{irraggiamento} quando un corpo semplicemente emette energia come radiazione.\\ 
La formula rilevante in questo caso \`e
\[\frac1A\frac{\delta Q}{\Delta t}=\e \sigma(T^4-T_0^4),\]
dove $T_0$ \`e la temperatura dell'ambiente, $\sigma$ \`e una costante uguale per tutti i materiali e $\e$ dipende dai materiali.

\section{Capacit\`a termica}
\begin{definition}[Capacit\`a termica]
Definiamo la \textbf{capacit\`a termica} come\footnote{Nota che NON \`e una derivata in quanto $Q$ non \`e una funzione di stato, quindi in particolare non \`e una funzione di $T$}
\[C=\lim_{\delta T\to 0}\frac{\delta Q}{\delta T}.\]
L'unit\`a di misura \`e $[C]=\mathrm{J}/\mathrm{K}$.\\
La \textbf{capacit\`a termica molare} \`e data da $c=C/n$.\\
Il \textbf{calore specifico} \`e dato da $C/m$.
\end{definition}

\begin{definition}[Termometro e Termostato]
Un \textbf{termostato} \`e un oggetto ideale con capacit\`a termica infinita\footnote{intuitivamente \`e un sistema grande a sufficienza in modo che anche se viene aggiunto calore, la temperatura non cambia.}.\\ 
Un \textbf{termometro} \`e un oggetto ideale con capacit\`a termica nulla\footnote{intuitivamente \`e un sistema piccolo a sufficienza in modo da poter trascurare gli scambi di calore.}.
\end{definition}

\begin{remark}
Possiamo scrivere la capacit\`a termica in termini di $U,\ V,\ p$ e $T$ come segue:
\[C=\frac{\delta Q}{\delta T}=\ppb TUV+\pa{\ppb VUT+p}\dd TV\]
\end{remark}
\begin{proof}
Sviluppando $dU$ troviamo
\[dU=\ppb TUVdT+\ppb VUTdV,\]
da cui
\[\delta Q=dU+pdV=\ppb TUVdT+\pa{\ppb VUT+p}dV.\]
Ora possiamo ``dividere" per $dT$ e trovare la tesi.
\end{proof}

\begin{definition}[Capacit\`a termica a volume/pressione costante]
Definiamo la \textbf{capacit\`a termica a volume} (risp. \textbf{pressione}) \textbf{costante} come le due seguenti quantit\`a
\begin{align*}
C_V=&\rbar{\frac{\delta Q}{\delta T}}_V=\ppb TUV\\
C_p=&\rbar{\frac{\delta Q}{\delta T}}_p=\ppb TUV +\pa{\ppb VUT+p}\ppb TVp=\ppb TUV +\pa{\ppb VUT+p}V\al
\end{align*}
\end{definition}

\begin{remark}[Disuguaglianza tra capacit\`a termiche]\label{DisguguaglianzaCapacitaTermiche}
Vale sempre $C_p>C_V$.
\end{remark}

\begin{remark}\label{DerivataEnergiaInternaRispettoAlVolume}
In un gas generale
\[\boxed{\ppb VUT=\frac{C_p-C_V}{V\al}-p}\]
\end{remark}


\chapter{Sistemi idrostatici}
Definiamo informalmente un sistema idrostatico come un sistema termodinamico determinato da pressione, volume e temperatura.


\begin{definition}[Principali processi quasistatici per sistemi idrostatici]
Un processo si dice
\begin{itemize}
\item \textbf{isotermo} se $T$ resta costante,
\item \textbf{isobaro} se $p$ resta costante,
\item \textbf{isocore} se $V$ resta costante o
\item \textbf{adiabatico} se non avviene scambio di calore.
\end{itemize}
\end{definition}

\section{Differenziale di volume}
\begin{definition}[Coefficiente di espansione volumetrica]
Definiamo il \textbf{coefficiente di espansione volumetrica} come
\[\al=\frac1V\ppb TVp=-\frac mV\frac1{\rho^2}\ppb T\rho p=-\frac1\rho\ppb T\rho p.\]
L'unit\`a di misura \`e $[\al]=\mathrm{K}\ii$.
\end{definition}


\begin{definition}[Compressibilit\`a isoterma]
Definiamo la \textbf{compressibilit\`a isoterma} come
\[\beta_T=-\frac1V\ppb pVT.\]
L'unit\`a di misura \`e $[\beta_T]=\mathrm{Pa}\ii$.\\
L'inversa $k_T=1/\beta_T$ \`e detta \textbf{modulo di compressibilit\`a isoterma}.
\end{definition}

\noindent Riportiamo alcuni valori di $\al$ e $\beta_T$ per dare una intuizione sui valori tipici\footnote{il Sitall \`e materiale fatto apposta per avere coefficiente di espansione volumetrica piccolo}
\begin{center}
\begin{tabular}[ht]{|c|c|c|}
\hline 
Materiale&$\al\ [\mathrm{K}\ii]$&$\beta_T\ [\mathrm{Pa}\ii]$\\\hline
Acqua&$0.2\cdot 10^{-3}$&$4.6\cdot 10^{-10}$\\
Diamante&$3\cdot 10^{-6}$&?\\
Sitall&$\leq 10^{-7}$&?\\
Sabbia&?&$\sim10^{-8}$\\
Mercurio&$1.8\cdot 10^{-4}$&$4\cdot10^{-11}$\\
Rame&?&$7.2\cdot10^{-12}$\\
\hline
\end{tabular}
\end{center}


\begin{remark}
Non \`e necessario battezzare $\displaystyle\ppb TpV$ in quanto per la propriet\`a ciclica (\ref{ProprietaDerivateParziali})
\[\ppb TpV=-\ppb VpT\ppb TVp=\frac\al{\beta_T}.\]
\end{remark}

\begin{remark}[Relazione differenziale tra $\al$ e $\beta_T$]
Per il teorema di Schwarz si ha che
\[\pp{p\del T}{^2V}=\ppb p\al T=-\ppb T{\beta_T}p.\]
\end{remark}

\begin{remark}[Differenziale del volume]
Dalle definizioni date segue che
\[dV=\al VdT-\beta_T Vdp.\]
\end{remark}

\begin{proposition}[Differenziale della pressione]\label{DifferenzialePressione}
Si ha che
\[dp=\frac\al{\beta_T}dT-\frac1{\beta_T V}dV.\]
\end{proposition}
\begin{proof}
Osserviamo che
\[\ppb TpV\pasgnl={(\ref{ProprietaDerivateParziali})}-\ppb VpT \ppb TVp=\frac \al{\beta_T},\]
dunque ricaviamo
\[dp=\ppb TpVdT+\ppb VpT=\frac\al{\beta_T}dT-\frac1{\beta_T V}dV.\]
\end{proof}
\begin{corollary}
In una trasformazione isocora $\Delta p=\frac\al{\beta_T}\Delta T$.
\end{corollary}

\begin{remark}[Differenziale logaritmico nel volume]\label{DifferenzialeLogaritmicoNelVolume}
Spesso torner\`a comodo ricordare il seguente sviluppo differenziale
\[d\log V=\frac1VdV=\al dT-\beta_T dp\]
\end{remark}
\begin{proof}
Segue calcolando:
\[\frac1VdV=\frac1V\pa{\ppb TVp dT+\ppb pVTdp}=\al dT-\beta_T dp\]
\end{proof}

\section{Lavoro per sistema idrostatico}
Immaginiamo di comprimere un sistema idrostatico come in figura 

[FIGURA]

\noindent
Se spingiamo molto lentamente possiamo con buona approssimazione supporre che il processo sia quasistatico, dunque $F=pS$. Segue che
\[\boxed{\delta W=Fdx=pSdx}\]
Se il sistema in questione \`e un gas ideale allora
\[\delta W=p(-dV)=-pdV\]
Il lavoro totale per passare da uno stato $A$ ad uno stato $B$ diventa
\[W=-\int_{A}^{B} p(V,T)dV,\]
ma $p$ come cambia al variare di $V$? Dipende dal tipo di processo.

[QUALCHE GRAFICO]

\noindent
Questo mostra in particolare che il lavoro non \`e una funzione di stato.

\section{Capacit\`a termica}
\begin{definition}[Capacit\`a termica]
Definiamo la \textbf{capacit\`a termica} come\footnote{Nota che NON \`e una derivata in quanto $Q$ non \`e una funzione di stato, quindi in particolare non \`e una funzione di $T$}
\[C=\lim_{\delta T\to 0}\frac{\delta Q}{\delta T}.\]
L'unit\`a di misura \`e $[C]=\mathrm{J}/\mathrm{K}$.\\
La \textbf{capacit\`a termica molare} \`e data da $c=C/n$.\\
Il \textbf{calore specifico} \`e dato da $C/m$.
\end{definition}

\begin{definition}[Termometro e Termostato]
Un \textbf{termostato} \`e un oggetto ideale con capacit\`a termica infinita\footnote{intuitivamente \`e un sistema grande a sufficienza in modo che anche se viene aggiunto calore, la temperatura non cambia.}.\\ 
Un \textbf{termometro} \`e un oggetto ideale con capacit\`a termica nulla\footnote{intuitivamente \`e un sistema piccolo a sufficienza in modo da poter trascurare gli scambi di calore.}.
\end{definition}

\begin{remark}
Possiamo scrivere la capacit\`a termica in termini di $U,\ V,\ p$ e $T$ come segue:
\[C=\frac{\delta Q}{\delta T}=\ppb TUV+\pa{\ppb VUT+p}\dd TV\]
\end{remark}
\begin{proof}
Sviluppando $dU$ troviamo
\[dU=\ppb TUVdT+\ppb VUTdV,\]
da cui
\[\delta Q=dU+pdV=\ppb TUVdT+\pa{\ppb VUT+p}dV.\]
Ora possiamo ``dividere" per $dT$ e trovare la tesi.
\end{proof}

\begin{definition}[Capacit\`a termica a volume/pressione costante]
Definiamo la \textbf{capacit\`a termica a volume} (risp. \textbf{pressione}) \textbf{costante} come le due seguenti quantit\`a
\begin{align*}
C_V=&\rbar{\frac{\delta Q}{\delta T}}_V=\ppb TUV\\
C_p=&\rbar{\frac{\delta Q}{\delta T}}_p=\ppb TUV +\pa{\ppb VUT+p}\ppb TVp=\ppb TUV +\pa{\ppb VUT+p}V\al
\end{align*}
\end{definition}

\begin{remark}[Disuguaglianza tra capacit\`a termiche]\label{DisguguaglianzaCapacitaTermiche}
Vale sempre $C_p>C_V$.
\end{remark}

\begin{remark}\label{DerivataEnergiaInternaRispettoAlVolume}
In un gas generale
\[\boxed{\ppb VUT=\frac{C_p-C_V}{V\al}-p}\]
\end{remark}

\subsection{Processi politropici}
Possiamo generalizzare i quattro tipi di processi principali come segue:
\begin{definition}[Processo politropico]
Un processo \`e \textbf{politropico} se la capacit\`a termica \`e costante.
\end{definition}

\begin{proposition}[Curve per processi politropici]\label{CurveProcessiPolitropici}
Considerando un processo politropico relativo ad un gas ideale e definiamo
\[\delta=\frac{C_p-C}{C_V-C},\]
allora seguendo questo processo si ha che $pV^\delta=cost.$.
\end{proposition}
\begin{proof}
Poich\'e $\delta Q=CdT=C_VdT+pdV=C_pdT-Vdp$ ricaviamo che
\[-\frac V{C-C_p}dp=dT=\frac p{C-C_V}dV,\]
da cui
\[-\frac Vp\dd Vp=\frac{C_p-C}{C_V-C}=\delta.\]
Questa espressione restituisce una equazione differenziale
\[-\frac{dp}p=\delta\frac{dV}V,\]
la cui soluzioni hanno la forma voluta.
\end{proof}

\noindent
Possiamo interpretare processi isocori, isobari, isotermi e adiabatici come processi politropici:
\begin{center}
\begin{tabular}[ht]{|c||c|c|c|c|}
\hline
Processo & Isocoro & Isobaro & Isotermo & Adiabatico\\\hline&&&&\\
$\delta$ & $\infty$ & $0$ & $1$ & $\displaystyle\gamma=\frac{c_p}{c_V}$\\ &&&&\\\hline
\end{tabular}
\end{center}



\section{Potenziali termodinamici}
\subsection{Energia interna ed entropia}

\begin{remark}[Differenziali]
Osserviamo che
\[TdS=\rbar{\delta Q}_{rev}=dU-\rbar{\delta W}_{rev}=dU+pdV,\]
dunque
\[\boxed{dU=TdS-pdV}\]
\end{remark}
\begin{remark}
\underline{\textbf{NOTA BENE:}} Questo sembra un altro modo di scrivere il primo principio $dU=\delta Q+\delta W$, ma se il processo non \`e reversibile allora non \`e detto che $pdV=\delta W$.
\end{remark}

\begin{proposition}[Equazione di Helmholtz]\label{EquazioneHelmholtz}
Vale la seguente identit\`a
\[\boxed{\ppb VUT=T\ppb TpV-p}=T^2\ppb T{(p/T)}V\]
\end{proposition}
\begin{proof}
Sviluppiamo il differenziale dell'entropia
\[dS=\frac{dU}T+\frac{pdV}T=\frac1T\pa{\ppb TUVdT+\ppb VUTdV}+\frac pTdV.\]
Poich\'e $dS$ \`e un differenziale esatto, le derivate incrociate devono coincidere:
\begin{gather*}
\pp V{}\pa{\frac1T\ppb TUV}=\rbar{\pp T{}\pa{\frac1T\pa{\ppb VUT+p}}}_V\\
\cancel{\frac1T\pp{V\del T}{^2U}}=-\frac1{T^2}\pa{\ppb VUT+p}+\cancel{\frac1T\pp{V\del T}{^2U}}+\frac1T\ppb TpV\\
\ppb VUT=T\ppb TpV-p.
\end{gather*}
\end{proof}
\begin{remark}
Osserviamo che 
\[dU=\under{=C_V}{\ppb TUV}dT+\ppb VUTdV,\]
quindi la formula di Helmholtz misura ``quanto un gas non \`e ideale" (ricordiamo (\ref{DerivataEnergiaInternaRispettoAlVolume}))
\end{remark}
\begin{remark}
Se $dU=0$ allora
\[dT=-\frac{\ppb VUT}{\ppb TUV}dV=\ppb VTUdV\pasgnl={(\ref{EquazioneHelmholtz})}\frac1{C_V}\pa{p-T\ppb TpV}dV.\]
Questa identit\`a pu\`o essere interpretata come test per Gas ideali, basta capire se $T\ppb pTV$ coincide con $\frac{nRT}V$.
\end{remark}

\begin{remark}
Valgono le seguenti identit\`a
\[C_V=T\ppb TSV,\quad C_p=T\ppb TSp,\qquad \gamma=\frac{\ppb VpS}{\ppb Vpp}=\frac{\beta_T}{\beta_S}\]
\end{remark}
\begin{proof}
ESERCIZIO
\end{proof}





\subsection{Entalpia}



\begin{definition}[Entalpia]
L'\textbf{entalpia} in un gas \`e definita da\footnote{cio\`e l'energia interna sommata al ``lavoro necessario per portare il volume da $0$ a $V$ a pressione costante".}
\[H=U+pV.\]
\end{definition}

\begin{remark}
L'entalpia \`e ottenuta da $U$ come trasformazione di Legendre:\\
Consideriamo $U(S,V)$ e cambiamo la coordinata $V$ in $-p=\ppb VUS$, che possiamo fare ponendo $H(S,p)=U-(-pV)=U+pV$.
\end{remark}

\begin{remark}[Differenziale dell'entalpia]
Il differenziale di $H$ \`e dato da
\[dH=dU+pdV+Vdp=TdS+Vdp\pasgnl={se reversibile}\delta Q+Vdp\]
oppure da
\[dH=dU+p(\al VdT-\beta_T Vdp)+Vdp=dU+p\al VdT+V(1-\beta_Tp)dp.\]
\end{remark}
\begin{corollary}
Valgono le seguenti identit\`a
\[\ppb SHp=T,\quad \ppb pHS=V,\quad \ppb THp=C_p.\]
\end{corollary}
\begin{proof}
Le prime due seguono immediatamente da $dH=TdS+Vdp$, mentre l'ultima segue sviluppando la definizione di $C_p$:
\[C_p=\rbar{\frac{\delta Q}{\delta T}}_p=\ppb TUp+p\ppb TVp=\rbar{\pp T{}(U+pV)}_p=\ppb THp.\]
\end{proof}




\begin{definition}[Coefficiente di Joule-Thomson]
Definiamo il \textbf{coefficiente di Joule-Thomson} come
\[\mu_{JT}=\ppb pTH.\]
\end{definition}
\begin{fact}
Per ogni gas esiste una temperatura, detta \textbf{temperatura di inversione}, tale che sotto questa temperatura $\mu_{JT}>0$
\end{fact}




\subsection{Energia libera di Helmholtz e Gibbs}
\noindent
Diamo un nome alle rimanenti trasformazioni di Legendre di $U$:
\begin{definition}[Energia libera di Helmholtz]
Dall'energia interna cambiamo $S$ in $\ppb SUV=T$, definendo $F=U-TS$, detta \textbf{energia libera di Helmholtz}. Osserviamo che
\[dF=-SdT-pdV.\]
\end{definition}

\begin{remark}
L'energia libera di Helmholtz ci permette di ricavare il ``massimo lavoro estraibile tra due stati".
\end{remark}
\begin{proof}
Per il secondo principio $Q/T\leq \Delta S$, dunque
\[W=\Delta U-Q\geq \Delta U-T\Delta S=\Delta F,\]
cio\`e il lavoro che pu\`o fare il sistema ($-W$) \`e al massimo $-\Delta F$.
\end{proof}

\begin{remark}
Si ha che
\[\ppb TFV=-S,\quad \ppb VFT=-p.\]
\end{remark}

\begin{definition}[Energia libera di Gibbs]
Dall'entalpia cambiamo $S$ in $\ppb SHp=T$, definendo $G=H-TS=U+pdV-TS=F+pV$, detta \textbf{energia libera di Gibbs}. Osserviamo che
\[dG=-SdT-Vdp.\]
\end{definition}

\begin{remark}
Si ha che
\[\ppb pGT=V,\quad \ppb TGp=-S.\]
\end{remark}

\begin{proposition}[Secondo principio per energia libera di Gibbs]\label{GibbsDiminuiscePerPressioneETemperaturaCostanti}
Se $T$ e $p$ sono costanti l'energia libera di Gibbs tende a diminuire.
\end{proposition}
\begin{proof}
Per il secondo principio $TdS-\delta Q\geq 0$. Sviluppando $\delta Q$ troviamo
\[TdS-dU-pdV\geq0.\]
Se $p$ \`e costante allora $dH=dU+pdV$, quindi in tal caso
\[TdS-dH\geq 0\]
Se ora $T$ resta costante si ha che $dG=dH-TdS$, cio\`e
\[dG\leq 0.\]
Segue dunque che se $T$ e $p$ sono costanti, $G$ tende a diminuire.
\end{proof}


\begin{proposition}[Relazioni di Maxwell]\label{RelazioniMaxwell}
Valgono anche le equazioni
\[\ppb VTS=-\ppb SpV,\quad \ppb pTS=\ppb SVp,\quad \ppb VST=\ppb TpV,\quad \ppb pST=-\ppb TVp.\]
\end{proposition}
\begin{proof}
Derivano dall'eguagliare derivate seconde delle energie che abbiamo definito. In ordine le quattro equazioni sono uguali a
\[\pp[2]{V\del S}U,\quad \pp[2]{p\del S}H,\quad -\pp[2]{V\del T}F,\quad \pp[2]{p\del T}G.\]
\end{proof}

\begin{remark}[Jacobiano $p,V$ - $T,S$]
Possiamo derivare le relazioni di Maxwell anche constratando che
\[\pp{(p,V)}{(T,S)}=1.\]
\end{remark}
\begin{proof}
Mostriamo che l'identit\`a vale e poi ricaviamo le relazioni da essa.
\setlength{\leftmargini}{0cm}
\begin{itemize}
\item[$\boxed{dPdV=dTdS}$] Per il primo principio
\[dU=TdS-PdV\leadsto 0=d^2U=dTdS-dPdV,\]
segue dunque per unicit\`a della scrittura in base che lo Jacobiano in esame vale $1$.
\item[$\boxed{\text{Relazioni}}$] Ricaviamo la prima, le altre seguono in modo analogo
\[\ppb VTS=\pp{(V,S)}{(T,S)}=\pa{\pp{(p,V)}{(T,S)}}\ii\pp{(V,S)}{(T,S)}=\pp{(V,S)}{(p,V)}=-\pp{(S,V)}{(p,V)}=-\ppb SpV.\]
\end{itemize}
\setlength{\leftmargini}{0.5cm}
\end{proof}




\subsection{Riassunto}
Riassumiamo il comportamento di queste quantit\`a nella seguente tabella:
\begin{center}
\begin{tabular}{|c||c|c|c|}
\hline
Energia & Definizione &Differenziale & Rel. Maxwell\\\hline
Interna & $\Delta U=Q+W$ & $+TdS-p\ dV$ & $\ppb VTS=-\ppb SpV$\\
Entalpia & $H=U+pV$ & $+TdS+Vdp$ & $\ppb pTS=\ppb SVp$\\
Helmholtz & $F=U-TS$ & $-SdT-p\ dV$ & $\ppb VST=\ppb TpV$\\
Gibbs & $G=U+pV-TS$ & $-SdT+Vdp$ & $-\ppb pST=\ppb TVp$\\\hline
\end{tabular}
\end{center}
Mettendo a confronto energia interna ed entalpia troviamo ulteriori similitudini:
\begingroup
\renewcommand{\arraystretch}{1.5}
\begin{center}
\begin{tabular}{|c||c||c|}
\hline
Argomento & Energia interna & Entalpia\\\hline\hline
Espansione & libera di Joule & di Joule-Thompson\\\hline
Quanto non ideale?& $\mu_J=\ppb VTU$ & $\mu_{JT}=\ppb pTH$\\\hline
Differenziale reversibile &$dU=\delta Q-pdV$ & $dH =\delta Q+Vdp$\\\hline
C. di Esp. Termica&$C_V=\ppb TUV$ & $C_p=\ppb THp$\\\hline
Su isocora & $\Delta U=Q$ & $\Delta H=Q$\\\hline
Su adiabatica & $\Delta U=-\int pdV$ &  $\Delta H=\int Vdp$\\\hline
In gas ideale & $\Delta U=\int c_VdT$ & $\Delta H=\int c_pdT$\\\hline
\end{tabular}
\end{center}
\endgroup

\bigskip
\noindent
Per ricordare le varie derivate parziali pu\`o essere utile il \textbf{diagramma di Konis-Born}
% https://q.uiver.app/#q=WzAsNCxbMCwwLCJUIl0sWzEsMCwiViJdLFswLDEsIlAiXSxbMSwxLCJTIl0sWzIsMywiSCIsMix7InN0eWxlIjp7ImhlYWQiOnsibmFtZSI6Im5vbmUifX19XSxbMCwxLCJGIiwwLHsic3R5bGUiOnsiaGVhZCI6eyJuYW1lIjoibm9uZSJ9fX1dLFsxLDMsIlUiLDAseyJzdHlsZSI6eyJoZWFkIjp7Im5hbWUiOiJub25lIn19fV0sWzAsMiwiRyIsMix7InN0eWxlIjp7ImhlYWQiOnsibmFtZSI6Im5vbmUifX19XSxbMiwxLCIiLDMseyJsYWJlbF9wb3NpdGlvbiI6NjAsInN0eWxlIjp7InRhaWwiOnsibmFtZSI6Im1vbm8ifX19XSxbMywwLCIiLDEseyJzdHlsZSI6eyJ0YWlsIjp7Im5hbWUiOiJtb25vIn19fV1d
\[\begin{tikzcd}
	T & V \\
	p & S
	\arrow["H"', no head, from=2-1, to=2-2]
	\arrow["F", no head, from=1-1, to=1-2]
	\arrow["U", no head, from=1-2, to=2-2]
	\arrow["G"', no head, from=1-1, to=2-1]
	\arrow[tail, from=2-1, to=1-2]
	\arrow[tail, from=2-2, to=1-1]
\end{tikzcd}\]
I due vertici di uno stesso lato contengono le variabili che appaiono nei differenziali mentre le diagonali indicano il valore della derivata dell'energia di un dato lato per la variabile del vertice. Il verso delle diagonali indica il segno di queste derivata, quindi per esempio per il lato destro ricaviamo
\[\ppb SUV=T,\quad \ppb VUS=-p.\]
Le due coppie di lati opposti corrispondono ognuna a due relazioni di Maxwell. Per capire i segni basta considerare i due triangoli che si formano scegliendo uno degli altri lati e completando con le due diagonali, per esempio la relazione
\[\ppb VTS=-\ppb SpV\]
si pu\`o leggere associando al membro di sinistra il triangolo $TVS$ e a quello di destra il triangolo $pSV$.

\section{Sistemi aperti di gas}
Per sistemi di gas aperto
\[\Delta U=Q+W+\Delta U_{materia},\]
o in forma differenziale
\[dU=TdS-pdV+\mu dn\]
dove $\mu$ \`e il \textbf{potenziale chimico}.
\begin{remark}
Valgono le identit\`a differenziali
\[\ppb SU{V,n}=T,\qquad \ppb VU{S,n}=-p,\qquad \ppb nU{S,V}=\mu.\]
\end{remark}

\chapter{Gas e Gas ideali}

\section{Definizioni e legge dei gas perfetti}
\begin{definition}[Mole]
Una \textbf{mole di una sostanza} corrisponde a $6.02\cdot 10^{23}$ particelle di quella sostanza. La costante \`e detta \textbf{numero di Avogadro} e la indichiamo con $N_a$. 
\end{definition}

\begin{definition}[Densit\`a]
Definiamo la \textbf{denstit\`a} come
\[\rho=\frac mV.\]
\end{definition}
\begin{remark}
Il differenziale della densit\`a \`e
\[d\rho=-\frac m{V^2}dV.\]
\end{remark}

\begin{definition}[Condizioni standard]
Un gas \`e in \textbf{condizioni standard} (\textbf{STP}) se \`e alla temperatura di $0^\circ\mathrm C$ e alla pressione di $1\ \mathrm{atm}=101.3245\ \mathrm{kPa}$.
\end{definition}

\noindent Per i gas ideali valgono le seguenti leggi:
\begin{fact}[Legge di Boyle]
Se $T$ \`e costante
\[V\propto \frac1p\]
\end{fact}
\begin{fact}[Legge di Charles]
Se $p$ \`e costante
\[V\propto (1+\al T)\]
\end{fact}
\begin{fact}[Legge di Gay-Lussac]
Se $V$ \`e costante
\[p\propto T\]
\end{fact}
\begin{fact}[Legge di Avogadro]
Se $p$ e $T$ sono fissate, tutti i gas occupano lo stesso volume se consistono della stessa quantit\`a di materia, in particolare
\[V\propto n.\]
Una mole di gas in condizioni standard occupa un volume di $22.4\ell$ (litri).
\end{fact}

\noindent Combinando le leggi appena citate arriviamo alla \textbf{legge dei Gas perfetti}
\[\boxed{pV=nRT}\]
dove $p$ \`e la pressione, $V$ \`e il volume, $n$ \`e il numero di moli, $T$ \`e la temperatura e $R$ \`e la \textbf{costante fondamentale dei gas} e vale $8.314 \frac{\mathrm{J}}{\mathrm{K}\ \mathrm{mol}}$.

\begin{definition}[Costante di Boltzmann]
Definiamo la \textbf{costante di Boltzmann} $k_b$ in modo tale che 
\[R=N_a k_b.\]
\end{definition}

\section{Coefficiente di espansione volumetrica e compressibilit\`a isoterma}

\begin{proposition}[$\al$ e $\beta_T$ per gas ideali]\label{CoefficienteEspansioneECompressibilitaIsotermaPerGasIdeali}
Se il sistema in esame \`e un gas ideale valgono le seguenti identit\`a:
\[\al=\frac 1T,\qquad \beta_T=\frac 1p.\]
\end{proposition}
\begin{proof}
Segue calcolando:
\begin{align*}
\al=&\frac1V\ppb T{(nRT/p)}p=\frac{nR}{pV}=\frac1T,\\
\beta_T=&-\frac1V\ppb p{(nRT/p)}T=\frac1V nRT\frac1{p^2}=\frac1p.
\end{align*}
\end{proof}

\section{Capacit\`a termica}

\begin{definition}[Coefficiente di Joule]
Definiamo il \textbf{coefficiente di Joule} come
\[\mu_J=\ppb VTU\]
\end{definition}

\begin{fact}[In gas ideale l'energia interna dipende solo dalla temperatura]\label{InGasIdealeUdipendeSoloDaT}
In un gas ideale $U$ dipende solo da $T$.
\end{fact}
\begin{proof}[Esperimento: Espansione libera adiabatica di Joule]
Consideriamo un contenitore adiabatico separato internamente da una parete adiabaita. In uno dei due volumi si trova un gas ideale, il secondo \`e vuoto.

[DISEGNO]

\noindent
Improvvisamente eliminiamo la parete interna e lasciamo che il gas si espanda\footnote{notiamo che questo NON \`e una processo quasistatico.}.\smallskip

\noindent
Chiaramente $Q=W=0$ in quanto il vuoto non subisce/effettua lavoro e non scambia calore, dunque $\Delta U=0$.\\
Segue che $\mu_J=\ppb VTU=\dd VT$ e Joule ha misurato che in queste circostanze la seconda \`e nulla, dunque
\[0=\ppb VTU\pasgnl={(\ref{ProprietaDerivateParziali})}-\pa{\ppb UVT}\ii\pa{\ppb TUV}\ii=-\ppb VUT\frac1{C_V},\]
in particolare $\displaystyle\ppb VUT=0$.\medskip

\noindent Poich\'e in un gas ideale $p$ \`e determinata da $V$ e $T$, $U=U(V,T)$. Per quanto appena detto $U$ non dipende da $V$, quindi dipende solo da $T$.
\end{proof}


\begin{corollary}
In un gas ideale, a prescindere dal tipo di processo,
\[\boxed{dU=C_V dT}\]
\end{corollary}
\begin{proof}
Ricordiamo che
\[C_V=\ppb TUV,\]
ma poich\'e $U$ non dipende da $V$ possiamo scrivere
\[C_V=\dd TU,\]
che \`e la tesi.
\end{proof}

\begin{proposition}[Relazione di Mayer]\label{RelazioneMayer}
Per gas ideali si ha che $c_p-c_V=R$, o equivalentemente $C_p-C_V=nR$.
\end{proposition}
\begin{proof}
Ricordiamo (\ref{CoefficienteEspansioneECompressibilitaIsotermaPerGasIdeali}) che per gas ideali $\al=T\ii$. Poich\'e $U$ dipende solo da $T$ si ha che
\[0=\ppb VUT\pasgnl={(\ref{DerivataEnergiaInternaRispettoAlVolume})}\frac{C_p-C_V}{V\al}-p,\]
da cui
\[C_p-C_V=pV\al=\frac{nRT}T=nR.\]
\end{proof}

\begin{notation}
Denotiamo il rapporto $\frac{C_V}{C_p}=\frac{c_V}{c_p}$ con $\gamma$.
\end{notation}



\begin{fact}[Calore specifico a volume costante in funzione dei gradi di libert\`a]
In un gas ideale
\[C_V=\frac\nu2nR\]
dove $\nu$ \`e il \textbf{numero di gradi di libert\`a}.
\end{fact}
\begin{remark}
Per un gas ideale monoatomico $\nu=3$, mentre per un gas biatomico $\nu=5$.\\
Segue che
\[c_V^{mono}=\frac32R\approx 12.47\frac{\mathrm{J}}{\mathrm{K\ mol}},\qquad c_V^{bi}=\frac52R\approx 20.74\frac{\mathrm{J}}{\mathrm{K\ mol}}.\]
Da queste scritture segue anche che
\[c_p^{mono}=\frac52R,\quad \gamma^{mono}=\frac53,\quad\qquad c_p^{bi}=\frac72R,\quad \gamma^{bi}=\frac75.\]
\end{remark}

\begin{remark}[L'aria \`e un gas ideale biatomico]
L'aria \`e composta principalmente da particelle biatomiche ($O_2$ e $N_2$).
\end{remark}

\begin{proposition}[Calore infinitesimale con capacit\`a]\label{CaloreInfinitesimaleConCapacita}
Per gas ideali valgono le seguenti equazioni
\begin{enumerate}
\item $\delta Q=C_VdT+pdV$
\item $\delta Q=C_p dT-Vdp$.
\end{enumerate}
\end{proposition}
\begin{proof}
Mostriamo i due punti:
\setlength{\leftmargini}{0cm}
\begin{itemize}
\item[$\boxed{1}$] Ricordiamo la relazione \[\delta Q=\under{=C_V}{\ppb TUV} dT+\pa{{\ppb VUT}+p}dV,\]
da cui, usando il fatto che $\ppb UVT=0$, troviamo che $\delta Q=C_V dT+pdV$.
\item[$\boxed{2}$] Osserviamo che il differenziale di $pV=nRT$ \`e
\[nRdT=pdV+Vdp,\]
da cui sfruttando la relazione precedente
\[\delta Q=C_VdT+pdV=({C_V+nR})dT-Vdp\pasgnl={(\ref{RelazioneMayer})}C_p dT-Vdp.\]
\end{itemize}
\setlength{\leftmargini}{0.5cm}
\end{proof}
\begin{remark}
Osservando la prima equazione ricaviamo nuovamente che $\delta Q$ non \`e un differenziale, infatti se lo fosse avremmo il seguente assurdo:
\[0=\ppb V{C_V}T=\ppb TpV=\frac{nR}V\neq 0.\]
\end{remark}


\section{Energia interna, lavoro e calore}

\noindent In questa sezione calcoliamo lavoro, calore e variazione di energia interna per i tipi principali di processi quasistatici.\medskip

\noindent Notiamo che $\Delta U=nc_V\Delta T$ in ogni circostanza in quanto $U$ non dipende da $V$.
\subsection{Isobara}
\begin{proposition}[Energie per isobara]\label{EnergieIsobara}
Per una trasformazione isobara valgono le seguenti identit\`a:
\[W=-nR\Delta T,\quad
Q=nc_p\Delta T,\quad
\Delta U=nc_V\Delta T.\]
\end{proposition}
\begin{proof}
Calcoliamo:
\begin{align*}
W=&-\int_{V_i}^{V_f}pdV\pasgnl={isobara}-p\Delta V\pasgnl={gas ideale}-nR\Delta T\\
Q\pasgnl={isobara}&\int_{T_i}^{T_f}nc_pdT=nc_p\Delta T\\
\Delta U=&Q+W=n(c_p-R)\Delta T=nc_V\Delta T.
\end{align*}
\end{proof}

\subsection{Isocora}
\begin{proposition}[Energie per isocora]\label{EnergieIsocora}
Per una trasformazione isocora valgono le seguenti identit\`a:
\[W=0,\quad
Q=nc_v\Delta T,\quad
\Delta U=nc_V\Delta T.\]
\end{proposition}
\begin{proof}
Calcoliamo:
\begin{align*}
W=&-\int_{V_i}^{V_f}pdV\pasgnlmath={V_i=V_f}0\\
Q\pasgnl={isocora}&\int_{T_i}^{T_f}nc_VdT=nc_V\Delta T\\
\Delta U=&Q+W=nc_V\Delta T.
\end{align*}
\end{proof}

\subsection{Isoterma}
\begin{proposition}[Energie per isoterma]\label{EnergieIsoterma}
Per una trasformazione isoterma valgono le seguenti identit\`a:
\[W=-nRT\log\pa{\frac{V_f}{V_i}},\quad
Q=nRT\log\pa{\frac{V_f}{V_i}},\quad
\Delta U=0.\]
\end{proposition}
\begin{proof}
Poich\'e stiamo considerando un gas ideale \[\Delta U=nc_V\Delta T\pasgnl={isoterma}0.\] 
Per il primo principio si ha $Q=-W$, quindi per concludere basta calcolare il lavoro.
\[W=-\int_{V_i}^{V_f}pdV\pasgnl={gas ideale}-nRT\int_{V_i}^{V_f}\frac1VdV=-nRT\log\pa{\frac{V_f}{V_i}}.\]
\end{proof}

\subsection{Adiabatica}
\begin{proposition}[Equazione di stato per adiabatica]\label{EquazioneStatoAdiabatica}
Poniamo $\gamma=c_p/c_V$. Si ha che $pV^\gamma$ \`e costante seguendo un processo adiabatico.
\end{proposition}
\begin{proof}
Poich\'e il sistema in esame \`e un gas ideale valgono le seguenti uguaglianze
\[0\pasgnl={adiabatica}\delta Q=dU-\delta W\pasgnl={gas ideale}nc_VdT+pdV=\frac{nc_V}{nR}d(pV)+p dV.\]
Segue che
\[-\frac{c_VV}{\cancel{R}}dp=\pa{\frac{pc_V+pR}{\cancel{R}}}dV\pasgnl={(\ref{RelazioneMayer})}\frac{pc_p}{\cancel{R}}dV,\]
da cui
\[-\frac{dp}p=\gamma\frac{dV}V.\]
Integrando troviamo
\[-\log p+Const.=\gamma\log V\coimplies \log pV^\gamma=Const.\coimplies pV^\gamma=e^{Const.}\]
che \`e quello che volevamo mostrare.
\end{proof}
\begin{remark}
Si ha che \[c_v=\frac R{\gamma-1}.\]
\end{remark}
\begin{proof}
Per definizione di $\gamma$
\[c_v=\frac{c_p}\gamma\pasgnl={(\ref{RelazioneMayer})}\frac{R+c_v}\gamma,\]
dunque
\[\gamma c_v=c_v+R\]
e la tesi segue.
\end{proof}

\begin{proposition}[Energie per adiabatica]\label{EnergieAdiabatica}
Per una trasformazione adiabatica valgono le seguenti identit\`a:
\[W=\frac{p_fV_f-p_iV_i}{\gamma-1},\quad
Q=0,\quad
\Delta U=\frac{p_fV_f-p_iV_i}{\gamma-1}.\]
\end{proposition}
\begin{proof}
Poich\'e il processo \`e adiabatico, $Q=0$. Segue per il primo principio che $\Delta Q=W$. Dato che stiamo considerando un gas ideale
\[\Delta U=nc_V\Delta T=n\frac R{\gamma-1}\Delta T=\frac 1{\gamma-1}\Delta (pV)=\frac{p_fV_f-p_iV_i}{\gamma-1}.\]
\end{proof}

\begin{remark}
Potevamo ricavare energia e lavoro anche sfruttando la relazione \[pV^\gamma=p_iV_i^\gamma=p_fV_f^\gamma,\] ma avendola ricavata come sopra sappiamo che l'espressione \`e \textbf{valida anche per processi adiabatici NON quasistatici}.
\end{remark}

\section{Ciclo di Carnot per Gas ideali}

\begin{proposition}[Efficienza del ciclo di Carnot]\label{EfficienzaCicloCarnot}
L'efficienza di un ciclo di Carnot per gas ideali tra le temperature $T_H$ e $T_L$ \`e data da
\[\eta=1-\frac{T_L}{T_H}.\]
\end{proposition}
\begin{proof}
Calcoliamo che quantit\`a coinvolte:
\[\abs{Q_H}=Q_{AB}\pasgnl={isoterma.}-W_{AB}=\int_A^BpdV=nRT_H\log\pa{\frac{V_B}{V_A}}>0\]
\[\abs{Q_L}=-Q_{CD}\pasgnl={isoterma.}W_{CD}=-\int_C^DpdV=nRT_L\log\pa{\frac{V_C}{V_D}},\]
\[\eta=1-\frac{\abs{Q_L}}{\abs{Q_H}}=1-\frac{T_L\log(V_C/V_D)}{T_H\log(V_B/V_A)}=1-\frac{T_L}{T_H},\]
dove nell'ultimo conto abbiamo usato le equazioni per le adiabatiche:
\[\pa{\frac{V_B}{V_C}}^{\gamma-1}=\frac{T_L}{T_H},\quad \pa{\frac{V_D}{V_A}}^{\gamma-1}=\frac{T_H}{T_L}\implies \frac{V_B}{V_A}=\frac{V_C}{V_D}.\]
\end{proof}

\begin{remark}[Efficienza massima per gas ideale]
Poich\'e il ciclo di Carnot \`e reversibile, per il teorema di Carnot (\ref{TeoremaDiCarnot}) il valore
\[1-\frac{T_L}{T_H}\]
\`e la massima efficienza possibile per un qualsiasi ciclo realizzato da un gas ideale.
\end{remark}

\begin{remark}[Coefficiente di prestazione massimo per gas ideale]
Per quanto detto il coefficente di prestazione massimo \`e
\[\frac{1-\eta_{Carnot}}{\eta_{Carnot}}= \frac{T_L}{T_H-T_L}.\]
Se $T_L=4^\circ\mathrm{C}$ e $T_H=20^\circ\mathrm{C}$ (caso tipico del frigorifero casalingo) allora $COP_{max}\approx 17.3$. Tipicamente $COP\approx 4$.
\end{remark}
\begin{remark}[Massima efficienza di una pompa di calore realizzata con gas ideale]
Per una pompa di calore, la massima efficienza \`e data da
\[\frac{T_H}{T_H-T_L}.\]
\end{remark}


\section{Tipi di energie nei gas}
\subsection{Energia interna ed entropia}

\begin{remark}[Differenziali]
Osserviamo che
\[TdS=\rbar{\delta Q}_{rev}=dU-\rbar{\delta W}_{rev}=dU+pdV,\]
dunque
\[\boxed{dU=TdS-pdV}\]
\end{remark}
\begin{remark}
\underline{\textbf{NOTA BENE:}} Questo sembra un altro modo di scrivere il primo principio $dU=\delta Q+\delta W$, ma se il processo non \`e reversibile allora non \`e detto che $pdV=\delta W$.
\end{remark}

\begin{proposition}[Equazione di Helmholtz]\label{EquazioneHelmholtz}
Vale la seguente identit\`a
\[\boxed{\ppb VUT=T\ppb TpV-p}=T^2\ppb T{(p/T)}V\]
\end{proposition}
\begin{proof}
Sviluppiamo il differenziale dell'entropia
\[dS=\frac{dU}T+\frac{pdV}T=\frac1T\pa{\ppb TUVdT+\ppb VUTdV}+\frac pTdV.\]
Poich\'e $dS$ \`e un differenziale esatto, le derivate incrociate devono coincidere:
\begin{gather*}
\pp V{}\pa{\frac1T\ppb TUV}=\rbar{\pp T{}\pa{\frac1T\pa{\ppb VUT+p}}}_V\\
\cancel{\frac1T\pp{V\del T}{^2U}}=-\frac1{T^2}\pa{\ppb VUT+p}+\cancel{\frac1T\pp{V\del T}{^2U}}+\frac1T\ppb TpV\\
\ppb VUT=T\ppb TpV-p.
\end{gather*}
\end{proof}
\begin{remark}
Osserviamo che 
\[dU=\under{=C_V}{\ppb TUV}dT+\ppb VUTdV,\]
quindi la formula di Helmholtz misura ``quanto un gas non \`e ideale" (ricordiamo (\ref{DerivataEnergiaInternaRispettoAlVolume}))
\end{remark}
\begin{remark}
Se $dU=0$ allora
\[dT=-\frac{\ppb VUT}{\ppb TUV}dV=\ppb VTUdV\pasgnl={(\ref{EquazioneHelmholtz})}\frac1{C_V}\pa{p-T\ppb TpV}dV.\]
Questa identit\`a pu\`o essere interpretata come test per Gas ideali, basta capire se $T\ppb pTV$ coincide con $\frac{nRT}V$.
\end{remark}

\begin{remark}
Valgono le seguenti identit\`a
\[C_V=T\ppb TSV,\quad C_p=T\ppb TSp,\qquad \gamma=\frac{\ppb VpS}{\ppb Vpp}=\frac{\beta_T}{\beta_S}\]
\end{remark}
\begin{proof}
ESERCIZIO
\end{proof}


\subsubsection{Per gas ideali}
Il calore scambiato per trasformazioni reversibili nei gas ideali si pu\`o sviluppare in
\[\rbar{\delta Q}_{rev}=dU+pdV=C_VdT+\frac{nRT}VdV,\]
da cui
\[dS=\frac{\delta Q}T=\frac{C_V}TdT+\frac{nR}VdV.\]
\begin{proposition}[Entropia per gas ideali]\label{EntropiaGasIdeali}
Valgono le seguenti espressioni:
\begin{align*}
S_B-S_A=&C_V\log\pa{\frac{T_B}{T_A}}+nR\log\pa{\frac{V_B}{V_A}}\\
=&C_V\log\pa{\frac{p_BV_B^\gamma}{p_AV_A^\gamma}}\\
=&nc_p\log\pa{\frac{T_B}{T_A}}-nR\log\pa{\frac{p_B}{p_A}}
\end{align*}
\end{proposition}
\begin{proof}
Ricaviamo le tre formulazioni:
\begin{itemize}
\item Integrando $dS=\frac{C_V}TdT+\frac{nR}VdV$ troviamo la prima espressione.
\item Sfruttando la proporzionalit\`a $\displaystyle \frac{T_B}{T_A}=\frac{p_BV_B}{p_AV_A}$ possiamo rielaborare la prima forma come segue
\begin{align*}
S_B-S_A=&C_V\log\pa{\frac{p_B}{p_A}}+(C_V+nR)\log\pa{\frac{V_B}{V_A}}=\\
=&C_V\log\pa{\frac{p_B}{p_A}}+C_p\log\pa{\frac{V_B}{V_A}}=\\
=&C_V\pa{\log\pa{\frac{p_B}{p_A}}+\log\pa{\pa{\frac{V_B}{V_A}}^\gamma}}=\\
=&C_V\log\pa{\frac{p_BV_B^\gamma}{p_AV_A^\gamma}},
\end{align*}
ricavando la seconda espressione.
\item Ricordiamo che $\delta Q=nc_p dT-Vdp$. Dividendo per $T$ e poi integrando\footnote{stiamo usando che $-V/T=-nR/p$.} ricaviamo
\[S_B-S_A=nc_p\log\pa{\frac{T_B}{T_A}}-nR\log\pa{\frac{p_B}{p_A}}.\]
\end{itemize}
\end{proof}


\begin{remark}
Se ci spostiamo lungo una adiabatica reversibile, dalla seconda formula ricaviamo $\Delta S=0$ come ci aspettiamo.
\end{remark}

\begin{proposition}[Entropia in gas ideali per processi standard]
Valgono le seguenti espressioni:

\begin{itemize}
\item[$\boxed{\text{Isocora}}$] $\Delta S=nc_V\log\pa{\frac{T_B}{T_A}}\leadsto dS=nc_V\frac{dT}T,$
\item[$\boxed{\text{Isobara}}$] $\Delta S=nc_p\log\pa{\frac{T_B}{T_A}}\leadsto dS=nc_p\frac{dT}T.$
\item[$\boxed{\text{Isoterma}}$] $\Delta S=nR\log\pa{\frac{V_B}{V_A}}=-nR\log\pa{\frac{p_B}{p_A}}.$
\end{itemize}
\end{proposition}
\begin{proof}
Basta applicare le espressioni trovate (\ref{EntropiaGasIdeali}).
\end{proof}



\subsection{Espansione di Joule-Thompson: Entalpia}

[DESCRIVI L'ESPERIMENTO]

\begin{definition}[Entalpia]
L'\textbf{entalpia} in un gas \`e definita da\footnote{cio\`e l'energia interna sommata al ``lavoro necessario per portare il volume da $0$ a $V$ a pressione costante".}
\[H=U+pV.\]
\end{definition}

\begin{remark}
L'entalpia \`e ottenuta da $U$ come trasformazione di Legendre:\\
Consideriamo $U(S,V)$ e cambiamo la coordinata $V$ in $-p=\ppb VUS$, che possiamo fare ponendo $H(S,p)=U-(-pV)=U+pV$.
\end{remark}

\begin{remark}[Differenziale dell'entalpia]
Il differenziale di $H$ \`e dato da
\[dH=dU+pdV+Vdp=TdS+Vdp\pasgnl={se reversibile}\delta Q+Vdp\]
oppure da
\[dH=dU+p(\al VdT-\beta_T Vdp)+Vdp=dU+p\al VdT+V(1-\beta_Tp)dp.\]
\end{remark}
\begin{corollary}
Valgono le seguenti identit\`a
\[\ppb SHp=T,\quad \ppb pHS=V,\quad \ppb THp=C_p.\]
\end{corollary}
\begin{proof}
Le prime due seguono immediatamente da $dH=TdS+Vdp$, mentre l'ultima segue sviluppando la definizione di $C_p$:
\[C_p=\rbar{\frac{\delta Q}{\delta T}}_p=\ppb TUp+p\ppb TVp=\rbar{\pp T{}(U+pV)}_p=\ppb THp.\]
\end{proof}




\begin{definition}[Coefficiente di Joule-Thomson]
Definiamo il \textbf{coefficiente di Joule-Thomson} come
\[\mu_{JT}=\ppb pTH.\]
\end{definition}
\begin{fact}
Per ogni gas esiste una temperatura, detta \textbf{temperatura di inversione}, tale che sotto questa temperatura $\mu_{JT}>0$
\end{fact}

\subsubsection{Per gas ideali}
Nel caso dei gas ideali si ha che
\[dH=dU+pdV+Vdp=C_VdT+d(pV)=n(c_V+R)dT=nc_pdT.\]
\begin{remark}
Poich\'e nei gas ideali $U$ dipende solo da $T$ e $pV=nRT$, si ha che anche $H$ dipende solo da $T$ per gas ideali. Segue che possiamo testare se un gas \`e ideale verificando se $\mu_{JT}=0$ o meno.
\end{remark}


\section{Energia libera di Helmholtz e Gibbs}
\noindent
Diamo un nome alle rimanenti trasformazioni di Legendre di $U$:
\begin{definition}[Energia libera di Helmholtz]
Dall'energia interna cambiamo $S$ in $\ppb SUV=T$, definendo $F=U-TS$, detta \textbf{energia libera di Helmholtz}. Osserviamo che
\[dF=-SdT-pdV.\]
\end{definition}

\begin{remark}
L'energia libera di Helmholtz ci permette di ricavare il ``massimo lavoro estraibile tra due stati".
\end{remark}
\begin{proof}
Per il secondo principio $Q/T\leq \Delta S$, dunque
\[W=\Delta U-Q\geq \Delta U-T\Delta S=\Delta F,\]
cio\`e il lavoro che pu\`o fare il sistema ($-W$) \`e al massimo $-\Delta F$.
\end{proof}

\begin{remark}
Si ha che
\[\ppb TFV=-S,\quad \ppb VFT=-p.\]
\end{remark}

\begin{definition}[Energia libera di Gibbs]
Dall'entalpia cambiamo $S$ in $\ppb SHp=T$, definendo $G=H-TS=U+pdV-TS=F+pV$, detta \textbf{energia libera di Gibbs}. Osserviamo che
\[dG=-SdT-Vdp.\]
\end{definition}

\begin{remark}
Si ha che
\[\ppb pGT=V,\quad \ppb TGp=-S.\]
\end{remark}


\begin{proposition}[Relazioni di Maxwell]\label{RelazioniMaxwell}
Valgono anche le equazioni
\[\ppb VTS=-\ppb SpV,\quad \ppb pTS=\ppb SVp,\quad \ppb VST=\ppb TpV,\quad \ppb pST=-\ppb TVp\]
o equivalentemente
\[\pp{(p,V)}{(T,S)}=1.\]
\end{proposition}
\begin{proof}
***********************************
\end{proof}

\subsection{Riassunto}
Riassumiamo il comportamento di queste quantit\`a nella seguente tabella:
\begin{center}
\begin{tabular}{|c||c||c|}
\hline
Esp. libera di Joule & Esp. di Joule-Thompson &\\\hline\hline
$U$ costante & $H$ costante & $F$ costante\\\hline
$\mu_J=\ppb VTU$ & $\mu_{JT}=\ppb pTH$&\\\hline
$dU=\delta Q-pdV$ & $dH =\delta Q+Vdp$&\\\hline
$C_V=\ppb TUV$ & $C_p=\ppb THp$&\\\hline
In isocora $\Delta U=Q$ & In isocora $\Delta H=Q$&\\\hline
In adiabatica $\Delta U=-\int pdV$ & In adiabatica $\Delta H=\int Vdp$&\\\hline
$\ppb SUV=T,\quad \ppb VUS=-p$ & $\ppb SHp=T,\quad \ppb pHS=V$&\\\hline
In gas ideale $\Delta U=\int c_VdT$ & In gas ideale $\Delta H=\int c_pdT$&\\\hline
\end{tabular}
\end{center}
******************************************************

\section{Sistemi aperti di gas}
Per sistemi di gas aperto
\[\Delta U=Q+W+\Delta U_{materia},\]
o in forma differenziale
\[dU=TdS-pdV+\mu dn\]
dove $\mu$ \`e il \textbf{potenziale chimico}.
\begin{remark}
Valgono le identit\`a differenziali
\[\ppb SU{V,n}=T,\qquad \ppb VU{S,n}=-p,\qquad \ppb nU{S,V}=\mu.\]
\end{remark}













\chapter{Transizione di fase}
Generalmente transizioni di stato avvengono per $p$ e $T$ costanti, quindi la forma di energia pi\`u utile da considerare \`e l'energia libera di Gibbs
\[\Delta G=\Delta H-T\Delta S.\]
Poich\'e $dG=-SdT-Vdp$, si ha che se $p$ e $T$ restano costanti allora $\Delta G=0$, cio\`e \[T\Delta S=\Delta H.\]



\section{Transizione tra due fasi}
Chiamiamo le due fasi ``liquido" e ``vapore". 
\begin{notation}
Siano $n_L$ le moli di liquido e $n_V$ le moli di vapore.
\end{notation}
\begin{remark}[Energia libera di Gibbs molare]
Si ha che\footnote{nell'ultima uguaglianza abbiamo usato il fatto che l'energia \`e una grandezza estensiva.}
\[G=G_L(p_L,T_L,n_L)+G_V(p_V,T_V,n_V)\pasgnl={}n_Lg_L(p_L,T_L)+n_Vg_V(p_V,T_V)\]
dove $g$ \`e l'\textbf{energia libera di Gibbs molare}.
\end{remark}

\begin{remark}[Proporzione tra le fasi]
Per conservazione della materia $n_L+n_V=n$ \`e costante. Chiamiamo $\al$ la proporzione di liquido, cio\`e $n_L=\al n$ e $n_V=(1-\al)n$.
\end{remark}
\noindent
Possiamo riscrivere l'energia libera di Gibbs in termini di $n$ ed $\al$:
\[G=n\al g_L(p_L,T_L)+n(1-\al)g_V(p_V,T_V).\]
Poich\'e consideriamo tutto in regime di quasi-equilibrio, $T_L=T_V$ e $p_L=p_V$\footnote{le pressioni sono le stesse perch\'e c'\`e equilibrio meccanico.}. Abbiamo dunque ricavato che $G$ dipende solo da $p,\ T$ e $\al$.

\begin{proposition}[Condizione di equilibrio tra due fasi]
All'equilibrio si ha che
\[g_L=g_V.\]
\end{proposition}
\begin{proof}
Per $p$ e $T$ costanti sappiamo (\ref{GibbsDiminuiscePerPressioneETemperaturaCostanti}) che $\Delta G\leq 0$ , quindi siamo all'equilibrio solo se $G(\al)$ \`e minima, cio\`e\footnote{Per $p$ e $T$ costanti $G$ dipende solo da $\al$.}
\[0=\pp\al G=ng_L+0-n g_V\implies g_L=g_V.\]
\end{proof}

\subsection{Contenitore chiuso}
Modelliamo una contenitore chiuso ($U=cost$, $V=cost$).
\begin{notation}[Volume, energia interna ed entropia molari]
Scriviamo
\begin{align*}
V=&n\al v_L+n(1-\al)v_V\\
U=&n\al u_L+n(1-\al)u_V\\
S=&n\al s_L+n(1-\al)s_V
\end{align*}
\end{notation}
\begin{proposition}[Condizione di equilibrio tra due fasi]
All'equilibrio si ha che
\[g_L=g_V.\]
\end{proposition}
\begin{proof}
All'equilibrio $0=dV=dU=dS=dn$, quindi (ricordando che $p$ e $T$ sono costanti)
\begin{align*}
0=&dU+pdV-TdS=\\
=&d(n\al u_L+n(1-\al)u_V)+pd(n\al v_L+n(1-\al)v_V)-Td(n\al s_L+n(1-\al)s_V)=\\
=&n\al (du_L+pdv_L-Tds_L)+n(1-\al)(du_V+pdv_V-Tds_V)+\\
&+nd\al (u_L+pv_L-Ts_L)-nd\al (u_V+pv_V-Ts_V).
\end{align*}
All'equilibrio si ha che $du_L+pdv_L-Tds_L=0$ e similmente per il vapore, dunque
\[u_L+pv_L-Ts_L=u_V+pv_V-Ts_V.\]
Evidenziando le moli in questa equazione si ha
\[\under{=\frac{G_L}{n_L}}{\frac{U_L}{n_L}+p\frac{V_L}{n_L}-T\frac{S_L}{n_L}}=\under{=\frac{G_V}{n_V}}{\frac{U_V}{n_V}+p\frac{V_V}{n_V}-T\frac{S_V}{n_V}},\]
ovvero $g_L=g_V$ come volevasi dimostrare.
\end{proof}


\section{Caso generale}
Consideriamo $N$ componenti\footnote{moralmente $N$ \`e il numero di sostanze diverse} e $F$ fasi. Sia $n^{(i)}_k$ il numero di moli della componente $i$ nella fase $k$.\medskip

\noindent
Osserviamo che
\[G=\sum_{k\in\cpa{1,\cdots, F}}G_k(T,p,(n_k^{(i)})_{i\in\cpa{1,\cdots, N}}),\]
cio\`e a priori $G$ dipende da $NF+2$ variabili.

\begin{proposition}[Regola delle fasi di Gibbs]\label{RegolaFasiGibbs}
Il numero di gradi di libert\`a \`e
\[\boxed{\nu=2+N-F}\]
Questa \`e la \textbf{regola delle fasi (di Gibbs)}.
\end{proposition}
\begin{proof}
Consideriamo due fasi ($a$ e $b$) e la transizione dalla fase $a$ alla fase $b$:
\[\begin{cases}
n_a^{(i)}\to n_a^{(i)}-\delta n^{(i)}\\
n_b^{(i)}\to n_b^{(i)}-\delta n^{(i)}
\end{cases}.\]
Si ha che all'equilibrio
\[0=dG=dG_a+dG_b=\pp {n_a^{(i)}}{G_a}(-\delta n^{(i)})+\pp {n_b^{(i)}}{G_a}(\delta n^{(i)}),\]
ma l'energia libera di Gibbs \`e una grandezza estensiva, quindi vale la dipendeza lineare
\[\pp {n_k^{(i)}}{G_k}=\frac{G_k}{n_k^{(i)}},\]
segue dunque che
\[g_a^{(i)}=\frac{G_a}{n_a^{(i)}}=\frac{G_b}{n_b^{(i)}}=g_b^{(i)}.\]
Queste sono $F-1$ condizioni indipendenti per ogni componente.
\bigskip

\noindent Osserviamo inoltre che per ogni fase possiamo eliminare un grado di libert\`a considerando i rapporti tra le moli di componenti in quella fase.\bigskip

\noindent Tirando le somme si ha che i gradi di libert\`a sono
\[NF+2-(N(F-1)+F)=2+N-F.\]
\end{proof}



\begin{example}
Studiamo i valori di $N$, $F$ e $\nu$ per alcuni sistemi
\begin{itemize}
\item Fluido omogeneo: $N=1$, $F=1$, $\nu=2$
\item Fluido omogeneo dato da due gas: $N=2$, $F=1$, $\nu=3$
\item Acqua e vapore: $N=1$, $F=2$, $\nu=1$
\item Acqua, vapore e ghiaccio: $N=1$, $F=3$, $\nu=0$.
\end{itemize}
\end{example}

\begin{definition}[Punto triplo]
Considerando come sistema termodinamico l'acqua, esiste una precisa combinazione di temperatura e pressione tale per cui essa risulta in trasizione tra gli stati solido liquido e gassoso simultaneamente.\\
Questo stato si chiama \textbf{punto triplo} e i valori in questione sono una temperatura di $0.01 ^\circ \mathrm{C}$ e una pressione di $0.006\ \mathrm{atm}$.
\end{definition}

\newpage


\subsection{Grafici delle transizioni di fase}
Inseriamo qualche grafico che mostra come e quando le transizioni di stato avvengono:

\begin{figure}[!htb]
    \centering
    \includegraphics[width=7cm]{images/Grafico_p_V_transizione_di_stato.png}
\end{figure}

\begin{figure}[!htb]
    \centering
    \includegraphics[width=11cm]{images/Grafico_p_T_transizione_stato.png}
\end{figure}


\subsection{Definizione di temperatura tramite gas}
L'esistenza del punto triplo ci permette di definire la temperatura in termini di una grandezza che possiamo misurare direttamente. 
\medskip

\noindent
A bassa pressione i gas tendono al regime di Gas ideale.\\
Se fissiamo il volume e le moli di gas possiamo definire $\theta$ in modo tale che $p=p_0(1+\al\theta)$, cio\`e poniamo 
\[\theta=\frac1\al\frac{p-p_0}{p_0}.\] 
Se imponiamo che l'acqua congeli per $\theta=0$ e evapori per $\theta=100$ allora ricaviamo $1/\al=273.15$. Notiamo inoltre\footnote{l'addizione di $\al\ii$ corrisponde alla traslazione che trasforma gradi Celsius in gradi Kelvin.}
\[\frac{p_2}{p_1}=\frac{\al\ii+\theta_2}{\al\ii +\theta_1}=\frac{\theta_2'}{\theta_1'}.\]
Possiamo dunque definire la temperatura (in Kelvin) come
\[T=\lim_{p^{(PT)}\to 0}273.16 \frac{p}{p^{(PT)}}\]
dove $p^{(PT)}$ \`e la pressione del gas nel termometro quando questo sistema \`e in equilibrio con il sistema di punto triplo con l'acqua. Il limite corrisponde a prendere gas sempre pi\`u rarefatti, cio\`e a lavorare nel limite dei gas perfetti dove vale la proporzionalit\`a sopra.
\medskip

\noindent Sfruttando questa definizione possiamo costruire un termometro a gas come in figura

\begin{figure}[!htb]
    \centering
    \includegraphics[width=11cm]{images/Termometro_a_gas.png}
\end{figure}


\noindent Quando il gas \`e alla temperatura che vogliamo misurare, misuriamo la differenza di altezza tra il livello a contatto con il gas e il livello di controllo posto a pressione atmosferica. 
Questa differenza \`e proporzionale alla differenza di pressione e questo ci permette di ricavare la temperatura se la fissiamo per quando \`e nel punto critico.




\section{Calore latente}
Consideriamo nuovamente il caso di due fasi (liquido e vapore). Osserviamo che fissata una temperatura, la pressione alla quale avviene la transizione di fase ne \`e una funzione. Segue che anche $V$ \`e una funzione di $T$.
\begin{remark}
Osserviamo che $dn_L=-dn_V$, quindi
\[\ppb VUT=\frac{u_V-u_L}{v_V-v_L}.\]
\end{remark}
\begin{proof}
Segue ricordando che $V=n_L v_L(T)+n_Vv_V(T)$ (e similmente per $U$) e che per le transizioni di fase $T$ \`e costante.
\end{proof}

\noindent
Per il primo principio
\[\delta Q=dU+pdV=dn_V(u_L-u_L+p(v_V-v_L)),\]
questo motiva la seguente
\begin{definition}[Calore latente]
Definiamo il \textbf{calore latente (molare) di vaporizzazione} come
\[\la=\frac{\delta Q}{dn_V}=u_V-u_L+p(v_V-v_L).\]
\end{definition}

\begin{proposition}[Equazione di Clapeyron]\label{EquazioneClapeyron}
Sulla transizione di fase
\[\dd Tp=\frac\la{T(v_V-v_L)}\]
\end{proposition}
\begin{proof}
Sviluppiamo $TdS$:
\[TdS=\under{=nc_V}{T\ppb TSVdT}+T\ppb VSTdV.\]
Applicando la relazione di Maxwell (\ref{RelazioniMaxwell}) data da $\ppb VST=\ppb TpV$ troviamo
\[TdS=nc_VdT+T\ppb TpVdV.\]
Combinando questo con $dU=TdS-pdV$ ricaviamo
\[dU=nc_VdT+\pa{T\ppb TpV-p}dV,\]
cio\`e
\[\ppb VUT=T\ppb TpV-p.\]
Ricordiamo ora che $\ppb VUT=\frac{u_V-u_L}{v_V-v_L}$, da cui
\[\frac\la{v_V-v_L}=T\ppb TpV,\]
che \`e la tesi se osserviamo che $\ppb TpV=\dd Tp$.
\end{proof}
\begin{remark}[Equazione di Clausius-Clapeyron]
Se $v_V\gg v_L$ allora per gas ideali
\[\dd Tp=\frac\la{RT^2}p\leadsto p\propto e^{-\la/RT}.\]
\end{remark}







\chapter{Teoria cinetica dei gas}

[QUESTO CAPITOLO DEVE ANCORA ESSERE SISTEMATO!!!!]

\section{Modello dei gas ideali}
Nella realt\`a i gas sono composti da tante particelle. Imponiamo alcune condizioni:
\setlength{\leftmargini}{0cm}
\begin{itemize}
\item[$\boxed{\text{Isotropo}}$] Le velocit\`a delle particelle sono equamente distribuite in ogni direzione.
\item[$\boxed{\text{Omogeneo}}$] Le particelle sono equamente distribuite.
\end{itemize}
\setlength{\leftmargini}{0.5cm}
Sia $dn(v)$ il numero di particelle con una data velocit\`a.
\begin{remark}
Se $N$ \`e il numero totale di particelle
\[N=\int dn(v)=\int_0^\infty \dd vndv\]
dove $\dd vn$ \`e in un qualche modo la ``densit\`a delle particelle di una data velocit\`a".
\end{remark}
\begin{remark}
Sia $\vec v$ una qualche velocit\`a.
\[dn(\vec v)\pasgnl={isotropia}dn(v)\frac{d\Omega}{4\pi},\]
dove $d\Omega$ \`e l'\textbf{angolo solido}, cio\`e l'area della ambiguit\`a sulla direzione voluta sulla sfera di raggio 1\footnote{$d\Omega=\sin\theta d\theta d\phi$.}.
\end{remark}

\begin{remark}
Per omogeneit\`a il numero di particelle in un volumetto \`e
\[dn=\frac NVdV=\frac NV dAvdt\cos\theta.\]
\end{remark}

\begin{remark}
L'impulso trasferito alla parete dall'impatto di una particella \`e $\abs{\Delta \vec p}=2mv\cos \theta$.
\end{remark}
\noindent Appurate queste equazioni possiamo scrivere il differenziale della pressione come segue:
\begin{align*}
d^2p=&\frac{dF}{dA}=\frac{\abs{d\vec p}/dt}{dA}=\frac1{dA}\frac1{dt}\abs{d\vec p}_{singola}dn dn(\vec v)\\
=&\frac1{dA}\frac1{dt}{2mv\cos \theta}\quad{\frac{N}V dAvdt\cos \theta}\quad{dn(v)\frac{d\Omega}{4\pi}}=\\
=&N\frac{2mv^2\cos^2\theta}Vdn(v)\frac{d\Omega}{4\pi}.
\end{align*}
Facendo la media su tutte le direzioni troviamo il vero differenziale della pressione:
\begin{align*}
dp=&\int_\Omega d^2p=\frac{mv^2}{2\pi}\frac NVdn(v)\int_0^{2\pi}d\phi\int_0^{\pi/2}\cos^2\theta\sin\theta d\theta=\\
=&\frac13mv^2\frac NVdn(v).
\end{align*}
Integrando ora sui possibili moduli delle velocit\`a troviamo la pressione:
\[p=\frac13m\frac NV\under{\doteqdot \ps{v^2}}{\int_0^\infty v^2dn(v)}.\]

\begin{definition}[Energia cinetica media]
Definiamo l'\textbf{energia cinetica media} come \[\ps{E_K}=\frac12mN\ps{v^2}.\]
\end{definition}
\begin{remark}
Vale la relazione
\[\boxed{\frac12m\ps{v^2}=\frac32 k_b T}\]
\end{remark}
\begin{proof}
Osserviamo che
\[pV=\frac13 mN\ps{v^2},\]
dunque
\[nRT=pV=\frac23\ps{E_K},\]
cio\`e
\[T=\frac 23\frac{\ps{E_K}}{nR}=\frac 23N_a\frac{\ps{E_K}}{NR}=\frac 23\frac{\ps{E_K}}{Nk_b}\]
In conclusione
\[{\frac12m\ps{v^2}=\frac32 k_b T}.\]
\end{proof}

\noindent
Consideriamo ora l'energia interna di questo sistema\footnote{affermare che $U=\ps{E_K}$ corrisponde ad assumere che il gas sia monoatomico.}:
\[U=\ps{E_K}=\frac32nRT=C_VT.\]
In generale $U=E_K+E_P$ per una qualche energia potenziale $E_P$. Per piccoli spostamenti $E_P=(E_P)_0+\frac12kx^2$\ \footnote{regime ragionevole per il tipo di forze che agisce all'interno di materiali.}. Nel caso biatomico per esempio $E_P=\frac12I\omega^2$.
\setlength{\leftmargini}{0cm}
\begin{itemize}
\item[$\boxed{\text{Solido}}$] 6 gradi di libert\`a: 3 potenziali (forze elastiche) e 3 cinetiche.
\item[$\boxed{\text{Gas perf. mono.}}$] 3 gradi di libert\`a, tutti cinetici.
\item[$\boxed{\text{Gas perf. bi.}}$] 5 gradi di libert\`a: 3 cinetici e 2 dalla rotazione \footnote{la rotazione lungo l'asse che congiunge le particelle \`e irrilevante}.
\end{itemize}
\setlength{\leftmargini}{0.5cm}


\begin{fact}[Principio di equipartizione]
Ogni grado di libert\`a contribuisce un addendo $\frac12 RT$ al calore specifico a volume costante.
\end{fact}

\section{Distribuzione delle velocit\`a}
Consideriamo ora un sistema isolato con temperatura costante. Cerchiamo di capire come \`e fatta la distribuzione delle velocit\`a.

Decomponiamo le velocit\`a $\vec v$ in $(v_x,v_y,v_z)$. Notiamo che
\[dn(v_x)=Nf(v_x)dv_x,\]
dove $f$ \`e la densit\`a di probabilit\`a che la componente $x$ sia $v_x$.\\
Per isotropia si ha che
\[dn(v_y)=Nf(v_y)dv_y,\quad dn(v_z)=Nf(v_z)dv_z,\]
dunque
\[dn(\vec v)=N f(v_x)f(v_y)f(v_z)dv_xdv_ydv_z.\]
Sempre per isotropia, in realt\`a $f(v_x)f(v_y)f(v_z)$ \`e una funzione del modulo $v=\sqrt{v_x^2+v_y^2+v_z^2}$, non delle singole componenti.

Segue dunque che
\[f(v)f(0)f(0)=\phi(v)=f(v_x)f(v_y)f(v_z).\]
Dividendo per $f(0)^3$ troviamo
\[\frac{f(v)}{f(0)}=\frac{f(v_x)}{f(0)}\frac{f(v_y)}{f(0)}\frac{f(v_z)}{f(0)}\]
\[\log\frac{f(v)}{f(0)}=\log \frac{f(v_x)}{f(0)}+\log\frac{f(v_y)}{f(0)}+\log\frac{f(v_z)}{f(0)}\]


Per brevit\`a sia $G(v)=\log\frac{f(v)}{f(0)}$, da cui
\[G(v)=G(v_x)+G(v_y)+G(v_z).\]
Derivando per $v_i$ con $i\in\cpa{x,y,z}$ troviamo
\[\frac{G'(v)v_i}v=G'(v_i),\]
si ha dunque che
\[\frac{G'(v)}{v}=\frac{G'(v_x)}{v_x}=\frac{G'(v_y)}{v_y}=\frac{G'(v_z)}{v_z}\doteqdot -2\al\]
Troviamo dunque che
\[G(v_i)=-\al v_i^2+C\leadsto f(v_i)=ae^{-\al v_i^2},\]
dunque
\[\phi(v)=Ae^{-\al (v_x^2+v_y^2+v_z^2)}=Ae^{-\al v^2}.\]
Poich\'e $\phi$ \`e una densit\`a di probabilit\`a\footnote{abbiamo usato il fatto che $\int_{-\infty}^{+\infty}e^{-\al x^2}dx=\sqrt{\frac{\pi}\al}$} si ha che $A=\pa{\frac{\al}\pi}^{3/2}$, cio\`e
\[dn(\vec v)=N\pa{\frac{\al}\pi}^{3/2}e^{-\al v^2} dv_xdv_ydv_z\]

\[dn(v)=\int_\Omega dn(\vec v)=N\int_\Omega \phi(v)dv_xdv_ydv_z=N\int_\Omega \phi(v)v^2dvd\Omega=4\pi N\phi(v)v^2dv\]

\[dn(v)=4N\sqrt{\frac{\al^3}\pi}v^2e^{-\al v^2}dv\]

Cerchiamo di capire chi \`e $\al$. \footnote{$\int_0^\infty x^ne^{-\al x^2}=\frac{\Gamma((n+1)/2)}{2a^{(n+1)/2}}$}
\[\frac{3k_bT}m=\ps{v^2}=\frac{\int_0^\infty v^2dn(v)}{\int_0^\infty dn(v)}=\frac 3{2\al}\leadsto \al=\frac m{2k_b T}.\]
Abbiamo dunque ricavato che
\[dn(v)=N 4\pi\pa{\frac{m}{2\pi k_b T}}^{3/2}v^2\exp\pa{-\frac m{2k_bT}v^2}dv\]
\textbf{Distribuzione di Maxwell-Boltzmann}.

La distribuzione ha un massimo pi\`u o meno quando $\frac {mv^2}{2k_bT}=0$ (non esattamente perch\'e c'\`e il termine $v^2$)

[GRAFICO DI $\dd vn$]

Qualcuno ha provato a misurare se effettivamente la distribuzione \`e questa:
fai girare una ruota velocemente con un buchino e poi metto a contatto la ruota con un tubo da cui viene il gas. Vedendo quante sono arrivate sulle varie parti interne della ruota uno pu\`o capire la distribuzione.

\section{Entropia nel modello statistico}
\begin{definition}[Macro- e Microstati]
Dato un sistema statistico come quello trattato in questo capitolo, un \textbf{microstato} \`e il dato di ogni singola posizione e velocit\`a, un \textbf{macrostato} \`e la classe di microstati con le stesse propriet\`a globali (per esempio volume, temperatura, pressione).
\end{definition}
\begin{fact}
\textbf{Tutti i microstati compatibili con un dato macrostato sono equiprobabili.}
\end{fact}

\begin{remark}
Consideriamo due sistemi. Uno in un macrostato con probabilit\`a $P_1$ di verificarsi ed entropia $S_1$, il secondo con dati analoghi $P_2$ e $S_2$. Notiamo che l'insieme dei due sistemi ha entropia $S_1+S_2$ e la probabilit\`a del macrostato di questo insieme \`e $P_1P_2$. Intuitivamente
\[S\propto \log P\propto \log\Omega,\]
dove $\Omega$ \`e il numero di microstati con lo stesso macrostato.\\
Supponiamo dunque $S=C\log\Omega$.
\end{remark}





Consideriamo la seguente situazione: Una scatola adiabatica con due compartimenti di volume $V/2$. Dentro il primo compartimento si trovano $r$ moli di gas e nel secondo $1-r$ moli, entrambi alla stessa temperatura. Ora rimuoviamo la parete
\[\Delta S=(S_{f_1}-S_{i_1})+(S_{f_2}-S_{i_2})=rR\log\pa{\frac{rV}{V/2}}+(1-r)R\log\pa{\frac{(1-r)V}{V/2}}=\]
\[=R\pa{r\log r+(1-r)\log(1-r)+\log 2}.\]


Consideriamo ora una situazione analoga ma con delle particelle: Nel primo compartimento abbiamo $N/2 -x$ particelle e nel secondo $N/2 +x$.
\[\Omega(N,x)=\binom{N}{x+N/2},\]
da cui
\[S_i=C\log(N!)-\log((N/2-x)!)-\log((N/2+x)!).\]
Applicando l'approssimazione di Stirling $\log(N!)\simeq N\log N$ si ha
\[S_i\simeq -NC\pa{\pa{\frac12-\frac xN}\log\pa{\frac12-\frac xN}+\pa{\frac12+\frac xN}\log\pa{\frac12+\frac xN}}.\]
(Nel caso di prima avremmo $r=\frac1{N_a}\pa{\frac{N_a}2-x}=\frac12-\frac x{N_a}$, da cui
\[S_i\simeq -N_a C\pa{r\log r+(1-r)\log (1-r)}\])

Per quanto riguarda lo stato finale il numero di microstati possibili ora \`e ogni combinazione di particelle nei due contenitori, $2^{N_a}$ possibilit\`a in totale, dunque
\[S_f=CN_a\log 2,\]
dunque
\[\Delta S=S_f-S_i=N_AC\pa{r\log r+(1-r)\log(1-r)+\log 2}\]
Se la costante $C$ la chiamiamo ``costante di Boltzmann" ricaviamo di nuovo effettivamente $S=k\log \Omega$.






Fissiamo la temperatura e consideriamo una espansione $V\to V+dV$. Come varia $\Omega$?\\
Mi aspetto qualcosa del tipo
\[\frac{\Omega_f}{\Omega_i}=\frac{(V+dV)^N}{V^N}=\pa{1+\frac{dV}V}^N\]
Stiamo considerando il modello di gas perfetto, quindi $dT=0\implies dU=0$, dunque
\[dV=\frac{\delta Q}p,\]
da cui
\[\frac{dV}V=\frac{\delta Q}{pV}=\frac{\delta Q}{nRT}=\frac{\delta Q}{Nk_b T}\]

Quindi verifichiamo che l'entropia di Boltzmann verifica la definizione di entropia che avevamo dato:
\[\Delta S=k_b\log\pa{\frac{\Omega_f}{\Omega_i}}=k_b N\log\pa{1+\frac{\delta Q}{Nk_b T}}\approx k_bN\frac{\delta Q}{Nk_bT}=\frac{\delta Q}T\]


Consideriamo ora una isocora e cambiamo la temperatura:
\[\ps{E_K}=\frac32k_bT\]

\[\frac12m\ps{v_x^2}=\frac12 kT\implies \ps{v_x^2}=\frac{k_bT}m,\]
inoltre per isotropia $\ps{v_x}$, quindi $\sqrt{\frac{k_bT}m}$ \`e la deviazione standard di $v_x$.
Quindi $v$ per una particella ha una deviazione standard nell'ordine di $\pa{\frac{k_bT}m}^{3/2}$ quindi nell'insieme si ha che $\ps{v}\propto T^{3N/2}$.

Quindi per questa trasformazione
\[\frac{\Omega_f}{\Omega_i}=\pa{1+\frac{dT}T}^{3N/2},\]
da cui
\[\Delta S=\frac{3N}2k_b\log\pa{1+\frac{dT}T}\approx \frac{3Nk_b}2\frac{dT}T=\frac{C_VdT}T=\frac{dU}T\pasgnl={isocora}\frac{\delta Q}T\]



\section{Informazione}
\begin{definition}[Informazione]
Sia $X$ variabile aleatoria discreta che pu\`o assumere $N$ valori $x_1,\cdots, x_N$ con densit\`a di probabilit\`a $P_i$.\\
Definiamo l'\textbf{informazione} derivante dal fatto che l'evento $x_i$ sia accaduto come
\[I_i=-\log P_i\]
\end{definition}
Vogliamo definire $\Hc(\cpa{P_i})$ come ``l'informazione che mi manca per capire l'esito data una distribuzione di probabilit\`a". Imponiamo alcune propriet\`a:
\begin{itemize}
\item $\Hc$ deve essere continua nelle $P_i$
\item Se per ogni $P_i=\frac1N$, $\Hc$ deve essere monotona e crescente in $N$.
\item [(Consistenza)\ $\bullet$] Sia $\pi$ una partizione di $N$ e per ogni elemento $g\in \pi$ sia $P_g=\sum_{i\in g}P_i$. Allora
\[\Hc(\cpa{P_i})=\Hc(\cpa{P_g}_{g\in \pi})+\sum_{g\in \pi} P_g\Hc(\cpa{P(x_i\mid g)})\]
\end{itemize}
\begin{theorem}
La funzione $\Hc$ deve assumere la forma
\[\Hc(\cpa{P_i})=-C\sum_i P_i\log(P_i).\]
Questa funzione di dice \textbf{entropia di Shannon/Gibbs}.
\end{theorem}
\begin{proof}
Consideriamo due casi
\setlength{\leftmargini}{0cm}
\begin{itemize}
\item[$\boxed{P_i=1/N}$] Sia $\Hc(\cpa{N\ii,\cdots, N\ii})=F(N)$. Consideriamo ora $n$ gruppi equiprobabili. Ogni gruppo $g$ contiene $N/n$ eventi, quindi $P_g=\frac1n$ e $P(x_i\mid g)=\frac nN$. Osserviamo dunque che $\Hc(\cpa{P(x_i\mid g)})=F(M/n)$. Per consistenza
\[F(N)=F(n)+\sum_n\frac1nF(N/n)=F(n)+F(N/n).\] 
\item Siano $s,t>1$ interi positivi e notiamo che esistono $\al,\beta$ interi tali che
\[\frac\al\beta\leq\frac{\log s}{\log t}<\frac{\al+1}\beta\implies t^\al\leq s^\beta<t^{\al+1}.\]
Per monotonia
\[F(t^\al)\leq F(s^\beta)<F(t^{\al+1}).\]
Per la propriet\`a mostrata nel caso particolare
\[\frac\al\beta\leq\frac{F(s)}{F(t)}<\frac{\al+1}\beta\implies t^\al\leq s^\beta<t^{\al+1},\]
dunque
\[\abs{\frac{F(s)}{F(t)}-\frac{\log s}{\log t}}\leq \frac1\beta,\]
quindi $F(s)=C\log s$ per una qualche costante $C$.
\item[$\boxed{\text{generale}}$] Sia $N_g=\#g$. Notiamo che $P_g=\frac{N_g}N$ e che $P(x_i\mid g)=\frac1{N_g}$. Osserviamo che
\[\Hc(\cpa{P_i})=F(N)=\Hc(\cpa{P_g})+\sum_g P_g F(N_g),\]
da cui
\begin{align*}
\Hc(\cpa{P_g})=& F(N)-\sum_g P_g F(N_g)=\sum_g P_g(F(N)-F(N_g))=\\
=& C\sum_g P_g\log(N/N_g)=\\
=&-C\sum_g P_g\log(P_g).
\end{align*}
\end{itemize}
\setlength{\leftmargini}{0.5cm}
\end{proof}

\begin{remark}
L'entropia \`e la media pesata del logaritmo delle probabilit\`a nella distribuzione a meno di costante.
\end{remark}

\begin{remark}
Se $P_i=1/N$ allora \[\Hc(\cpa{P_i})=-C\sum_i \frac1N\log\frac1N=-C\log N,\]
che a meno della notazione \`e l'equivalente di $S=k_b\log \Omega$.
\end{remark}


\subsection{Principio di massima entropia}
Consideriamo un sistema con $N$ stati e nessun vincolo. Sia $P_i$ la probabilit\`a dello stato $i$. Vogliamo trovare i $P_i$ che massimizzano $S$ sapendo che $\sum_i P_i=1$:\\
$S=-k_b\sum_i P_i\log P_i$. Per moltiplicatori di lagrange vogliamo massimizzare $S+\al \sum_i P_i$, cio\`e
\[0=\pp{P_i}{}(S+\al \sum_i P_i)=-k_b............\]


Consideriamo ora un vincolo $\ps{f(x_i)}=\sum_i P_i f(x_i)=cost.$

********************************


$U=\ps{E}$, $S=k\log Z-\frac UT$

\[P_i=\frac12 e^{\frac{E(x_i)}{k_bT}}\]






\section{Terzo principio}
\begin{fact}[Terzo principio della termodinamica]
\textbf{In un processo reversibile isotermo $\lim_{T\to 0}\Delta S=0$.}
\end{fact}
\begin{remark}
Stiamo dicendo che le isoterme per $T$ vicino a $0$ si avvicinano ad essere adiabatiche.
\end{remark}

\begin{remark}
Moralmente il principio dice che \`e difficile raffreddare verso temperature vicine allo 0 assoluto.
\end{remark}

\begin{remark}
Possiamo riformulare il principio affermando che ogni sistema ha la stessa entropia allo zero assoluto.
\end{remark}

\part{Relativit\`a speciale}
\chapter{Introduzione}
\section{I principi della relativit\`a}

\subsection{Principio di relativit\`a}

\begin{definition}[Sistema di riferimento inerziale]
Un sistema di riferimento \`e \textbf{inerziale} se valgono le leggi di Newton.
\end{definition}

\begin{fact}[Principio di Relativit\`a]
\textbf{Le leggi della fisica sono le stesse in qualsiasi sistema di riferimento in moto rettilineo uniforme.}
\end{fact}

Dati due sistemi di riferimento $S$ e $S'$ come in figura

[DISEGNO]

supponiamo che $S'$ si muova in moto rettilineo uniforme rispetto a $S$. Supponiamo inoltre che per $t=0$ si abbia $O=O'$.
\medskip

\noindent
Consideriamo ora un punto $P$. Nel sistema $S$ esso \`e espresso tramite le coordinate $(x,y,z)$, mentre in $S'$ le sue coordinate sono $(x',y',z')$. Osserviamo che al variare di $t$ si ha che la terna $(x',y',z')$ cambia.\\
Se i sistemi sono come in figura troviamo la forma canonica delle \textbf{trasformazioni di Galileo}:
\[\begin{cases}
x'=x-ut\\
y'=y\\
z'=z\\
t'=t
\end{cases}\]
Osserviamo che
\[\dd t{x'}=\dd t{}(x-ut)\pasgnl={moto rett. unif.}\dd tx-u,\]
dunque
\[\begin{cases}
v_x'=v_x-u\\
v_y'=v_y\\
v_z'=v_z
\end{cases}\]
e derivando di nuovo troviamo le trasformazioni anche per le accelerazioni:
\[\begin{cases}
a_x'=a_x\\
a_y'=a_y\\
a_z'=a_z
\end{cases}\]

\begin{example}[Le leggi della fisica non cambiano]
Consideriamo la legge di Hooke
\[m\dd[2] tx=-k(x-x_0).\]
Cambiando sistema di riferimento $x=x'+ut$, $x_0=x_0'+ut$, dunque
\[m\dd[2]t{x'}=m\dd[2] t{}(x'+ut)=m\dd[2] tx=-k(x-x_0)=-k(x'+ut-(x_0'+ut))=-k(x'-x_0'),\]
cio\`e la legge continua a valere sostituendo le grandezze di un sistema con le equivalenti nell'altro.
\end{example}

\subsection{Costanza della velocit\`a della luce}
\begin{fact}[Equazioni di Maxwell]
Valgono le seguenti equazioni
\begin{align*}
&div\vec E=\frac{\rho}{\e_0}
&&div\vec B=0\\
&rot\vec E=-\pp t{\vec B}
&&rot\vec B=\mu_0\pa{\vec J+\e_0\pp t{\vec E}}
\end{align*}
\end{fact}
\noindent
Osserviamo che
\begin{align*}
\pp t{}=&\pp {t}{t'}\dd{t'}{}+\pp t{x'}\pp {x'}{}+\pp t{y'}\pp {y'}{}+\pp t{z'}\pp {z'}{}=\pp{t'}{}-u\pp{x'}{}\\
\pp x{}=&\pp {x'}{}\\
\pp y{}=&\pp {y'}{}\\
\pp z{}=&\pp {z'}{}
\end{align*}

\begin{example}
Le equazioni di Maxwell e la relativit\`a Galileiana non sono compatibili
\end{example}
\begin{proof}
Cambiando sistema di riferimento alle equazioni di Maxwell (per esempio una di quelle con rotore) troviamo
\[\pp {y'}{E_{z'}}-\pp {z'}{E_{y'}}=-\pp {t'}{B_{x'}}+u\pp {x'}{B_{x'}}\]
ovvero, applicando l'equazione con la divergenza
\[\pp {y'}{E_{z}+uB_y}-\pp {z'}{E_{y}-uB_z}=-\pp {t'}{B_{x}}\]
possiamo dunque ipotizzare
\[\begin{cases}
E_z'=E_z+uBy\\
E'_y=E_y-uB_z\\
E'_x=E_x\\
\vec B'=\vec B
\end{cases}\]
Problema, esiste un'altra equazione di Maxwell
\[\pp y{B_z}-\pp z{B_y}=\frac1{c^2}\pp t{E_x}\]
che non coninua a valere con le sostituzioni sopra.
\end{proof}

\noindent Ricordiamo che dalle equazioni di Maxwell segue la legge
\[\pp[2]t{\vec E}=c^2\nabla^2\vec E.\]
Questa equazione sembra quella di un'onda, ma allora la relativit\`a magari funziona se teniamo conto degli stessi effetti che subiscono le onde sotto queste trasformazioni.\\
In particolare ipotizziamo l'esistenza di un mezzo attraverso il quale la luce si propaga: l'etere.

\begin{fact}[Esperimento di Michelson-Morley]
L'etere e la terra non sono indipendenti.
\end{fact}
\begin{proof}[Descrizione dell'esperimento]
Esperimento con interferometro e specchi.

[DISEGNO]

Supponiamo che il vento d'etere sia diretto lungo $BD$, cio\`e la terra si muove in quella direzione rispetto al riferimento dell'etere:
\begin{align*}
&t(BD)=t_1\\
&t(DB)=t_2\\
&t(BC)=t(CB)=t_3
\end{align*}
Se $L$ \`e la lunghezza di $BD$ e di $BC$, se $u$ \`e la velocit\`a dell'a terra rispetto all'etere allora
\[t_1=\frac L{c-u}\quad t_2=\frac L{c+u}.\]
Consideriamo ora il sistema di riferimento dove la velocit\`a della luce \`e $c$, cio\`e nel sistema solidale all'etere:\\
Per quanto riguarda il tratto $BD$ in questo sitema 
\[ct_1=L+ut_1,\quad ct_2=L-ut_2\quad\implies\quad t_1+t_2=\frac{2Lc}{c^2-u^2},\]
mentre sul tratto $BC$ si ha
\[(ct_3)^2=L^2+(ut_3)^2\quad\implies\quad t_3=\frac L{\sqrt{c^2-u^2}}\]
e il tempo che ci interessa \`e $2t_3$. Si ha dunque
\[2t_3=\frac{2L}{\sqrt{c^2-u^2}}\neq \frac{2Lc}{c^2-u^2}=t_1+t_2.\]
Ammettiamo allora di aver sbagliato qualche misura in modo tale che le distanze $BC$ e $BD$ non siano identiche. Si pu\`o ricavare
\[t_1+t_2-2t_3\approx \frac{2L_{BD}}c\pa{1+\frac{u^2}{c^2}}-\frac{2L_{BC}}c\pa{1+\frac{u^2}{2c^2}}.\]
Se ruotiamo l'esperimento, l'effetto \`e scambiare i valori di $L_{BD}$ e $L_{BC}$ ma in ogni caso non \`e stata misurata una differenza.
\end{proof}
\noindent
Cosa potrebbe star succedendo?
\setlength{\leftmargini}{0cm}
\begin{enumerate}
\item[$\boxed{Fizeau}$] Magari la terra trascina l'etere
\item[$\boxed{Fitzgerald}$] Magari le lunghezze parallele parallele alla direzione di modo si contraggono
\[L_{\parallel}=L_0\sqrt{1-\frac{u^2}{c^2}}.\]
Con questo cambiamento la differenza dei tempi trovata nell'esperimento di MIchelson-Morley effettivamente si annulla. 
\end{enumerate}
\setlength{\leftmargini}{0.5cm}

\section{Trasformazioni di Lorenz}
Scriviamo \[\gamma=\frac1{\sqrt{1-\frac{u^2}{c^2}}}\] e notiamo che la correzione di Fitzgerald corrisponde a dire $\gamma L_\parallel=L_0$. Deduciamo per il principio di relativit\`a che
\[x=\frac{x'}\gamma+ut,\qquad x'=\frac{x}\gamma-ut'\]
da cui ricaviamo le \textbf{trasformazioni di Lorenz}
\[\begin{cases}
x'=\gamma (x-ut)\\
y'=y\\
z'=z\\
t'=\gamma\pa{t-\frac{ux}{c^2}}
\end{cases}\]

\begin{remark}
$\gamma$ \`e sempre maggiore o uguale a $1$ in quanto $0\leq u\leq c$.
\end{remark}

\begin{remark}
Se $\frac uc\to 0$ allora $\gamma\to 1$ e quindi ritroviamo le trasformazioni di Galileo.
\end{remark}

\noindent 
\begin{proposition}[Trasformazioni di Lorenz per direzioni arbitrarie]\label{LorenzDirezioneArbitraria}
Per un moto rettilineo uniforme a velocit\`a $\vec u$ si ha
\[\begin{cases}
\vec x'=\vec x+(\gamma -1)\frac{\vec x\cdot \vec u}{u^2}\vec u-\gamma t \vec u\\
t'=\gamma\pa{t-\frac{\vec u\cdot \vec x}{c^2}}
\end{cases}\]
\end{proposition}
\begin{proof}
Basta notare che la componente parallela al moto \`e
\[\vec x_\parallel=\frac{\vec x\cdot \vec u}{u}\frac{\vec u}{u}\]
mentre quella perpendicolare \`e $\vec x_\perp=\vec x-\vec x_\parallel$.
\end{proof}

\begin{remark}[Forma pi\`u simmetrica delle trasformazioni di Lorenz]
Se definiamo $\beta=\frac uc$ allora osserviamo che
\[\begin{cases}
x'=\gamma (x-\beta ct)\\
y'=y\\
z'=z\\
ct'=\gamma\pa{ct-\beta x}
\end{cases}\]
\end{remark}

\begin{example}[Equazione delle onde]
Consideriamo l'equazione
\[\nabla^2\phi=\frac1{c^2}\pp[2]t\phi.\]
Si ha che questa equazione \`e invariante per le trasformazioni di Lorenz.
\end{example}
\begin{proof}
ESERCIZIO
\end{proof}


\begin{definition}[Intervallo invariante]
L'\textbf{intervallo invariante} \`e
\[(ct)^2-x^2-y^2-z^2.\]
\end{definition}
\noindent Questa quantit\`a \`e detta intervallo invariante perch\'e \`e invariante rispetto alle trasformazioni di Lorenz.

\begin{remark}
Se stiamo misurando della luce allora l'intervallo invariante vale $0$.
\end{remark}


\section{Dilatazione dei tempi e contrazione delle lunghezze}
\begin{definition}[Intervallo di tempo proprio]
Consideriamo due eventi che nel sistema di riferimento $S'$ hanno la stessa posizione. L'intervallo di tempo misurato da un orologio fermo rispetto a $S'$ tra questi \`e detto \textbf{intervallo di tempo proprio} tra i due
\[\Delta t'=\Delta t_0\]
\end{definition}

\noindent
Sia $\Delta t'=t_1'-t_2'$ l'intervallo di tempo proprio tra due eventi. Cambiando sistema di riferimento troviamo
\[\Delta t=t_1-t_2=\frac{\Delta t'+\cancelto0{\Delta x'} \frac{u}{c^2}}{\sqrt{1-\frac {u^2}{c^2}}}=\gamma \Delta t'\]
dunque l'intervallo di tempo misurato da un sistema in movimento rispetto alla posizione dei due eventi \`e maggiore rispetto a quello misurato da un sistema che li vede alla stessa posizione.\\
Questo fenomeno \`e detto \textbf{dilatazione dei tempi}.
\bigskip


\begin{definition}[Lunghezza propria]
Fissiamo un sistema di riferimento $S'$ e consideriamo una distanza $\Delta x'$ tra due punti fermi in questo sistema. Questa \`e detta la \textbf{lunghezza propria}.
\end{definition}
Sia $\Delta x'$ la distanza misurata tra due punti fermi rispetto a $S'$ in $S'$. Cambiando sistema di riferimento troviamo
\[\Delta x=x_1-x_2=\frac{\Delta x'+u\Delta t'}{\sqrt{1-\frac{u^2}{c^2}}}\]
Affinch\'e la misura di questa lunghezza abbia senso in $S$ dovremo avere $\Delta t=0$ (mentre $\Delta t'$ a priori pu\`o essere qualsiasi valore, tanto la distanza in $S'$ non dipende dal tempo). Consideriamo dunque la trasformazione inversa
\[\Delta x'=\frac{\Delta x+u\cancelto0{\Delta t}}{\sqrt{1-\frac{u^2}{c^2}}}=\gamma \Delta x\]
dunque $\Delta x=\gamma\ii \Delta x'$, cio\`e la lunghezza misurata in $S$ \`e pi\`u piccola rispetto a quella in $S'$.\\
Questo fenomeno \`e detto \textbf{contrazione delle lunghezze}.

\begin{example}[Relativit\`a della simultaneit\`a]
Se due osservatori sono in moto l'uno rispetto all'altro e uno dei due misura due eventi con posizioni diverse ma allo stesso istante allora il secondo osservatore vede i due eventi come non simultanei. Questo segue immediatamente da
\[\Delta t'=\gamma\pa{\Delta t-\frac{u\Delta x}{c^2}}=-\frac{\gamma u}{c^2} \Delta x\neq 0.\]
\end{example}



\section{Addizione delle velocit\`a}
\begin{proposition}[Formula del Boost]\label{FormulaBoost}
Se un oggetto si muove parrallelo al moto tra due sistemi di riferimento inerziali allora
\[v=\dfrac{v'+u}{\displaystyle 1+\frac{uv'}{c^2}}.\]
\end{proposition}
\begin{proof}
Consideriamo $x'(t')=v't'$ e portiamo nel sistema $S$
\[x=\gamma(x'+ut')=\gamma(v'+u)t'=\gamma^2(v'+u)\pa{t-\frac{ux}{c^2}}\]
dunque
\[x\pa{1+\frac u{c^2}\gamma^2(v'+u)}=\gamma^2(v'+u)t,\]
da cui la formula voluta
\[x=\frac{v'+u}{1+\frac{uv'}{c^2}}t.\]
\end{proof}

\begin{remark}
Se approssimiamo $\frac{u}{c}\to 0$ e $\frac{v'}{c}\to 0$ allora ritroviamo il Boost Galileiano.
\end{remark}

\begin{remark}
Se $v'=c$ oppure $u=c$ troviamo $v=c$, che \`e l'assioma sulla costanza della velocit\`a della luce.
\end{remark}

\begin{remark}
Se il moto non \`e allineato con quello dei sistemi troviamo
\[\begin{cases}
v_x=\dfrac{v_x'+u}{1+\frac{v_x' u}{c^2}}\\\\
v_y=\dfrac{v_y'}{\gamma\pa{1+\frac{v_x' u}{c^2}}}\\\\
v_z=\dfrac{v_z'}{\gamma\pa{1+\frac{v_x' u}{c^2}}}
\end{cases}\]
\end{remark}

\noindent La formula vettoriale per il boost di Lorenz \`e
\[\vec v=\frac1{1+\frac{\vec u\cdot \vec v'}{c^2}}\pa{\frac1\gamma \vec v'+\vec u+\pa{1+\frac1\gamma}\frac{\vec u\cdot \vec v'}{c^2}\vec u}\]
o equivalentemente
\[\vec v=\frac1{1+\frac{\vec u\cdot \vec v'}{c^2}}\pa{\vec v'+\vec u+\frac1{c^2}\frac{\gamma}{1+\gamma} \vec u\times (\vec u\times \vec v')}\]


\begin{remark}
Se interpretiamo $\gamma$ come una funzione $\gamma(u)=\pa{1+\frac{u^2}{c^2}}^{-\frac12}$ allora ricaviamo
\[\gamma(v)=\gamma(u)\gamma(v')\pa{1+\frac{\vec u\cdot \vec v'}{c^2}}\]
\end{remark}


\begin{example}
Consideriamo come cambia la velocit\`a della luce entro un mezzo con indice di rifrazione $n$ tra due sistemi in moto relativo a velocit\`a $u$:
\[v=\frac{c/n+u}{1+\frac{cu}{nc^2}},\]
da cui la differenza tra le velocit\`a \`e
\[\Delta v=v-\frac cn=\frac{u(1-n^{-2})}{1+\frac{uc}n},\]
che per $u\ll c$ si approssima a $u\pa{1-\frac1{n^2}}$.
\end{example}

\begin{definition}[Rapidit\`a]
Definiamo la \textbf{rapidit\`a} di un boost a velocit\`a $u$ come
\[\xi=\mathrm{arctanh}\pa{-\frac uc}\]
\end{definition}
\noindent
Ricordando la forma simmetrica delle trasformazioni di Lorenz
\[\begin{cases}
x'=\gamma (x-\beta ct)\\
y'=y\\
z'=z\\
ct'=\gamma\pa{ct-\beta x}
\end{cases},\]
il fatto che $\cosh^2\xi-\sinh^2\xi=1$ e che $(ct')^2-x'^2=(ct)^2-x^2$ si ha che
\[\begin{cases}
x'=\cosh\xi x+\sinh\xi ct\\
ct'=\cosh \xi +\sinh\xi x
\end{cases}\]
Segue che
\[\mat{ct'\\ x'}=\mat{\cosh \xi &\sinh \xi\\\sinh \xi & \cosh \xi}\mat{ct\\ x},\]
cio\`e il boost corrisponde ad una rotazione iperbolica.

\begin{remark}
Considerando una composizione di velocit\`a 
\[u=\dfrac{u_1+u_2}{1+\frac{u_1u_2}{c^2}}\]
allora le corrispondenti rapidit\`a si sommano, cio\`e
\[\xi=\xi_1+\xi_2.\]
\end{remark}

\section{Quadrivettori}
\begin{definition}[Quadrivettore]
Defniamo il \textbf{quadrivettore posizione} come
\[x^\mu=(ct,x,y,z).\]
In particolare $x^0=ct$, $x^1=x$, $x^2=y$ e $x^3=z$.
\end{definition}

Cerchiamo le trasformazioni lineari\footnote{Le cerchiamo lineari perch\'e vogliamo indipendenza dall'origine} che trasformano il quadrivettore preservando l'intervallo invariante.

\begin{definition}[Tensore metrico di Minkowski]
Definiamo il \textbf{Tensore metrico di Minkovski} tramite la matrice
\[\eta_{\mu\nu}=\mat{1&&&\\
&-1&&\\
&&-1&\\
&&&-1}\]
\end{definition}
\noindent
Osserviamo che $L$ preserva l'intervallo invariante se e solo se
\[\eta=L^\top\eta L.\]
\begin{remark}
Passando ai determinanti segue subito che $\det L=\pm 1$.
\end{remark}


\noindent Consideriamo l'equazione sopra in coordinate:
\[\sum_\nu\sum_\mu(L^\top)_{\al\mu}\eta_{\mu\nu}L_{\nu\beta}=\eta_{\al\beta}\]
nella convezione di Einstein possiamo evitare di scrivere i simboli di somma, dunque
\[L_{\mu\al}\eta_{\mu\nu}L_{\nu\beta}=\eta_{\al\beta}.\]
Osserviamo che queste equazioni non cambiano scambiando $\al$ e $\beta$.

Come convensione indici con lettere greche possono assumere valori tra $1$ e $4$, mentre indici latini solo tra $1$ e $3$.\\
Supponiamo che il moto avvenga lungo l'asse $x$ ($y'=y$, $z'=z$ e $x'$ non dipende da $y$ o $z$), cio\`e
\[L_{ij}=0 \quad i,j\in\cpa{1,2,3},\ i\neq j\]
Consideriamo ora vari casi:
\begin{itemize}
\item Se $\al=i$ e $\beta=j$ per $i\neq j$ allora
\[L_{\mu i}\eta_{\mu\nu}L_{\nu j}=\eta_{ij}=0\implies L_{02}=L_{03}=0\]
\item Se $\al=0$ e $\beta=j$ allora
\[L_{\mu0}\eta_{\mu\nu}L_{\nu j}=\eta_{0j}=0\implies L_{00}L_{0j}-L_{j0}L_{jj}.\]
Intuitivamente $L_{00}$ e $L_{jj}$ non sono nulli perch\'e altrimenti $x'$ non dipenderebbe da $x$ e similmente per le altre componenti, quindi abbiamo trovato
\[L_{j0}=L_{0j}\frac{L_{00}}{L_{jj}}\]
In particolare $L_{20}=L_{30}=0$ e $L_{10}=L_{01}\frac{L_{00}}{L_{11}}$.
\item Se $\al=\beta=0$ allora abbiamo
\[L_{00}^2-L_{10}^2=\eta_{00}=1\]
\item Se $\al=\beta=i$ allora
\[L_{0i}^2-L_{ii}^2=\eta_{ii}=-1,\]
dunque $L_{01}^2=L_{11}^2-1$ e $L_{22}^2=L_{33}^2=1$.
\end{itemize}
\noindent
Battezziamo $L_{11}=\gamma>0$ e notiamo che
\[\begin{cases}
L_{10}=L_{01}\frac{L_{00}}{\gamma}\\
L_{01}=\pm\sqrt{\gamma^2-1}\\
L^2_{00}=L^2_{10}+1
\end{cases}\]
da cui
\[L_{00}^2=L_{00}^2\pa{1-\frac1{\gamma^2}}+1\implies L_{00}=\pm\gamma\]
e $L_{10}=\pm L_{01}=\pm \sqrt{\gamma^2-1}$.
\medskip

\noindent Mettendo tutto insieme (quindi anche $\det L=1$) troviamo la seguente forma per $L$\footnote{abbiamo supposto $L_{00}>0$ perch\'e altrimenti futuro e passato si scambierebbero}:
\[L=\mat{\gamma &\pm\sqrt{\gamma^2-1}&\\\pm\sqrt{\gamma^2-1}&\gamma&\\&&1&\\&&&1}\]
Queste sono esattamente le trasformazioni di Lorenz, infatti se $\gamma=\pa{1-\frac{u^2}{c^2}}^{-\frac12}$ allora $\sqrt{\gamma^2-1}=\frac uc\gamma$.

\subsection{Diagramma di Minkowski}
[DISEGNINO]
\begin{definition}[Quadrivettore di tipo tempo/spazio luce]
Dato un quadrivettore $x^\mu$ definiamo la sua norma di Minkowski come \[s^2=x^\mu \eta_{\mu\nu}x^\nu=(x^0)^2-(x^1)^2-(x^2)^2-(x^3)^2.\]
Affermiamo che il quadrivettore \`e di tipo 
\begin{itemize}
\item \textbf{tempo} se $s^2>0$
\item \textbf{spazio} se $s^2<0$
\item \textbf{luce} se $s^2=0$
\end{itemize}
Se il quadrivettore \`e di tipo tempo allora esso appartiene al \textbf{futuro} se $x^0>0$ o al \textbf{passato} se $x^0<0$.
\end{definition}

\begin{remark}
Un quadrivettore ortogonale ad uno di tipo tempo \`e di tipo spazio, ma un quadrivettore ortogonale ad uno di tipo spazio non necessariamente \`e di tipo tempo.
\end{remark}

\begin{proposition}[La causalit\`a viene rispettata]
Consideriamo due eventi $A$ e $B$. Se $\Delta t>0$ e $\Delta t'<0$ allora i due eventi non possono essere l'uno la causa dell'altro, cio\`e il quadrivettore dato dalla loro differenza \`e di tipo spazio.
\end{proposition}
\begin{proof}
Applicando la trasformazione di Lorenz
\[\Delta t'=\gamma\pa{\Delta t-\frac u{c^2}\Delta x}\]
dunque se $\Delta t'<0$ e $\Delta t>0$ allora
\[-\gamma\Delta t>-\gamma\frac u{c^2}\Delta x\implies \Delta x>\frac{c^2}{u}\Delta t>c\Delta t\]
cio\`e la distanza tra gli eventi \`e maggiore rispetto alla distanza che la luce potrebbe percorrere in quell'intervallo di tempo, quindi i due eventi non possono essere l'uno la causa dell'altro.
\end{proof}

\noindent Per semplicit\`a ignoriamo le componenti $y$ e $z$. Definiamo una base ortonormale dello spazio di Minkowski:
\[\wt e_0=\mat{\cosh \xi\\\sinh\xi},\quad \wt e_1=\pm \mat{\sinh \xi\\\cosh\xi}.\]
Poich\'e $\cosh^2\xi-\sinh^2\xi=1$ per ogni $\xi$ effettivamente sono vettori di norma di Minkowski $1$, inoltre sono evidentemente ortogonali.
\bigskip

\noindent
Se $\wt e_0'$ e $\wt e_1'$ sono definiti come sopra ma a partire da una rapidit\`a $\xi'$ allora
\begin{align*}
\wt e_0\cdot \wt e_0'=&+\cosh(\xi-\xi')\\
\wt e_1\cdot \wt e_0'=&\pm \sinh(\xi-\xi')\\
\wt e_0\cdot \wt e_1'=&\mp\sinh(\xi-\xi')\\
\wt e_1\cdot \wt e_1'=&-\cosh(\xi-\xi')
\end{align*}

\noindent
Consideriamo ora un cambio di base
\[\wt a=x_0\wt e_0+x_1\wt e_1=x_0'\wt e_0'+x_1'\wt e_1'\]
Si pu\`o verificare che
\[\begin{cases}
\wt e_0=\cosh(\xi-\xi')\wt e_0'+\sinh(\xi-\xi')\wt e_1'\\
\wt e_1=\sinh(\xi-\xi')\wt e_0'+\cosh(\xi-\xi')\wt e_1'
\end{cases}\]
da cui ricaviamo
\[\begin{cases}
x_0'=x_0\cosh(\xi-\xi')+x_1\sinh(\xi-\xi')\\
x_1'=x_0\sinh(\xi-\xi')+x_1\cosh(\xi-\xi')
\end{cases}\]
\chapter{Cinematica relativistica}

\section{Introduzione ai quadrivettori}
\begin{definition}[Quadrivettore]
Consideriamo lo spazio vettoriale $\R^4$ con prodotto scalare indotto dal \textbf{tensore metrico di Minkovsky}
\[\eta_{\mu\nu}=\mat{1&&&\\
&-1&&\\
&&-1&\\
&&&-1}.\]
Un elemento di questo spazio \`e detto \textbf{quadrivettore}.
\end{definition}

\begin{definition}[Quadrivettore posizione]
Defniamo il \textbf{quadrivettore posizione} come
\[x^\mu=(ct,x,y,z).\]
In particolare $x^0=ct$, $x^1=x$, $x^2=y$ e $x^3=z$.\\
Come notazione scriviamo anche $\vec x=(x,y,z)$ e $\wt x=(ct,\vec x)$.
\end{definition}

\begin{notation}
Come notazione, se $\wt x$ \`e un quadrivettore posizione definiamo
\[s^2=\wt x\cdot \wt x=x^\mu\eta_{\mu\nu}x^\nu=(ct)^2-x^2-y^2-z^2.\]
\end{notation}

\begin{definition}[Quadrivettore di tipo tempo/spazio luce]
Affermiamo che un quadrivettore \`e di tipo 
\begin{itemize}
\item \textbf{tempo} se $s^2>0$
\item \textbf{spazio} se $s^2<0$
\item \textbf{luce} se $s^2=0$
\end{itemize}
Se il quadrivettore \`e di tipo tempo allora esso appartiene al \textbf{futuro} se $t>0$ o al \textbf{passato} se $t<0$.
\end{definition}

\begin{remark}
Un quadrivettore ortogonale ad uno di tipo tempo \`e di tipo spazio, ma un quadrivettore ortogonale ad uno di tipo spazio non necessariamente \`e di tipo tempo.
\end{remark}

\subsubsection{Causalit\`a}
\begin{proposition}[La causalit\`a viene rispettata]
Consideriamo due eventi $A$ e $B$. Se $\Delta t>0$ e $\Delta t'<0$ allora i due eventi non possono essere l'uno la causa dell'altro, cio\`e il quadrivettore dato dalla loro differenza \`e di tipo spazio.
\end{proposition}
\begin{proof}
Applicando la trasformazione di Lorenz
\[\Delta t'=\gamma\pa{\Delta t-\frac u{c^2}\Delta x}\]
dunque se $\Delta t'<0$ e $\Delta t>0$ allora
\[-\gamma\Delta t>-\gamma\frac u{c^2}\Delta x\implies \Delta x>\frac{c^2}{u}\Delta t>c\Delta t\]
cio\`e la distanza tra gli eventi \`e maggiore rispetto alla distanza che la luce potrebbe percorrere in quell'intervallo di tempo, quindi i due eventi non possono essere l'uno la causa dell'altro.
\end{proof}

\subsection{Derivazione delle trasformazioni di Lorenz}
Poich\'e le trasformazioni di Lorenz rappresentano un cambio di sistema di riferimento esse devono rispettare combinazioni lineari, cio\`e cerchiamo una trasformazione lineare.\\
Osserviamo che la norma di un quadrivettore posizione \`e esattamente l'intervallo invariante, quindi vogliamo che le trasformazioni siano isometrie per la metrica di Minkowski, cio\`e
\[\eta=L^\top\eta L.\]
\begin{remark}
Passando ai determinanti segue subito che $\det L=\pm 1$.
\end{remark}

\noindent Consideriamo l'equazione sopra in coordinate:
\[\sum_\nu\sum_\mu(L^\top)_{\al\mu}\eta_{\mu\nu}L_{\nu\beta}=\eta_{\al\beta}\]
nella convezione di Einstein possiamo evitare di scrivere i simboli di somma, dunque
\[\boxed{L_{\mu\al}\eta_{\mu\nu}L_{\nu\beta}=\eta_{\al\beta}}\]
Osserviamo che queste equazioni non cambiano scambiando $\al$ e $\beta$.\\
Come convenzione indici con lettere greche possono assumere valori tra $1$ e $4$, mentre indici latini solo tra $1$ e $3$.\\
Supponiamo che il moto avvenga lungo l'asse $x$ ($y'=y$, $z'=z$ e $x'$ non dipende da $y$ o $z$), cio\`e
\[L_{ij}=0 \quad \forall i,j\in\cpa{1,2,3},\ i\neq j\]
Consideriamo ora vari casi:
\begin{itemize}
\item Se $\al=i$ e $\beta=j$ per $i\neq j$ allora
\[L_{\mu i}\eta_{\mu\nu}L_{\nu j}=\eta_{ij}=0\implies L_{02}=L_{03}=0\]
\item Se $\al=0$ e $\beta=j$ allora
\[L_{\mu0}\eta_{\mu\nu}L_{\nu j}=\eta_{0j}=0\implies L_{00}L_{0j}-L_{j0}L_{jj}.\]
Intuitivamente $L_{00}$ e $L_{jj}$ non sono nulli perch\'e altrimenti $x'$ non dipenderebbe da $x$ e similmente per le altre componenti, quindi abbiamo trovato
\[L_{j0}=L_{0j}\frac{L_{00}}{L_{jj}}\]
In particolare $L_{20}=L_{30}=0$ e $L_{10}=L_{01}\frac{L_{00}}{L_{11}}$.
\item Se $\al=\beta=0$ allora abbiamo
\[L_{00}^2-L_{10}^2=\eta_{00}=1\]
\item Se $\al=\beta=i$ allora
\[L_{0i}^2-L_{ii}^2=\eta_{ii}=-1,\]
dunque $L_{01}^2=L_{11}^2-1$ e $L_{22}^2=L_{33}^2=1$.
\end{itemize}
\noindent
Battezziamo $L_{11}=\gamma>0$ e notiamo che
\[\begin{cases}
L_{10}=L_{01}\frac{L_{00}}{\gamma}\\
L_{01}=\pm\sqrt{\gamma^2-1}\\
L^2_{00}=L^2_{10}+1
\end{cases}\]
da cui
\[L_{00}^2=L_{00}^2\pa{1-\frac1{\gamma^2}}+1\implies L_{00}=\pm\gamma\]
e $L_{10}=\pm L_{01}=\pm \sqrt{\gamma^2-1}$.
\medskip

\noindent Mettendo tutto insieme (quindi anche $\det L=1$) troviamo la seguente forma per $L$\footnote{abbiamo supposto $L_{00}>0$ perch\'e altrimenti futuro e passato si scambierebbero}:
\[L=\mat{\gamma &\pm\sqrt{\gamma^2-1}&\\\pm\sqrt{\gamma^2-1}&\gamma&\\&&1&\\&&&1}\]
Queste sono esattamente le trasformazioni di Lorenz, infatti se $\gamma=\pa{1-\frac{u^2}{c^2}}^{-\frac12}$ allora $\sqrt{\gamma^2-1}=\frac uc\gamma$.


\subsection{Rapidit\`a}

\begin{definition}[Rapidit\`a]
Definiamo la \textbf{rapidit\`a} di un boost a velocit\`a $u$ come
\[\xi=\arctanh\pa{-\beta(u)}.\]
Segue che
\[\gamma(u)=\cosh\xi,\quad -\gamma(u)\beta(u)=\sinh\xi,\quad -\beta(u)=\tanh \xi.\]
\end{definition}
\begin{remark}
Si ha che
\[\mat{ct'\\ x'}=\mat{\cosh \xi &\sinh \xi\\\sinh \xi & \cosh \xi}\mat{ct\\ x},\]
cio\`e il boost corrisponde ad una rotazione iperbolica.
\end{remark}
\begin{remark}
L'identit\`a trigonometrica iperbolica
\[\cosh^2\xi-\sinh^2\xi=1\]
corrisponde a $\gamma^2(1-\beta^2)=1$, che \`e la definizione di $\gamma$.
\end{remark}


\begin{remark}
Considerando una composizione di velocit\`a 
\[u=\dfrac{u_1+u_2}{1+\frac{u_1u_2}{c^2}}\]
notiamo che le corrispondenti rapidit\`a si sommano\footnote{vedi le formule di prostaferesi per la tangente iperbolica (\ref{SommaTangenteIperbolica})}, cio\`e
\[\xi=\xi_1+\xi_2.\]
\end{remark}


\subsubsection{Diagramma di Minkowski}
Per semplicit\`a ignoriamo le componenti $y$ e $z$. Definiamo una base ortonormale dello spazio di Minkowski:
\[\wt e_0=\mat{\cosh \xi\\\sinh\xi},\quad \wt e_1=\pm \mat{\sinh \xi\\\cosh\xi}.\]
Poich\'e $\cosh^2\xi-\sinh^2\xi=1$, per ogni $\xi$ questi sono vettori di norma di Minkowski $1$, inoltre sono evidentemente ortogonali.
\bigskip

\noindent
Se $\wt e_0'$ e $\wt e_1'$ sono definiti come sopra ma a partire da una rapidit\`a $\xi'$ allora
\begin{align*}
\wt e_0\cdot \wt e_0'=&+\cosh(\xi-\xi')\\
\wt e_1\cdot \wt e_0'=&\pm \sinh(\xi-\xi')\\
\wt e_0\cdot \wt e_1'=&\mp\sinh(\xi-\xi')\\
\wt e_1\cdot \wt e_1'=&-\cosh(\xi-\xi')
\end{align*}

\noindent
Consideriamo ora un cambio di base
\[\wt a=x_0\wt e_0+x_1\wt e_1=x_0'\wt e_0'+x_1'\wt e_1'\]
Si pu\`o verificare che
\[\begin{cases}
\wt e_0=\cosh(\xi-\xi')\wt e_0'+\sinh(\xi-\xi')\wt e_1'\\
\wt e_1=\sinh(\xi-\xi')\wt e_0'+\cosh(\xi-\xi')\wt e_1'
\end{cases}\]
da cui ricaviamo
\[\begin{cases}
x_0'=x_0\cosh(\xi-\xi')+x_1\sinh(\xi-\xi')\\
x_1'=x_0\sinh(\xi-\xi')+x_1\cosh(\xi-\xi')
\end{cases}\]


\section{Quadrivelocit\`a e Quadriaccelerazione}
\subsection{Tempo proprio e quadrivelocit\`a}
Poich\'e il tempo non \`e invariante per trasformazioni di Lorenz, non avrebbe senso definire una velocit\`a relativistica in termini solo del tempo. La generalizzazione giusta \`e la seguente

\begin{definition}[Tempo proprio]
Definiamo il \textbf{tempo proprio} come
\[\tau=\frac{\sqrt{s^2}}c.\]
Equivalentemente chiediamo che
\[d\tau^2=\frac{ds^2}{c^2}=dt^2-\frac1{c^2}\pa{dx^2+dy^2+dz^2}\]
\end{definition}

\begin{proposition}[Derivata del tempo rispetto al tempo proprio]\label{DerivataTempoPerTempoProprio}
Si ha che
\[\dd \tau t=\gamma(v),\]
dove $v=\sqrt{\pa{\dd tx}^2+\pa{\dd ty}^2+\pa{\dd tz}^2}$.
\end{proposition}
\begin{proof}
Sapendo che $d\tau^2=dt^2-\frac1{c^2}\pa{dx^2+dy^2+dz^2}$, ricaviamo che
\[d\tau^2=dt^2\pa{1-\frac{v^2}{c^2}}=\frac{dt^2}{\gamma(v)^2},\]
cio\`e
\[\dd \tau t=\gamma(v).\]
\end{proof}

\noindent Grazie al tempo proprio possiamo definire l'analogo relativistico della velocit\`a

\begin{definition}[Quadrivelocit\`a]
Definiamo la \textbf{quadrivelocit\`a} come
\[\wt v=\dd \tau{\wt x}=\dd \tau t\dd t{\wt x}=\gamma(v)\pa{c\dd tt,\dd t{\vec x}}=(c\gamma(v),\gamma(v)\vec v)=\gamma(v)(c,\vec v),\]
dove $\tau$ \`e il tempo proprio e $\vec v$ \`e la velocit\`a in senso non relativistico, cio\`e $\vec v=\dd t{\vec x}$.\\
La quantit\`a $\dd \tau{\vec x}=\gamma(v)\vec v$ \`e detta \textbf{velocit\`a propria}, \textbf{celerit\`a} o \textbf{velocit\`a relativistica}.
\end{definition}

\begin{remark}
La $4$-upla $(c,\vec v)$ NON \`e un quadrivettore perch\'e non \`e invariante cambiando sistema di riferimento, il fattore $\gamma(v)$ \`e dunque necessario per questo.
\end{remark}

\begin{remark}
Il modulo di Minkovski di una quadrivelocit\`a \`e
\[\wt v\cdot \wt v=c^2\gamma(v)^2-\vec v^2\gamma(v)^2=\gamma(v)^2(c^2-\vec v\cdot \vec v)=c^2\gamma(v)^2\pa{1-\frac{v^2}{c^2}}=c^2.\]
Segue in particolare che ogni quadrivelocit\`a \`e di tipo tempo.
\end{remark}

\noindent Se consideriamo un boost di velocit\`a $u$ parallela all'asse $x$ allora\footnote{$\gamma(u)$ e $\beta(u)$ non dipendono da $\tau$.}
\[\begin{cases}
\wt v_0'=\gamma(u)(\wt v_0-\beta(u)\wt v_1)\\
\wt v_1'=\gamma(u)(\wt v_1-\beta(u)\wt v_0)\\
\wt v_2'=\wt v_2\\
\wt v_3'=\wt v_3
\end{cases}\]

\begin{proposition}[Identit\`a tra i fattori $\gamma$]
Vale la seguente identit\`a
\[\gamma(v')=\gamma(u)\gamma(v)\pa{1-\frac{\vec u\cdot \vec v}{c^2}}\]
\end{proposition}
\begin{proof}
Osserviamo che
\[c\gamma(v')=\wt v_0'=\gamma(u)(\wt v_0-\beta(u)\wt v_1)=\gamma(u)(c\gamma(v)-\beta(u)\gamma(v)v_1),\]
dunque
\[\gamma(v')=\gamma(u)\gamma(v)\pa{1-\frac{v_1 u}{c^2}},\]
che \`e la tesi perch\'e supponiamo che $\vec u$ abbia la stessa direzione dell'asse $x$.
\end{proof}

\subsection{Quadriaccelerazione}
\begin{definition}[Quadriaccelerazione]
Definiamo la \textbf{quadriaccelerazione} come
\[\wt a=\dd\tau{\wt v}=\dd t{\wt v}\dd\tau t=\gamma(v)\dd t{}(c\gamma(v),\vec v\gamma(v))\]
\end{definition}
\begin{lemma}
Vale la seguente identit\`a
\[\boxed{\dd t{\gamma(v)}=\gamma(v)^3\frac{\vec v\cdot \vec a}{c^2}}\]
\end{lemma}
\begin{proof}
Calcoliamo:
\[\dd t\gamma=\dd t{}\pa{1-\frac{\vec v\cdot \vec v}{c^2}}^{-\frac12}=-\frac12\pa{1-\frac{\vec v\cdot \vec v}{c^2}}^{-\frac32}\pa{-\frac1{c^2}2\vec v\cdot \dd t{\vec v}}=\gamma(v)^3\frac{\vec v\cdot \vec a}{c^2},\]
in definitiva
\[{\dd t{\gamma(v)}=\gamma(v)^3\frac{\vec v\cdot \vec a}{c^2}}.\]
\end{proof}
\begin{corollary}
Segue dal lemma che
\[\dd t{}(\gamma(v)\vec v)=\gamma(v)^3\frac{\vec v\cdot \vec a}{c^2}\vec v+\gamma(v)\vec a.\]
\end{corollary}

\begin{remark}
Per quanto appena detto
\[\wt a=\pa{\gamma(v)^4\frac{\vec v\cdot \vec a}{c},\gamma(v)^4\frac{\vec v\cdot \vec a}{c^2}\vec v+\gamma(v)^2\vec a}=\gamma(v)^4\pa{\frac{\vec v\cdot \vec a}{c},\vec a+\frac{\vec v\times(\vec v\times \vec a)}{c^2}}\]
\end{remark}

\begin{remark}
Se $\vec a=0$ allora $\wt a=0$.
\end{remark}

\begin{remark}[Accelerazione propria]
Se ci troviamo un un sistema di riferimento a riposo ($\vec v=0$) allora $\gamma(v)=1$ e
\[\wt a=(0,\vec a).\]
In analogia a quanto detto possiamo definire $(0,\vec a)$ come l'\textbf{accelerazione propria}.
\end{remark}

\begin{remark}
Se l'accelerazione avviene lungo la direzione della velocit\`a allora
\[\wt a=\gamma(v)^4\abs{\vec a}(\beta(v),\wh v)\]
\end{remark}

\begin{remark}
Calcoliamo il seguente scalare, invariante per trasformazioni di Lorenz
\[\wt a\cdot \wt v=\dd[2]\tau{\wt x}\cdot \dd \tau{\wt x}=\frac12\dd\tau{}\pa{\dd \tau{\wt x}}^2=\frac 12\dd\tau{}(\wt v^2)=\frac 12\dd\tau c=0,\]
cio\`e la quadriaccelerazione e la quadrivelocit\`a sono sempre ortogonali per la metrica di Minkowski.
\end{remark}

\begin{remark}
Calcoliamo il modulo della quadrivelocit\`a
\[\wt a^2=\cdots=-\gamma(v)^4\vec a\cdot\vec a-\gamma(v)^6\frac{(\vec v\cdot \vec a)^2}{c^2}\]
in particolare $\wt a$ \`e un quadrivettore di tipo spazio.
\end{remark}

\begin{remark}
Supponiamo che $\vec v\perp \vec a$, allora
\[\wt a^2=-\gamma(v)^4\vec a\cdot \vec a,\]
che possiamo trovare dall'osservazione precedente o sfruttando l'equazione \[\wt a=\pa{\gamma(v)^4\frac{\vec v\cdot \vec a}{c},\gamma(v)^4\frac{\vec v\cdot \vec a}{c^2}\vec v+\gamma(v)^2\vec a}=(0,\gamma(v)^2\vec a)\]
\end{remark}


\section{Quadrigradiente}

Ricordiamo che
\[\begin{cases}
\pp t{}=\gamma\pp{t'}{}-\gamma u\pp{x'}{}\\
\pp x{}=-\gamma \frac{u}{c^2}\pp{t'}{}+\gamma \pp{x'}{}\\
\pp y{}=\pp {y'}{}\\
\pp z{}=\pp {z'}{}
\end{cases}\]

\begin{definition}[Quadrigradiente]
Definiamo la \textbf{quadridivergenza} come
\[\wt \nabla=\pa{\frac1c\dd t{},-\pp x{},-\pp y{},-\pp z{}}.\]
\end{definition}

\begin{remark}
Consideriamo l'equivalente del laplaciano
\[\wt \nabla\cdot\wt\nabla=\frac1{c^2}\pps[2]t{}-(-\vec\nabla)(-\vec \nabla)=\frac1{c^2}\pps[2]t{}-\nabla^2=\square\]
questo \`e l'operatore \textbf{dalembertiano}.
\end{remark}


\chapter{Dinamica relativistica}
\section{Leggi di Newton relativistiche}
Cerchiamo di capire come leggere le leggi di Newton in chiave relativistica.\\
La prima legge \`e il primo postulato della relativit\`a ma con la seconda legge ($F=ma$) cominciamo ad avere qualche problema.

\subsection{Secondo principio della dinamica}
Per capire come procedere consideriamo il seguente esperimento

\begin{example}[Urto elastico di due palline identiche]
Fissiamo un sistema $S$ e cosideriamo un sistema $S'$ in moto relativo lungo l'asse $x$ a velocit\`a $u$. Due osservatori laciano delle palline ($1$ e $2$) lungo la direzione $y$ con la stessa velocit\`a in modo che queste si scontrino.\smallskip

\noindent Formalmente imponiamo
\[\begin{cases}
(v_1)_y=v_0\\
(v_1)_x=0\\
(v_2)_y'=-v_0\\
(v_2)_x'=0
\end{cases}\]
\begin{figure}[!htb]
    \centering
    \includegraphics[width=9cm]{images/Palline_relativistiche_def_massa.png}
\end{figure}

\noindent
Considerando le formule di boost troviamo
\[\begin{cases}
(v_2)_x=\frac{0+u}{1+0}=u\\
(v_2)_y=\frac{-v_0}{\gamma(u)(1+0)}=-\frac{v_0}{\gamma(u)}.
\end{cases}\]
Imponiamo la conservazione degli impulsi (potenzialmente ammettendo che la massa dipenda dalla velocit\`a) lungo l'asse $y$:
\[2M(v_1)v_0=2M(v_2)\frac{v_0}{\gamma(u)},\]
da cui $\gamma(u)M(v_1)=M(v_2)$.\medskip

\noindent Ricordando\footnote{nota che $\vec u$ e $\vec v_1$ sono ortogonali} che $\gamma(v_2)=\gamma(v_1)\gamma(u)$ osserviamo che
\[\frac{M(v_2)}{\gamma(v_2)}=\frac{M(v_1)}{\gamma(v_1)}=\frac{M(0)}{\gamma(0)}=M(0)=m.\]

\end{example}

\begin{definition}[Massa a riposo]
Definiamo la \textbf{massa a riposo} $m$ come la massa misurata in un sistema dove il corpo \`e a riposo.
\end{definition}

\begin{definition}[Impulso relativistico]
Definiamo l'\textbf{impulso relativistico} come
\[\vec p=M(v)\vec v=m\gamma(v)\vec v.\]
\end{definition}
\begin{remark}
Con l'aumentare della velocit\`a, la ``massa effettiva" cresce molto fino a diventare ``infinita" per una velocit\`a che tende verso la velocit\`a della luce.
\end{remark}

\begin{definition}[Quadriimpulso]
Definiamo il \textbf{quadriimpulso} come
\[\wt p=m\wt v=(M(v)c,\vec p)=m\gamma(v)(c,\vec v).\]
L'impulso relativistico \`e la componente spaziale del quadriimpulso.
\end{definition}

\begin{remark}
Il modulo del quadriimpulso \`e
\[\wt p^2=m^2\wt v^2=m^2c^2.\]
In particolare il quadriimpulso \`e di tipo tempo (o di tipo luce se $m=0$).
\end{remark}

\noindent Abbiamo capito chi \`e la componente spaziale del quadriimpulso, ma la componente temporale?
\[\wt p_0c=m\gamma(v)c^2=m\gamma(v)c^2=\frac{mc^2}{\sqrt{1-\frac{v^2}{c^2}}}\]
per $v\to 0$ possiamo approssimare questa quantit\`a al primo ordine in $v^2$ come
\[mc^2-\under{\text{energia cinetica}}{\frac12 mv^2}\]

\begin{definition}[Energia]
Definiamo l'\textbf{energia} come $E=\wt p_0c$. In particolare se $v=0$ allora l'energia vale $mc^2$, che chiamiamo \textbf{energia di riposo}.
\end{definition}

\begin{remark}
L'esistenza dell'energia di riposo significa che se esiste un processo che fa diminuire la massa di un oggetto allora per conservazione dell'energia viene liberata dell'energia.\\
Questi processi esistono e vengono impiegati per l'estrazione di energia atomica.\\
Esistono processi che distruggono interamente la massa, per esempio lo scontro tra una particella e la sua antiparticella.
\end{remark}

\begin{remark}
Una volta definita l'energia possiamo riscrivere il quadriimpulso come
\[\wt p=m\wt v=(E/c,\vec p).\]
\end{remark}


\begin{remark}
Se cambiamo sistema di riferimento, l'energia cambia come segue:
\[E'=m\gamma(v')c^2=m\gamma(v)\gamma(u)(c^2-uv)=\gamma(u)(E-up_x).\]
Osserviamo che questo \`e compatibile col fatto che $\wt p$ \`e un quadrivettore, cio\`e rispetta le trasformazioni di Lorenz:
\begin{align*}
p'_x=&\frac{E' v'}{c^2}=\frac{(m\gamma(v)\gamma(u)(c^2-uv))\pa{\dfrac{v-u}{c^2-uv}c^2}}{c^2}=\\
=&m\gamma(v)\gamma(u)(v-u)=\\
=&\gamma(u)\pa{p_x-u\frac{E}{c^2}}.
\end{align*}
\end{remark}

\begin{remark}
Calcoliamo il modulo del quadriimpulso
\[m^2c^2=\wt p^2=\pa{\frac Ec}^2-(\vec p)^2,\]
cio\`e
\[\boxed{E^2=(\vec p)^2c^2+m^2c^4}\]
\end{remark}

\begin{remark}
Poich\'e $v=\frac{\abs{\vec p}c^2}E$ si ha che $\beta(v)=\frac{\abs{\vec p}c}E$ e quindi
\[\boxed{\gamma=\frac{E}{mc^2}}\]
ovvero $E=\gamma(v)mc^2$.
\end{remark}

\begin{definition}[Energia cinetica]
Definiamo l'\textbf{energia cinetica relativistica} come
\[T=E-mc^2=mc^2(\gamma(v)-1).\]
\end{definition}
\begin{remark}
Se $v$ \`e piccolo $\gamma{v}\approx 1+\frac12\frac{v^2}{c^2}$, da cui $T\approx mc^2\frac 12\frac{v^2}{c^2}=\frac12 mv^2$.
\end{remark}

\noindent Consideriamo ora questa domanda: \`e possibile che la luce sia composta da particelle?\medskip

\noindent In questo caso $\gamma(v)=\gamma(c)=\infty$, quindi affinch\'e l'energia di questa ipotetica particella abbia senso, $m=0$. In questo caso avremmo
\[E=\abs{\vec p}c.\]
Queste particelle sono dette \textbf{fotoni}.

\subsection{Terzo principio della dinamica}
Consideriamo adesso la questione delle forze uguali e contrarie. Se queste forze opposte vengono applicate allo stesso punto e allo stesso istante non abbiamo problemi.\medskip

\noindent Se le forze uguali e contrarie agiscono su punti diversi, la relativit\`a della simultaneit\`a comincia a creare problemi su \textit{quando} le due forze sono uguali e contrarie.\medskip

\noindent La soluzione \`e osservare che quello che vogliamo \`e che la variazione totale dell'impulso sia nulla. La forza deriva da variazioni di impulsi
\[\vec F_A=\dd t{\vec p_A},\quad \vec F_B=\dd t{\vec p_B},\quad \dd t{}(\vec p_A+\vec p_B)=0.\]
\bigskip

\noindent Un altro modo di interpretare la questione della simultaneit\`a \`e pensare a campi di forze piuttosto che forze generate direttamente da oggetti.

\begin{definition}[Quadriforza]
Definiamo la \textbf{quadriforza} come
\[\wt F=\dd \tau{\wt p}=m\dd\tau{\wt v}=m\wt a.\]
Una forma equivalente \`e 
\[\wt F=\dd \tau{\wt p}=\gamma(v)\pa{\frac1c\dd tE,\vec F},\]
dove $\vec F=\dd t{\vec p}$.
\end{definition}

\begin{remark}
Poich\'e $\wt v\cdot \wt a=0$ si ha che $\wt v\cdot\wt F=0$. Scrivendo esplicitamente la seconda equazione
\[\vec F\cdot \vec v=\dd tE=mc^2\dd t\gamma=m\gamma^3 \vec v\cdot \vec a.\]
\end{remark}

\begin{remark}
Troviamo le seguenti identit\`a
\[\wt F=\gamma(v)\pa{\frac{\vec F\cdot \vec v}c,\vec F}\]
\[\vec F=\dd t{\vec p}=\dd t{}(m\vec v\gamma)=m\gamma\vec a+\frac{\vec F\cdot \vec v}{c^2}\vec v,\]
in particolare \underline{$\vec F$ non \`e allineata con $\vec a$ in generale}.
\end{remark}

\begin{remark}
Osserviamo che
\[\wt F\cdot \wt v=\dd\tau{}(m\wt v)\cdot \wt v=\dd\tau m\wt v^2+m\ \under{=0}{\wt a\cdot \wt v}=c^2\dd\tau m.\]
\end{remark}

\section{Effetto Doppler}
Supponiamo che un sistema $S'$ emetta delle onde con frequenza $\nu'$ e lunghezza d'onda $\la'$ constanti (quindi i fronti d'onda hanno velocit\`a $v'=\la'\nu'$ rispetto a $S'$).

\subsection{Versione classica}
Se $S'$ si muove con velocit\`a $u$ osserviamo che
\[\la=v\Delta t'-u\Delta t'=(v-u)\Delta t'=\frac{v-u}{\nu'},\]
da cui
\[\nu=\frac v\la=\frac1{1-u/v}\nu',\]
quindi la frequenza dell'onda misurata da $S$ \`e cambiata di un fattore che dipende dalla velocit\`a relativa tra i sistemi.\medskip

\noindent Supponendo ora che sia $S$ ad emettere l'onda notiamo con calcoli analoghi che
\[\nu'=\pa{1-\frac uv}\nu,\]
che come notazione \`e la stessa di prima ma la frequenza ricevuta e quella emessa hanno scambiato i ruoli.

\subsection{Versione relativistica}
Consideriamo il caso di una sorgente in movimento $S'$ che emette l'onda (a velocit\`a $v$ in $S'$) e che si muove verso $S$.\\
Fissiamo degli eventi:
\begin{enumerate}
\item \ul{Emissione di un fronte d'onda}: In $S'$ questo evento ha coordinate $(t_e',0)$.
\item \ul{Ricezione del fronte d'onda}: In $S'$ questo evento ha coordinate $(t'_r,x'_r)$ dove $t'_r=t_e'+\delta t'$ e $x'_r=v\delta t'$. Osserviamo per\`o che $x'_r$ \`e anche l'origine di $S$ vista in $S'$. Se $x'_0$ \`e la coordinata dell'origine di $S$ al tempo $t'=0$ si ha che $x'_r=x_0'-ut_r'$.
\end{enumerate}
Mettendo tutto insieme
\[\delta t'=\frac{x'_0-ut'_e}{u+v}.\]
Consideriamo adesso due emissioni (1 e 2), una a $t_e'=0$ e $t'_e=T$. Chiaramente $\Delta t_e'=T$, mentre
\begin{align*}
\Delta t'_r=&(T+\delta t'(T)) - (0+\delta t'(0))=\\
=&T+\frac{x'_0-uT}{u+v}-\frac{x'_0}{u+v}=\\
=&T\pa{1-\frac u{u+v}}=\\
=&T\frac v{u+v}.
\end{align*}
Applichiamo adesso una trasformazione di Lorenz per passare al sistema $S$:
\[\Delta t'_r=\gamma(\Delta t_r-u\under{=0}{\Delta x_r})=\gamma \Delta t_r,\]
quindi
\[\Delta t_r=\frac T\gamma\frac v{u+v}.\]
Consideriamo ora il caso di onde luminose ($v=c$)
\[\frac1{\Delta t_r}=\nu=\nu'\gamma\frac{u+c}c=\sqrt{\frac{1+\beta(u)}{1-\beta(u)}}\nu'.\]
\bigskip

\noindent Per il principio di relativit\`a, se $S'$ si stesse allontanando e $S$ fosse la sorgente troviamo 
\[\nu'=\sqrt{\frac{1-\beta(u)}{1+\beta(u)}}\nu.\]

\bigskip

\noindent Possiamo riassumere entrambe le formule in
\[\boxed{\nu_{\mathrm{ricevuta}}=\sqrt{\frac{1+\beta}{1-\beta}}\nu_{\mathrm{emessa}}}\]



\section{Particelle}

\begin{definition}[Elettron-Volt]
Definiamo l'\textbf{elettron-volt} come
\[\mathrm{eV}=1.6\cdot 10^{-19}\ \mathrm{J}.\]
Quando lavoriamo in questo sistema possiamo misurare la quantit\`a di moto in $\mathrm{eV}/c$ e la massa in $\mathrm{eV}/c^2$.
\end{definition}


\begin{definition}[Unit\`a naturali]
Pu\`o essere comodo porre $c=1$ per semplificare le formule. In queste unit\`a di misura
\[[\text{Energia}]=[\text{Quantit\`a di moto}]=[\text{Massa}].\]
\end{definition}






\section{Elettro-Magnetismo}
Ricordiamo la \textbf{Forza di Lorentz}
\[\vec F=q(\vec E+\vec v\times \vec B)\]

\begin{example}
Consideriamo un filo entro il quale scorre una corrente $I$. Vicino al filo abbiamo una carica $q<0$ che si muove a velocit\`a $v_0$ parallela al filo e in direzione opposta a $I$

\begin{figure}[!htb]
    \centering
    \includegraphics[width=9cm]{images/Filo_e_carica.png}
\end{figure}

\noindent
Il campo magnetico vale
\[\vec B=\frac1{4\pi \e_0c^2}\frac{2 \vec I\times \wh r}r,\]
da cui, se il filo \`e complessivamente neutro,
\[\vec F=q\vec v_0\times \vec B=\frac1{4\pi \e_0c^2}\frac{2 qIv_0}r.\]
Mettiamoci ora in un sistema dove $q$ ha velocit\`a nulla. Abbiamo una forza di Lorentz?
\end{example}
















\part{Accenni di meccanica quantistica}
\chapter{Problema del corpo nero}

\section{Richiami su energia trasportata da onde}
\begin{definition}[Intensit\`a]
L'\textbf{intensit\`a} di un'onda \`e data da
\[I=\frac{dE}{dA_\perp dt}\qquad\qquad [I]=\frac{\mathrm{W}}{\mathrm{m}^{2}}.\]
La \textbf{radianza} \`e l'intensit\`a per angolo solido
\[I_\Omega=\frac{dE}{dA_\perp dtd\Omega}=\dd \Omega I.\]
\end{definition}

\begin{definition}[Emittanza e Irradianza]
L'\textbf{emittanza} come la potenza emessa da una supeficie per unit\`a di superficie. Si indica con $M$.\\
L'\textbf{irradianza} \`e la potenza ricevuta su una superficie per unit\`a di superficie. Anche questa si indica con $I$.
\end{definition}

\section{Pressione di radiazione}
Consideriamo un'onda elettromagnetica che incide su una superficie conduttrice in modo ortogonale. Sia $\sigma$ la densit\`a di carica superficiale. Se la superficie ha area $A$ allora possiamo scrivere la forza di Lorentz come
\[\frac{\vec F}A=\sigma(\vec E+\vec v\times \vec B).\]
L'onda induce un movimento delle cariche interno alla superficie e $\vec v\times \vec B$ ha verso sempre diretto verso la superficie

\begin{figure}[!htb]
    \centering
    \includegraphics[width=5cm]{images/onda_elettromagnetica_che_incide_ortogonalmente.png}
\end{figure}

\noindent
Calcoliamo l'intensit\`a assorbita\footnote{potenza per area} e la pressione esercitata sulla superficie
\[\begin{cases}
~\\
I_{\text{assorbita}}=\dfrac{\vec F\cdot \vec v}A=\sigma \vec E\cdot \vec v+\cancelto{0}{\sigma \vec v\cdot(\vec v\times \vec B)}=\sigma \vec E\cdot \vec v\\\\
p\overset{\vec E\text{ parallelo}}{\overset{\text{a superficie}}=}\sigma\abs{\vec v\times \vec B}=\sigma vB=\sigma v\dfrac Ec=\dfrac{I_{\text{assorbita}}}c\\~
\end{cases}\]
Se la superficie assorbe tutta l'energia che riceve
\begin{align*}
&\Delta E=I_{\text{assorbita}}\Delta t A\\
&\abs{\vec p}=\frac1cI_{\text{assorbita}}\Delta t A
\end{align*}
Se la superficie riflette tutta l'energia che riceve
\begin{align*}
&\Delta E=0\\
&\abs{\vec p}=2\frac1cI_{\text{assorbita}}\Delta t A
\end{align*}

\subsection{Gas di onde elettromagnetiche}
Consideriamo una semisfera dentro la quale ci sono onde elettromagnetiche distribuite in modo omogeneo e isotropo. Sia $w$ la densit\`a di energia nella semisfera (ben definita per omogeneit\`a).\\
Consideriamo poi una superficie $A$ come in figura perfettamente assorbente e cerchiamo di capire quanta energia e quantit\`a di moto riceve per unit\`a di tempo.

\begin{figure}[!htb]
    \centering
    \includegraphics[width=5cm]{images/semisfera_luce.png}
\end{figure}

\noindent
Consideriamo un ragionamento analogo a quello fatto per il modello dei gas perfetti:\\
Fissata una direzione consideriamo il cilindro di particelle che si scontrano contro la nostra area in una unit\`a di tempo $dt$

\begin{figure}[!htb]
    \centering
    \includegraphics[width=2.5cm]{images/cilindretto_luce.png}
\end{figure}

\noindent
Il volume del cilindretto \`e $A cdt\cos \theta$, dunque l'energia che colpisce $A$ \`e
\[w Acdt\cos \theta.\]
Segue che tutta l'energia che raggiunge $A$ in un intervallo di tempo $dt$ \`e l'integrale sull'angolo solido diviso per due\footnote{stiamo considerando solo met\`a sfera} della quantit\`a trovata, cio\`e
\[dE=\frac12\int_\Omega w Acdt\cos\theta \frac{d\Omega}{4\pi}=\frac12\frac{w Acdt}{4\pi}\int_\Omega\cos\theta 2\pi \under{=d(\cos\theta)}{\sin\theta d\theta} =\frac{w cAdt}4,\]
dunque $I_{\text{assorbita}}=\dfrac{w c}4$.
\medskip

\noindent
Consideriamo ora l'impulso assorbito
\[d\abs{\vec p_\perp}\pasgnlmath={m_{\mathrm{luce}}=0}\frac{dE}c\cos\theta=\frac12\int_\Omega \frac w c Acdt(\cos\theta)^2 \frac{d\Omega}{4\pi}=\frac{w Adt}6,\]
dunque la pressione \`e
\[p=\frac w6.\]

\bigskip

\noindent
Se la superficie \`e riflettente allora $I_{\text{assorbita}}=0$ per definizione, quindi
\[d\abs{\vec p_\perp}=2\frac{w Adt}6\implies p=\frac w3.\]

\section{Radiazione di Corpo nero}
\subsection{Definizione e corpo nero scatola}
\begin{definition}[Assorbanza]
Un corpo irraggiato assorbe una frazione $a(\nu)$ della radiazione a frequenza $\nu$ che riceve. Questa quantit\`a \`e detta \textbf{assorbanza} (relativa alla frequenza $\nu$).
\end{definition}

\begin{definition}[Corpo bianco e corpo nero]
Se $a(\nu)=0$ per ogni $\nu$ allora il corpo \`e detto \textbf{bianco} o \textbf{perfettamente riflettente}.\\
Se $a(\nu)=1$ per ogni $\nu$ allora il corpo \`e detto \textbf{nero} o \textbf{perfettamente assorbente}.
\end{definition}

\begin{remark}
Poich\'e ogni oggetto contiene qualche particella carica, ogni corpo irraggia delle onde elettromagnetiche.
\end{remark}


\noindent
Una possibile realizzazione di un corpo nero \`e una scatola con pareti opache e un forellino minuscolo come in figura

\begin{figure}[!htb]
    \centering
    \includegraphics[width=2.5cm]{images/corpo_nero_scatola.png}
\end{figure}

\begin{proposition}
La densit\`a di energia di un corpo nero dipende solo da $T$. Inoltre la densit\`a di energia per ogni frequenza dipende solo da $T$.
\end{proposition}
\begin{proof}
Supponiamo di avere due corpi neri scatola, una con pareti interne perfettamente riflettenti e piene di energia e l'altra con pareti interne perfettamente assorbenti e poca energia.

\begin{figure}[!htb]
    \centering
    \includegraphics[width=7cm]{images/dimostrazione_w_dipende_solo_da_T.png}
\end{figure}

\noindent
Mettendo a contatto i forellini come in figura notiamo che le onde passerebbero dalla scatola riflettente a quella assorbente. Questo significa che anche se le sorgenti hanno la stessa temperatura, una tenderebbe a scaldarsi, negando il secondo principio.
\bigskip

\noindent
Se ora immaginiamo di mettere nel corridoio tra le scatole una parete che fa passare solo luce di una certa frequenza negeremmo di nuovo i principi della termodinamica se le densit\`a di energia fossero diverse per frequenze diverse.
\end{proof}

\subsection{Legge di Kirchhoff}
Immaginiamo ora di inserire un secondo corpo nero dentro la scatola. 
Per mantenere la sola dipendenza da $T$ si ha che per il corpo nero interno $I=M$, cio\`e assorbe tanta energia quanta ne emette.\\
Definiamo questa quantit\`a \textbf{radiazione di corpo nero}.

\begin{figure}[!htb]
    \centering
    \includegraphics[width=2.5cm]{images/corpo_nero_con_dentro_corpo_nero.png}
\end{figure}

\noindent
Consideriamo ora lo steso scenario ma con un corpo non nero. Con un argomento analogo $aI=M$ in generale e per ogni frequenza $a(\nu)I_\nu=M_\nu$.

\begin{figure}[!htb]
    \centering
    \includegraphics[width=2.5cm]{images/corpo_nero_con_dentro_corpo_grigio.png}
\end{figure}

\begin{remark}[Legge di Kirchhoff]
Il rapporto tra quanta radiazione un corpo emette rispetto ad un corpo nero alla stessa temperatura \`e esattamente la frazione di energia che assorbe
\[\frac{M}{M_{\text{corpo nero}}}=a.\]
Questa \`e la \textbf{legge di Kirchhoff}.
\end{remark}

\noindent Il ragionamento fatto funziona anche per frequenze fissate, quindi
\[\frac{M_\nu}{a(\nu)}=(M_\nu)_{\text{corpo nero}}.\]
Questo dimostra che la radiazione di corpo nero $(M_\nu)_{\text{corpo nero}}$ \`e una legge universale.

\subsection{Legge di Stefan-Boltzmann}
Osserviamo che
\[M=\int_0^\infty I_\nu d\nu=\frac{E_{\text{totale emessa}}}{\Delta t A}.\]
Sperimentalmente Stefan scopre che
\[M\propto T^4.\]
Boltzmann deduce questo risultato per via teorica:

\begin{theorem}[Legge di Stefan-Boltzmann]\label{LeggeStefanBoltzmann}
Vale la \textbf{legge di Stefan-Boltzmann}
\[M=\sigma T^4.\]
\end{theorem}
\begin{proof}
Modelliamo il corpo nero come una scatola con pareti interne perfettamente riflettenti.\\
Per definizione $w=\frac UV$, inoltre questo rapporto dipende solo da $T$.\\ Abbiamo calcolato prima che per pareti perfettamente riflettenti $p=\frac w3$, che dipende solo da $T$. Ricordando che per sistemi idrostatici $dU=TdS-pdV$, si ha che
\[dS=\frac {dU}T+\frac pTdV=\frac VTdw+\pa{\frac wT+\frac pT}dV.\]
Poich\'e $w$ dipende solo da $T$, se $T$ \`e costante allora $w$ \`e costante, quindi dalla relazione sopra segue che
\[s\doteqdot\ppb VST=\ppb VSw=\pa{\frac wT+\frac pT}=\frac 43 \frac wT,\]
dunque $S=sV$ e $s$ dipende solo da $T$.\\
Sviluppando di nuovo il primo principio si ha
\[wdV+Vdw=dU=TdS-pdV=TdS-\frac w3dV=T(sdV+Vds)-\frac w3dV.\]
Dividendo per $V$ troviamo
\[dw=Tds+\pa{Ts-\frac 43w}\frac{dV}V.\]
Portando $Tds$ al primo membro osserviamo che,
poich\'e $w$ e $s$ dipendono solo da $T$ (e in particolare non da $V$) entrambi i membri sono nulli, 
\[dw=Tds\quad\text{e}\quad Ts=\frac 43w.\]
Questo mostra che
\[\dd sw=T=\frac 43\frac ws\quad\pasgnl\implies{risolvi eq. diff.}\quad w\propto s^{4/3}\propto (w/T)^{4/3},\]
quindi $w\propto T^4$.
Segue che $M=I=\frac{wc}4\propto T^4$. Sia $\sigma$ la costante di proporzionalit\`a.
\end{proof}

\begin{fact}
La costante che appare nella legge di Stefan-Boltzmann, detta \textbf{costante di Stefan-Boltzmann} ha il valore
\[\sigma=5.67\cdot 10^{-8} \ \mathrm{W}\ \mathrm{m}^{-2}\ \mathrm{K}^{-4}.\]
\end{fact}
\chapter{Meccanica Quantistica}

\begin{definition}
Definiamo \textbf{acca tagliato} come
\[\hbar=\frac h{2\pi}.\]
\end{definition}

\section{Richiami di onde}
\begin{definition}[Onda]
Un'\textbf{onda} \`e una funzione di posizione e tempo della forma
\[f(x,y)=f(kx-\omega t).\]
La \textbf{lunghezza d'onda} \`e $\la=\frac{2\pi}k$ e la sua \textbf{frequenza} \`e $\nu=\frac\omega{2\pi}$.\\
La velocit\`a di un'onda o \textbf{velocit\`a di fase} \`e $\frac{\omega}k=\nu\la$.
\end{definition}

\noindent
Se l'onda si comporta anche come una particella, la velocit\`a di questa particella \`e la velocit\`a di fase?
\begin{itemize}
\item Caso non relativistico:
\[\begin{cases}
p=mv\\
E=\frac12mv^2
\end{cases}\implies v_{\text{particella}}=\frac{2E}p=\frac{2h\nu}{h/\la}=2\nu\la\neq \nu\la.\]
\item Caso relativistico:
\[\begin{cases}
p=mv\gamma\\
E=mc^2\gamma
\end{cases}\implies v_{\text{particella}}=\frac{pc^2}E=\frac{hc^2}{h\la\nu}=\frac{c^2}{\la\nu}\]
e questa quantit\`a vale $\la\nu$ solo se $\la\nu=c$, ma non tutte le particelle che vogliamo trattare vanno alla velocit\`a della luce.
\end{itemize}
\noindent La giusta definizione di $v_{\text{particella}}$ \`e

\begin{definition}[Velocit\`a di gruppo]
La \textbf{velocit\`a di gruppo} di un'onda \`e
\[v_{\text{particella}}=\dd k\omega.\]
\end{definition}

\noindent
Se l'onda in esame \`e una semplice sinusoide allora le due velocit\`a coincidono, ma se consideriamo somme di diverse sinusoidi\footnote{per esempio in una serie di Fourier} allora le due quantit\`a sono distinte e $v_{\text{particella}}$ \`e la misura di velocit\`a giusta per l'insieme delle onde.

\begin{remark}
Dato che $E=h\nu$ e $\la=h/p\coimplies p=\frac h{2\pi}k$ si ha che $E=\hbar\omega$ e $\vec p=\hbar \vec k$. In effetti
\[(\nu/c,\vec k)\]
\`e un quadrivettore.
\end{remark}


\section{Equazione di Schr\"odinger}

\begin{theorem}[Equazione di Schr\"odinger]\label{EquazioneSchrodinger}
Per onde vale l'\textbf{equazione di Schr\"odinger}
\[\boxed{i\hbar \pp t\psi=-\frac{\hbar^2}{2m}\pps[2]x\psi +V}\]
dove $V$ \`e un termine di energia potenziale.
\end{theorem}
\begin{proof}[Derivazione intuitiva]
Trascuriamo momentaneamente la relativit\`a.
\[E=\frac12mv^2+V=\frac{\vec p^2}{2m}+V.\]
Poich\'e $E=\hbar \omega$ e $\vec p=\hbar \vec k$ si ha che
\[\hbar \omega=\frac{\hbar^2}{2m}\vec k^2+V.\]
Sia $\psi(x,t)$ l'equazione dell'onda. In particolare consideriamo
\[\psi(x,t)=Ae^{i(kx-\omega t)}.\]
Notiamo che
\[\pp t\psi=-i\omega\psi,\quad \pp x\psi=ik\psi,\quad \pps[2]x\psi=-k^2\psi,\]
dunque mettendo tutto insieme
\[i\hbar \pp t\psi=-\frac{\hbar^2}{2m}\pps[2]x\psi +V.\]
\end{proof}
\begin{center}
``Ma i numeri complessi che ci fanno in fisica?''
\end{center}
L'interpretazione fisica di $\psi$ \`e che $\abs{\psi(x,t)}^2\geq 0$ rappresenta la \textit{densit\`a di probabilit\`a di trovare la particella nel punto $x$ al tempo $t$}.
\bigskip

\noindent
Abbiamo quindi rinunciato a trovare una posizione esatta. Sappiamo solo delle probabilit\`a.




\appendix
\chapter{Richiami matematici}
\section{Derivate parziali e Jacobiane}
Da una relazione $f(x,y,z)=0$ possiamo ricavare $x=x(y,z)$ e $y=y(x,z)$.\\
Possiamo dunque sviluppare i differenziali
\begin{align*}
dx=&\ppb yxzdy+\ppb zxydz\\
dy=&\ppb xyzdx+\ppb zyxdz.
\end{align*}

\begin{proposition}[Propriet\`a delle derivate parziali]\label{ProprietaDerivateParziali}
Valgono le seguenti propriet\`a, dette \textbf{dell'inversa} e \textbf{ciclicit\`a} rispettivamente:
\[\ppb yxz=\pa{\ppb xyz}\ii,\qquad \ppb yxz\ppb zyx\ppb xzy=-1.\]
\end{proposition}
\begin{proof}
Considerando le espressioni date sopra e sostituiendo $dy$ dentro lo sviluppo di $dx$ ricaviamo l'equazione
\[\pa{1-\ppb yxz\ppb xyz}dx=\pa{\ppb yxz\ppb zyx+\ppb zxy}dz.\]
Se fissiamo $z$ il membro di sinistra non cambia, mentre quello di destra risulta nullo ($dz=0$). Poich\'e questo \`e vero anche per $dx\neq 0$ necessariamente ricaviamo
\[1=\ppb yxz\ppb xyz\]
che \`e la propriet\`a dell'inversa.\medskip

\noindent Avendo mostrato questo ricaviamo che il membro di sinistra \`e sempre nullo, anche per $dz\neq 0$, quindi segue l'equazione
\[\ppb yxz\ppb zyx+\ppb zxy=0,\]
la quale corrisponde alla propriet\`a di ciclicit\`a.
\end{proof}

\noindent Consideriamo le seguenti relazioni
\[\begin{cases}
x=x(u,v)\\
y=y(u,v)
\end{cases}.\]
Poniamo
\[\pp{(u,v)}{(x,y)}=det{\mat{\displaystyle\pp ux &\displaystyle\pp vx\\\\ \displaystyle\pp uy & \displaystyle\pp vy}}.\]
\begin{remark}[Jacobiane notevoli]\label{JacobianeNotevoli}
Si ha che
\[\pp{(x,y)}{(x,y)}=1,\quad \pp{(u,v)}{(x,x)}=0,\quad \pp{(u,v)}{(x,y)}=-\pp{(u,v)}{(y,x)}=\pp{(u,v)}{(-x,y)}.\]
Inoltre
\[\pp{(u,y)}{(x,y)}=\ppb uxy,\quad \pp{(u,v)}{(x,u)}=\pp{(r,s)}{(x,u)}\pp{(u,v)}{(r,s)},\quad \pp{(u,v)}{(x,y)}=\pa{\pp{(x,y)}{(u,v)}}\ii.\]
\end{remark}

\section{Differenziali esatti}
Ricordiamo che una forma $\sum A_i dx_i$ \`e chiusa quando per ogni coppia $i,j$
\[\pp {x_i}{A_j}=\pp{x_j}{A_i}.\]
Se il dominio \`e semplicemente connesso allora questa condizione caratterizza anche le forme esatte.

\begin{proposition}[Esattezza tramite Pfaff]\label{EsattezzaPfaff}
Sia $\sum_i A_idx_i$ una forma. Se l'equazione Pfaff
\[\sum_i A_idx_i=0\]
\`e integrabile (cio\`e i punti che la verificano sono descrivibili tramite una equazione $F(x_1,\cdots, x_n)=cost.$) allora la forma \`e chiusa ed esiste $u(x_1,\cdots, x_n)$ tale che $\sum uA_i dx_i$ \`e esatta.
\end{proposition}
\begin{proof}
Sia $\cpa{F=0}$ l'equazione del luogo dove vale l'equazione Pfaff. Segue che
\[dF=\sum_i\pp{x_i}F dx_i=0=\sum_i A_idx_i,\]
dunque possiamo definire $u$ in modo tale che
\[\pp{x_i}F=u(x_1,\cdots, x_n)A_i.\]
Osserviamo inoltre che
\[\pp{x_i}{}(uA_j)=\pp{x_i\del x_j}{^2F}=\pp{x_j}{}(uA_i),\]
cio\`e $\sum_i uA_idx_i$ \`e chiusa.
\end{proof}



\chapter{Chiusura}
A questo link potete trovare un formulario per il corso redatto l'anno scorso (22/23) da Andrea Rocca e Alessio Sgubin:
\begin{center}
\url{https://poisson.phc.dm.unipi.it/~rocca/materiali/Fisica3.pdf}
\end{center}

\bigskip 

\noindent Errata al formulario\footnote{almeno per quanto ho notato}:
\begin{enumerate}
\item Somma relativistica delle velocit\`a.
\end{enumerate}


\newpage
\noindent
Ringrazio i seguenti per aver collaborato nella stesura di alcune parti, per aver offerto consigli su come migliorare le dispense in generale o per aver segnalato degli errori:
\begin{itemize}
\item Alessandro Borghesi
\end{itemize}

\end{document}