\documentclass[a4paper]{report}
\input{../../Templates/preamble.tex}
%PER CAMBIARE I MARGINI
\usepackage[margin=4cm]{geometry}

%----------- Setup stilistico ----------------
\definecolor{DarkRed}{HTML}{B6321C}
\hypersetup{
    colorlinks=true,
    linkcolor=DarkRed,
    filecolor=blue,
    citecolor = black,
    urlcolor=cyan,
}
\renewcommand\thefootnote{\textcolor{blue}{\arabic{footnote}}}
% ============================================


%---------- Comandi specifici ----------------


%--------- Comandi dattilografici ------------
\NewDocumentCommand{\ddb}{O{}mmm}{
    {\left.\frac{d^{#1}{#3}}{d{#2}^{#1}}\right|_{#4}}
}
\NewDocumentCommand{\ppb}{O{}mmm}{
    {\left.\frac{\partial^{#1}{#3}}{\partial{#2}^{#1}}\right|_{#4}}
}

% ============================================
\title{Fisica 3\\
\large Corso del prof. Sozzi Marco}

\author{Francesco Sorce}
\date{Università di Pisa\\
Dipartimento di Matematica\\
A.A. 2023/24}

\begin{document}
\maketitle

%\newpage
\tableofcontents
\newpage


\part{Termodinamica}
\chapter{Equilibrio e Calore}
\noindent
La termodinamica \`e lo studio di sistemi dal punto di vista macroscopico.\\
Le massime fondamentali della termodinamica sono
\begin{itemize}
\item L'energia dell'universo \`e costante
\item L'entropia dell'universo tende ad aumentare.
\end{itemize}

\section{Prime definizioni}
\begin{definition}[Sistema termodinamico]
Un \textbf{sistema termodinamico} \`e un sistema omogeneo composto da ``molti" elementi.\\
Lo \textbf{stato} di un sistema termodinamico \`e univocamente determinato da un numero contenuto di parametri\footnote{Per esempio temperatura, pressione o volume.} detti \textbf{funzioni di stato}.\\
Il numero di funzioni di stato necessarie per specificare lo stato \`e detto \textbf{numero di gradi di libert\`a}.
\end{definition}

\begin{remark}
Le funzioni di stato di un sistema non dipendono da come esso \`e venuto ad esistere; se due procedimenti portano da un particolare stato ad un altro, le differenze nelle funzioni di stato dipendono univocamente dallo stato iniziale e quello finale.
\end{remark}

\begin{remark}[Sistema ambiente]
Spesso torna comodo considerare una coppia di sistemi, uno detto semplicemente sistema e l'altro \textbf{ambiente}.
\end{remark}

\begin{definition}[Variabili estensive e intensive]
Dato un sistema termodinamico, delle variabili ad esso inerenti si dicono \textbf{estensive} se sono proporzionali alla quantit\`a di materia contenuta nel sistema e \textbf{intensive} altrimenti.
\end{definition}

\begin{example}
Il volume e l'energia sono grandezze estensive mentre la pressione e la temperatura sono intensive.
\end{example}

\begin{remark}
Il lavoro meccanico \`e dato da $W=\int \vec F\cdot \vec{d\ell}$. \`E un fatto generale che il lavoro ha la forma
\[\int(\text{intensiva})d(\text{estensiva}).\]
\end{remark}

\begin{definition}[Sistemi isolati, chiusi e aperti]
Un sistema termodinamico si dice 
\begin{itemize}
\item \textbf{isolato} se non ammette scambio con l'ambiente,
\item \textbf{chiuso} se non ammette scambio di materia con l'ambiente,
\item \textbf{aperto} se ammette scambi con l'ambiente.
\end{itemize}
\end{definition}

Per considerare pi\`u sistemi termodinamici dobbiamo considerarli come separati da una \textit{parete}.

\begin{definition}[Tipi di parete]
Una parete tra due sistemi \`e
\begin{itemize}
\item \textbf{adiabatica} se non permette scambi,
\item \textbf{diatermica} se non ammette scambi di materia,
\item \textbf{semipermeabile} se fa passare alcuni tipi di materia.
\item \textbf{permeabile}\footnote{una parete permeabile \`e come se non ci fosse} se permette ogni tipo di scambio.
\end{itemize}
\end{definition}

\section{Principio 0, Equilibrio e Temperatura empirica}
\subsection{Tipi di equilibrio}
\begin{definition}[Equilibrio]
Un sistema \`e in \textbf{equilibrio} se le sue funzioni di stato restano ``costanti" (per molto tempo rispetto alla scala temporale rilevante).\\
Un sistema \`e in \textbf{equilibrio termico} se non ci sono differenze di temperatura\footnote{definiremo la temperatura in seguito.}.\\
Un sistema \`e in \textbf{equilibrio termodinamico} se \`e in equilibrio meccanico, termico e chimico.
\end{definition}

\begin{remark}
I sistemi tendono spontaneamente ed irreversibilmente all'equilibrio termodinamico.
\end{remark}

\begin{fact}[0-esimo principio della termodinamica]
\textbf{Due sistemi in equilibrio termico con un terzo sono in equilibrio tra loro.}
\end{fact}

\begin{definition}[Equazione di stato]
Se quando un sistema \`e in equilibrio vale una equazione tra le funzioni di stato, queste si dicono \textbf{equazioni di stato}.
\end{definition}

\begin{remark}[Segno degli scambi di energia]
\underline{\textbf{NOTA BENE:}} Affermiamo per convenzione che uno scambio di energia ha segno \textit{positivo} se il sistema \textbf{\textit{acquista energia dall'ambiente}}.
\end{remark}



\subsection{Processi quasistatici}

\begin{definition}[Processi quasistatici]
Un sistema \`e \textbf{quasi in equilibrio} se \`e cos\`i vicino all'equilibrio che le equazioni di stato si possono considerare valide.\\ 
Un \textbf{processo quasistatico} \`e un processo tale per cui il sistema \`e quasi in equilibrio in ogni istante.\\
Se non sono presenti ``attriti", un processo quasistatico \`e detto \textbf{reversibile}.\\
Un processo \`e detto \textbf{totalmente reversibile} se \`e reversibile e la sua interazione con l'ambiente \`e reversibile.
\end{definition}

\begin{remark}
Un processo quasi statico va pensato come un processo molto lento; cos\`i lento da poter pensare al sistema come ``sempre in equilibrio".
\end{remark}



\subsection{Temperatura empirica}

\begin{proposition}[Temperatura empirica]\label{TemperaturaEmpirica}
Ogni sistema termodinamico ammette una funzione che \`e costante in stato di equilibrio. La costante \`e detta \textbf{temperatura empirica}.
\end{proposition}
\begin{proof}
Consideriamo tre sistemi, con funzioni di stato $(x_1, y_1),\ (x_2,y_2)$ e $(x_3,y_3)$ in equilibrio tra loro. Esistono dunque equazioni di stato della forma
\[\begin{cases}
x_3=f(x_1,y_1,y_3)\\
x_3=g(x_2,y_2,y_3)
\end{cases}\]
poich\'e i sistemi 1 e 2 sono in equilibrio, se eguagliamo le due equazioni sappiamo che ci\`o che otteniamo non dipende da $y_3$, quindi
\[\begin{cases}
f(x_1,y_1,y_3)=\phi_1(x_1,y_1)\zeta(y_3)+\eta(y_3)\\
g(x_2,y_2,y_3)=\phi_2(x_2,y_2)\zeta(y_3)+\eta(y_3)
\end{cases}\]
dunque se 1 e 2 sono in equilibrio si ha che 
\[\phi_1(x_1,y_1)=\phi_2(x_2,y_2),\]
ma i due membri dipendono da insiemi di variabili disgiunti, quindi esiste $\theta_0$ tale che entrambe queste espressioni eguagliano $\theta_0$ se sono in equilibrio. Il valore $\theta_0$ \`e detto la temperatura empirica dei sistemi, i quali sono in equilibrio solo se hanno la stessa temperatura empirica.
\end{proof}

\begin{definition}[Isoterme]
Dato un sistema termodinamico e un valore $\theta_0$ di temperatura empirica, chiamiamo \textbf{isoterma a livello $\theta_0$} l'insieme degli stati del sistema la cui temperatura \`e $\theta_0$.
\end{definition}


\subsection{Definizione di temperatura tramite gas}
\begin{fact}[Punto triplo]
\emph{Considerando come sistema termodinamico dell'acqua esiste una precisa combinazione di temperatura e pressione tale per cui essa risulta in trasizione tra gli stati solido liquido e gassoso simultaneamente.\\
Questo stato si chiama \textbf{punto triplo} e i valori in questione sono una temperatura di $0.01 ^\circ \mathrm{C}$ e una pressione di $0.006\ \mathrm{atm}$.}
\end{fact}
\medskip

\noindent
A bassa pressione i gas si comportano tutti allo stesso modo\footnote{rispettano l'equazione di stato $pV=f(\theta)$}.\\
Se fissiamo il volume e la quantit\`a di materia del gas possiamo definire $\theta$ in modo tale che $p=p_0(1+\al\theta)$, cio\`e poniamo 
\[\theta=\frac1\al\frac{p-p_0}{p_0}.\] 
Se imponiamo che l'acqua congeli per $\theta=0$ e evapori per $\theta=100$ allora ricaviamo $1/\al=273.15$. Notiamo inoltre\footnote{l'addizione di $\al\ii$ corrisponde alla traslazione che trasforma gradi Celsius in gradi Kelvin.}
\[\frac{p_2}{p_1}=\frac{\al\ii+\theta_2}{\al\ii +\theta_1}=\frac{\theta_2'}{\theta_1'}.\]
Possiamo dunque definire la temperatura (in Kelvin) come
\[T=\lim_{p^{(PT)}\to 0}273.16 \frac{p}{p^{(PT)}}\]
dove $p^{(PT)}$ \`e la pressione del gas nel termometro quando questo sistema \`e in equilibrio con il sistema di punto triplo con l'acqua. Il limite corrisponde a prendere gas sempre pi\`u rarefatti, cio\`e a lavorare nel limite dei gas perfetti dove vale la proporzionalit\`a sopra.
\medskip

\noindent Sfruttando questa definizione possiamo costruire un termometro a gas come in figura

[FIGURA TERMOMETRO A GAS]

\noindent Quando il gas \`e alla temperatura che vogliamo misurare, misuriamo la differenza di altezza tra il livello a contatto con il gas e il livello di controllo posto a pressione atmosferica. 
Questa differenza \`e proporzionale alla differenza di pressione e questo ci permette di ricavare la temperatura se la fissiamo per quando \`e nel punto critico.


\section{Primo principio e Definizione di calore}

\begin{fact}[Primo principio della termodinamica]
\textbf{L'energia interna di un sistema di conserva.}
\end{fact}

\begin{definition}[Calore]
Il \textbf{calore} \`e la differenza tra la variazione di energia interna e il lavoro compiuto su un sistema termodinamico, esplicitamente
\[\boxed{\Delta U=Q+W}\]
\end{definition}

\begin{remark}
Il calore e il lavoro non sono funzioni di stato, ma la loro somma s\`i.
\end{remark}

\begin{remark}[Primo principio in forma differenziale]
Scrivendo il primo principio in termini di infinitesimi troviamo
\[dU=\delta Q+\delta W,\]
in particolare per i gas ideali vale
\[dU=\delta Q-pdV.\]
\end{remark}


\begin{definition}[Caloria]
Una \textbf{caloria} \`e la quantit\`a di calore necessaria per far variare la temperatura di un grammo di acqua da $14.5^\circ\mathrm{C}$ a $15.5^\circ\mathrm{C}$.\\
In Joule si ha che
\[\boxed{1\ \mathrm{cal}=4.186\ \mathrm{J}}\]
\end{definition}


\begin{remark}
In una trasformazione adiabatica, il lavoro \`e dato dalla differenza di energia interna.
\end{remark}
\begin{example}[Coppia di sistemi dentro un contenitore adiabatico]
Consideriamo due sistemi $A$ e $B$ dentro un contenitore adiabatico. Per il primo principio
\[0=\Delta U=\Delta U_A+\Delta U_B=Q_A+Q_B+\under{=W}{W_A+W_B}.\]
I trasferimenti di calore possono avvenire solo tra $A$ e $B$, quindi $Q_A+Q_B=0$ e $W=0$. Quanto scritto \`e una ``legge di conservazione del calore" in questo tipo di sistema.
\end{example}
\chapter{Trasferimento di calore}

\section{Modalit\`a di trasferimento di calore}
Il trasperimento di calore, cio\`e di energia derivante da una differenza di temperatura, avviene in tre modi: conduzione, covezione ed irraggiamento.

\subsection{Conduzione}
Parliamo di \textbf{conduzione} quando il tresferimento di calore avviene per contatto ma senza scambio di materia (attraverso una parete diatermica).\medskip

\noindent Empiricamente riscontriamo
\begin{fact}[Legge di Fourier]
Vale la relazione
\[\frac1A\frac{\delta Q}{\Delta t}=-\kappa\frac{\Delta T}{\Delta X},\]
dove $T$ \`e la temperatura, $X$ \`e la distanza tra i punti tra cui stiamo calcolando la differenza di temperatura, $A$ \`e l'area ortogonale alla direzione lungo la quale si propaga il calore e $\kappa$ \`e una costante detta \textbf{conducibilit\`a termica}.
\end{fact}

\noindent L'unit\`a di misura della conducibilit\`a termica \`e
\[[\kappa]=\frac W{mK}\approx \begin{cases}
10^2 &\text{metalli}\\
0.1 &\text{gas}
\end{cases}.\]
\noindent Possiamo precisare la legge di Fourier introducendo la
\textbf{corrente di calore} $\vec J_Q$. La legge assume la forma
\[\vec J_Q=-k\vec \nabla T.\]
Concentrandosi su uno dei sistemi possiamo scrivere
\[\boxed{\delta Q= cm\delta T}\]
dove $m$ \`e la massa e $c$ \`e il \textbf{calore specifico}.\bigskip

\noindent Possiamo calcolare il calore totale che entra dentro una superficie per unit\`a di tempo come
\[\int_V c\pp tT\rho dV=\frac1{\Delta t}\int_{\del V} \delta Q=-\int_{\del V} \vec J_Q\cdot \vec d\Sigma=-\int_V \nabla \cdot \vec J_Q dV=\int_Vk\nabla^2 T dV.\]
Ricaviamo dunque
\[\boxed{\pp tT=\frac{\kappa}{\rho c}\nabla^2T}\]
Questa \`e la famosa \textit{equazione del calore}.

\subsection{Convezione}
Parliamo di \textbf{convezione} quando il trasferimento di calore avviene tramite lo spostamento di materia.\\ La formula rilevante in questo caso \`e
\[\frac1A\frac{\delta Q}{\Delta t}=h\Delta T,\]
dove $h$ \`e il \textbf{coefficiente convettivo}.

\subsection{Irraggiamento}
Parliamo di \textbf{irraggiamento} quando un corpo semplicemente emette energia come radiazione.\\ 
La formula rilevante in questo caso \`e
\[\frac1A\frac{\delta Q}{\Delta t}=\e \sigma(T^4-T_0^4),\]
dove $T_0$ \`e la temperatura dell'ambiente, $\sigma$ \`e una costante uguale per tutti i materiali e $\e$ dipende dai materiali.

\section{Primo principio della termodinamica}

\begin{fact}[Primo principio della termodinamica]
\textbf{L'energia interna di un sistema di conserva. Esplicitamente}
\[\boxed{\Delta U=Q+W}\]
\end{fact}

\begin{remark}
Il calore e il lavoro non sono funzioni di stato, ma la loro somma s\`i.
\end{remark}

\begin{remark}[Primo principio in forma differenziale]
Scrivendo il primo principio in termini di infinitesimi restituisce
\[dU=\delta Q+\delta W,\]
in particolare per i gas ideali troviamo
\[dU=\delta Q-pdV.\]
\end{remark}



\begin{definition}[Caloria]
Una \textbf{caloria} \`e la quantit\`a di calore necessaria per far variare la temperatura di un grammo di acqua da $14.5^\circ\mathrm{C}$ a $15.5^\circ\mathrm{C}$.\\
In Joule si ha che
\[\boxed{1\ \mathrm{cal}=4.186\ \mathrm{J}}\]
\end{definition}


\begin{remark}
In una trasformazione adiabatica, il lavoro \`e dato dalla differenza di energia interna.
\end{remark}
\begin{example}[Coppia di sistemi dentro un contenitore adiabatico]
Consideriamo due sistemi $A$ e $B$ dentro un contenitore adiabatico. Per il primo principio
\[0=\Delta U=\Delta U_A+\Delta U_B=Q_A+Q_B+\under{=W}{W_A+W_B}.\]
I trasferimenti di calore possono avvenire solo tra $A$ e $B$, quindi $Q_A+Q_B=0$ e $W=0$. Quanto scritto \`e una ``legge di conservazione del calore" in questo tipo di sistema.
\end{example}



\section{Capacit\`a termica}
\begin{definition}[Capacit\`a termica]
Definiamo la \textbf{capacit\`a termica} come\footnote{Nota che NON \`e una derivata in quanto $Q$ non \`e una funzione di stato, quindi in particolare non \`e una funzione di $T$}
\[C=\lim_{\delta T\to 0}\frac{\delta Q}{\delta T}.\]
L'unit\`a di misura \`e $[C]=\mathrm{J}/\mathrm{K}$.\\
La \textbf{capacit\`a termica molare} \`e data da $c=C/n$.\\
Il \textbf{calore specifico} \`e dato da $C/m$.
\end{definition}

\begin{definition}[Termometro e Termostato]
Un \textbf{termostato} \`e un oggetto ideale con capacit\`a termica infinita\footnote{intuitivamente \`e un sistema grande a sufficienza in modo che anche se viene aggiunto calore, la temperatura non cambia.}.\\ 
Un \textbf{termometro} \`e un oggetto ideale con capacit\`a termica nulla\footnote{intuitivamente \`e un sistema piccolo a sufficienza in modo da poter trascurare gli scambi di calore.}.
\end{definition}

\begin{remark}
Possiamo scrivere la capacit\`a termica in termini di $U,\ V,\ p$ e $T$ come segue:
\[C=\frac{\delta Q}{\delta T}=\ppb TUV+\pa{\ppb VUT+p}\dd TV\]
\end{remark}
\begin{proof}
Sviluppando $dU$ troviamo
\[dU=\ppb TUVdT+\ppb VUTdV,\]
da cui
\[\delta Q=dU+pdV=\ppb TUVdT+\pa{\ppb VUT+p}dV.\]
Ora possiamo ``dividere" per $dT$ e trovare la tesi.
\end{proof}

\begin{definition}[Capacit\`a termica a volume/pressione costante]
Definiamo la \textbf{capacit\`a termica a volume} (risp. \textbf{pressione}) \textbf{costante} come le due seguenti quantit\`a
\begin{align*}
C_V=&\rbar{\frac{\delta Q}{\delta T}}_V=\ppb TUV\\
C_p=&\rbar{\frac{\delta Q}{\delta T}}_p=\ppb TUV +\pa{\ppb VUT+p}\ppb TVp=\ppb TUV +\pa{\ppb VUT+p}V\al
\end{align*}
\end{definition}

\begin{remark}[Disuguaglianza tra capacit\`a termiche]\label{DisguguaglianzaCapacitaTermiche}
Vale sempre $C_p>C_V$.
\end{remark}

\begin{remark}\label{DerivataEnergiaInternaRispettoAlVolume}
In un gas generale
\[\boxed{\ppb VUT=\frac{C_p-C_V}{V\al}-p}\]
\end{remark}

\subsection{Capacit\`a termica per gas ideali}

\begin{definition}[Coefficiente di Joule]
Definiamo il \textbf{coefficiente di Joule} come
\[\mu_J=\ppb VTU\]
\end{definition}

\begin{fact}[In gas ideale l'energia interna dipende solo dalla temperatura]\label{InGasIdealeUdipendeSoloDaT}
In un gas ideale $U$ dipende solo da $T$.
\end{fact}
\begin{proof}[Esperimento: Espansione libera adiabatica di Joule]
Consideriamo un contenitore adiabatico separato internamente da una parete adiabaita. In uno dei due volumi si trova un gas ideale, il secondo \`e vuoto.

[DISEGNO]

\noindent
Improvvisamente eliminiamo la parete interna e lasciamo che il gas si espanda\footnote{notiamo che questo NON \`e una processo quasistatico.}.\smallskip

\noindent
Chiaramente $Q=W=0$ in quanto il vuoto non subisce/effettua lavoro e non scambia calore, dunque $\Delta U=0$.\\
Segue che $\mu_J=\ppb VTU=\dd VT$ e Joule ha misurato che in queste circostanze la seconda \`e nulla, dunque
\[0=\ppb VTU\pasgnl={(\ref{ProprietaDerivateParziali})}-\pa{\ppb UVT}\ii\pa{\ppb TUV}\ii=-\ppb VUT\frac1{C_V},\]
in particolare $\displaystyle\ppb VUT=0$.\medskip

\noindent Poich\'e in un gas ideale $p$ \`e determinata da $V$ e $T$, $U=U(V,T)$. Per quanto appena detto $U$ non dipende da $V$, quindi dipende solo da $T$.
\end{proof}


\begin{corollary}
In un gas ideale
\[dU=nc_V dT.\]
\end{corollary}
\begin{proof}
Ricordiamo che
\[C_V=\ppb TUV,\]
ma poich\'e $U$ non dipende da $V$ possiamo scrivere
\[C_V=\dd TU,\]
che \`e la tesi.
\end{proof}

\begin{proposition}[Relazione di Mayer]\label{RelazioneMayer}
Per gas ideali si ha che $c_p-c_V=R$, o equivalentemente $C_p-C_V=nR$.
\end{proposition}
\begin{proof}
Ricordiamo (\ref{CoefficienteEspansioneECompressibilitaIsotermaPerGasIdeali}) che per gas ideali $\al=T\ii$. Poich\'e $U$ dipende solo da $T$ si ha che
\[0=\ppb VUT\pasgnl={(\ref{DerivataEnergiaInternaRispettoAlVolume})}\frac{C_p-C_V}{V\al}-p,\]
da cui
\[C_p-C_V=pV\al=\frac{nRT}T=nR.\]
\end{proof}

\begin{notation}
Denotiamo il rapporto $\frac{C_V}{C_p}=\frac{c_V}{c_p}$ con $\gamma$.
\end{notation}



\begin{fact}[Calore specifico a volume costante in funzione dei gradi di libert\`a]
In un gas ideale
\[C_V=\frac\nu2nR\]
dove $\nu$ \`e il \textbf{numero di gradi di libert\`a}.
\end{fact}
\begin{remark}
Per un gas ideale monoatomico $\nu=3$, mentre per un gas biatomico $\nu=5$.\\
Segue che
\[c_V^{mono}=\frac32R\approx 12.47\frac{\mathrm{J}}{\mathrm{K\ mol}},\qquad c_V^{bi}=\frac52R\approx 20.74\frac{\mathrm{J}}{\mathrm{K\ mol}}.\]
Da queste scritture segue anche che
\[c_p^{mono}=\frac52R,\quad \gamma^{mono}=\frac53,\quad\qquad c_p^{bi}=\frac72R,\quad \gamma^{bi}=\frac75.\]
\end{remark}

\begin{remark}[L'aria \`e un gas ideale biatomico]
L'aria \`e composta principalmente da particelle biatomiche ($O_2$ e $N_2$).
\end{remark}

\begin{proposition}[Calore infinitesimale con capacit\`a]\label{CaloreInfinitesimaleConCapacita}
Per gas ideali valgono le seguenti equazioni
\begin{enumerate}
\item $\delta Q=C_VdT+pdV$
\item $\delta Q=C_p dT-Vdp$.
\end{enumerate}
\end{proposition}
\begin{proof}
Mostriamo i due punti:
\setlength{\leftmargini}{0cm}
\begin{itemize}
\item[$\boxed{1}$] Ricordiamo la relazione \[\delta Q=\under{=C_V}{\ppb TUV} dT+\pa{{\ppb VUT}+p}dV,\]
da cui, usando il fatto che $\ppb UVT=0$, troviamo che $\delta Q=C_V dT+pdV$.
\item[$\boxed{2}$] Osserviamo che il differenziale di $pV=nRT$ \`e
\[nRdT=pdV+Vdp,\]
da cui sfruttando la relazione precedente
\[\delta Q=C_VdT+pdV=({C_V+nR})dT-Vdp\pasgnl={(\ref{RelazioneMayer})}C_p dT-Vdp.\]
\end{itemize}
\setlength{\leftmargini}{0.5cm}
\end{proof}
\begin{remark}
Osservando la prima equazione ricaviamo nuovamente che $\delta Q$ non \`e un differenziale, infatti se lo fosse avremmo il seguente assurdo:
\[0=\ppb V{C_V}T=\ppb TpV=\frac{nR}V\neq 0.\]
\end{remark}




\section{Energia in funzione delle trasformazioni per gas ideali}

\noindent In questa sezione calcoliamo lavoro, calore e variazione di energia interna per i tipi principali di processi quasistatici.\medskip

\noindent Notiamo che $\Delta U=nc_V\Delta T$ in ogni circostanza in quanto $U$ non dipende da $V$.
\subsection{Isobara}
\begin{proposition}[Energie per isobara]\label{EnergieIsobara}
Per una trasformazione isobara valgono le seguenti identit\`a:
\[W=-nR\Delta T,\quad
Q=nc_p\Delta T,\quad
\Delta U=nc_V\Delta T.\]
\end{proposition}
\begin{proof}
Calcoliamo:
\begin{align*}
W=&-\int_{V_i}^{V_f}pdV\pasgnl={isobara}-p\Delta V\pasgnl={gas ideale}-nR\Delta T\\
Q\pasgnl={isobara}&\int_{T_i}^{T_f}nc_pdT=nc_p\Delta T\\
\Delta U=&Q+W=n(c_p-R)\Delta T=nc_V\Delta T.
\end{align*}
\end{proof}

\subsection{Isocora}
\begin{proposition}[Energie per isocora]\label{EnergieIsocora}
Per una trasformazione isocora valgono le seguenti identit\`a:
\[W=0,\quad
Q=nc_v\Delta T,\quad
\Delta U=nc_V\Delta T.\]
\end{proposition}
\begin{proof}
Calcoliamo:
\begin{align*}
W=&-\int_{V_i}^{V_f}pdV\pasgnlmath={V_i=V_f}0\\
Q\pasgnl={isocora}&\int_{T_i}^{T_f}nc_VdT=nc_V\Delta T\\
\Delta U=&Q+W=nc_V\Delta T.
\end{align*}
\end{proof}

\subsection{Isoterma}
\begin{proposition}[Energie per isoterma]\label{EnergieIsoterma}
Per una trasformazione isoterma valgono le seguenti identit\`a:
\[W=-nRT\log\pa{\frac{V_f}{V_i}},\quad
Q=nRT\log\pa{\frac{V_f}{V_i}},\quad
\Delta U=0.\]
\end{proposition}
\begin{proof}
Poich\'e stiamo considerando un gas ideale \[\Delta U=nc_V\Delta T\pasgnl={isoterma}0.\] 
Per il primo principio si ha $Q=-W$, quindi per concludere basta calcolare il lavoro.
\[W=-\int_{V_i}^{V_f}pdV\pasgnl={gas ideale}-nRT\int_{V_i}^{V_f}\frac1VdV=-nRT\log\pa{\frac{V_f}{V_i}}.\]
\end{proof}

\subsection{Adiabatica}
\begin{proposition}[Equazione di stato per adiabatica]\label{EquazioneStatoAdiabatica}
Poniamo $\gamma=c_p/c_V$. Si ha che $pV^\gamma$ \`e costante seguendo un processo adiabatico.
\end{proposition}
\begin{proof}
Poich\'e il sistema in esame \`e un gas ideale valgono le seguenti uguaglianze
\[0\pasgnl={adiabatica}\delta Q=dU-\delta W\pasgnl={gas ideale}nc_VdT+pdV=\frac{nc_V}{nR}d(pV)+p dV.\]
Segue che
\[-\frac{c_VV}{\cancel{R}}dp=\pa{\frac{pc_V+pR}{\cancel{R}}}dV\pasgnl={(\ref{RelazioneMayer})}\frac{pc_p}{\cancel{R}}dV,\]
da cui
\[-\frac{dp}p=\gamma\frac{dV}V.\]
Integrando troviamo
\[-\log p+Const.=\gamma\log V\coimplies \log pV^\gamma=Const.\coimplies pV^\gamma=e^{Const.}\]
che \`e quello che volevamo mostrare.
\end{proof}
\begin{remark}
Si ha che \[c_v=\frac R{\gamma-1}.\]
\end{remark}
\begin{proof}
Per definizione di $\gamma$
\[c_v=\frac{c_p}\gamma\pasgnl={(\ref{RelazioneMayer})}\frac{R+c_v}\gamma,\]
dunque
\[\gamma c_v=c_v+R\]
e la tesi segue.
\end{proof}

\begin{proposition}[Energie per adiabatica]\label{EnergieAdiabatica}
Per una trasformazione adiabatica valgono le seguenti identit\`a:
\[W=\frac{p_fV_f-p_iV_i}{\gamma-1},\quad
Q=0,\quad
\Delta U=\frac{p_fV_f-p_iV_i}{\gamma-1}.\]
\end{proposition}
\begin{proof}
Poich\'e il processo \`e adiabatico, $Q=0$. Segue per il primo principio che $\Delta Q=W$. Dato che stiamo considerando un gas ideale
\[\Delta U=nc_V\Delta T=n\frac R{\gamma-1}\Delta T=n\frac R{\gamma-1}\Delta (pV)=\frac{p_fV_f-p_iV_i}{\gamma-1}.\]
\end{proof}

\begin{remark}
Potevamo ricavare energia e lavoro anche sfruttando la relazione \[pV^\gamma=p_iV_i^\gamma=p_fV_f^\gamma,\] ma avendola ricavata come sopra sappiamo che l'espressione \`e \textbf{valida anche per processi adiabatici NON quasistatici}.
\end{remark}


\section{Processi politropici}
\begin{definition}[Processo politropico]
Un processo \`e \textbf{politropico} se la capacit\`a termica \`e costante.
\end{definition}

\begin{proposition}[Curve per processi politropici]\label{CurveProcessiPolitropici}
Considerando un processo politropicorelativo ad un gas ideale e definiamo
\[\delta=\frac{C_p-C}{C_V-C},\]
allora seguendo questo processo si ha che $pV^\delta=cost.$.
\end{proposition}
\begin{proof}
Poich\'e $\delta Q=CdT=C_VdT+pdV=C_pdT-Vdp$ ricaviamo che
\[-\frac V{C-C_p}dp=dT=\frac p{C-C_V}dV,\]
da cui
\[-\frac Vp\dd Vp=\frac{C_p-C}{C_V-C}=\delta.\]
Questa espressione restituisce una equazione differenziale
\[-\frac{dp}p=\delta\frac{dV}V,\]
la cui soluzioni hanno la forma voluta.
\end{proof}

\noindent
Possiamo interpretare processi isocori, isobari, isotermi e adiabatici come processi politropici:
\begin{center}
\begin{tabular}[ht]{|c||c|c|c|c|}
\hline
Processo & Isocoro & Isobaro & Isotermo & Adiabatico\\\hline&&&&\\
$\delta$ & $\infty$ & $0$ & $1$ & $\displaystyle\gamma=\frac{c_p}{c_V}$\\ &&&&\\\hline
\end{tabular}
\end{center}



\section{Processo ciclico}
\begin{definition}[Processo ciclico]
Un processo \`e \textbf{ciclico} se lo stato iniziale e finale sono lo stesso. Se qualcosa realizza un processo ciclico \`e detto \textbf{motore}.
\end{definition}

\begin{remark}
Per un processo ciclico, $\Delta U=0$, dunque $Q=-W$.
\[-W=Q=Q_H+Q_L\]
dove $Q_H$ \`e il calore che il sistema acquista da una sorgente calda e $Q_L$ \`e il calore che acquista da una sorgente fredda\footnote{$Q_L$ \`e negativo}. Notiamo che
\[\abs W=\abs{Q_H}-\abs{Q_L}.\]
\end{remark}

\begin{definition}[Efficienza]
L'\textbf{efficienza} di un processo ciclico \`e data da\footnote{Intuitivamente l'efficienza \`e una misura di quanto lavoro riesco a realizzare in proporzione a quanto calore abbiamo dovuto inserire nel sistema. L'altra forma ci dice che l'efficienza \`e data dalla perfezione eccetto per il calore che viene disperso senza diventare lavoro ($Q_L$).}
\[\eta=\frac{\abs W}{\abs{Q_H}}=1-\frac{\abs{Q_L}}{\abs{Q_H}}.\]
\end{definition}
\include{03EsempiProcessiQuasistatici}
\include{04GasIdeali}



%\part{Relativit\`a speciale}

%\part{Basi di meccanica quantistica}


\appendix
\chapter{Richiami matematici}
\section{Derivate parziali e Jacobiane}
Da una relazione $f(x,y,z)=0$ possiamo ricavare $x=x(y,z)$ e $y=y(x,z)$.\\
Possiamo dunque sviluppare i differenziali
\begin{align*}
dx=&\ppb yxzdy+\ppb zxydz\\
dy=&\ppb xyzdx+\ppb zyxdz.
\end{align*}

\begin{proposition}[Propriet\`a delle derivate parziali]\label{ProprietaDerivateParziali}
Valgono le seguenti propriet\`a, dette \textbf{dell'inversa} e \textbf{ciclicit\`a} rispettivamente:
\[\ppb yxz=\pa{\ppb xyz}\ii,\qquad \ppb yxz\ppb zyx\ppb xzy=-1.\]
\end{proposition}
\begin{proof}
Considerando le espressioni date sopra e sostituiendo $dy$ dentro lo sviluppo di $dx$ ricaviamo l'equazione
\[\pa{1-\ppb yxz\ppb xyz}dx=\pa{\ppb yxz\ppb zyx+\ppb zxy}dz.\]
Se fissiamo $z$ il membro di sinistra non cambia, mentre quello di destra risulta nullo ($dz=0$). Poich\'e questo \`e vero anche per $dx\neq 0$ necessariamente ricaviamo
\[1=\ppb yxz\ppb xyz\]
che \`e la propriet\`a dell'inversa.\medskip

\noindent Avendo mostrato questo ricaviamo che il membro di sinistra \`e sempre nullo, anche per $dz\neq 0$, quindi segue l'equazione
\[\ppb yxz\ppb zyx+\ppb zxy=0,\]
la quale corrisponde alla propriet\`a di ciclicit\`a.
\end{proof}

\noindent Consideriamo le seguenti relazioni
\[\begin{cases}
x=x(u,v)\\
y=y(u,v)
\end{cases}.\]
Poniamo
\[\pp{(u,v)}{(x,y)}=det{\mat{\displaystyle\pp ux &\displaystyle\pp vx\\\\ \displaystyle\pp uy & \displaystyle\pp vy}}.\]
\begin{remark}[Jacobiane notevoli]\label{JacobianeNotevoli}
Si ha che
\[\pp{(x,y)}{(x,y)}=1,\quad \pp{(u,v)}{(x,x)}=0,\quad \pp{(u,v)}{(x,y)}=-\pp{(u,v)}{(y,x)}=\pp{(u,v)}{(-x,y)}.\]
Inoltre
\[\pp{(u,y)}{(x,y)}=\ppb uxy,\quad \pp{(u,v)}{(x,u)}=\pp{(r,s)}{(x,u)}\pp{(u,v)}{(r,s)},\quad \pp{(u,v)}{(x,y)}=\pa{\pp{(x,y)}{(u,v)}}\ii.\]
\end{remark}

%\part*{Formulario}
%\include{Riconoscimenti.tex}

\end{document}