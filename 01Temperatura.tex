\chapter{Introduzione e Temperatura empirica}
\noindent
La termodinamica \`e lo studio di sistemi dal punto di vista macroscopico.\\
Le massime fondamentali della termodinamica sono
\begin{itemize}
\item L'energia dell'universo \`e costante
\item L'entropia dell'universo tende ad aumentare.
\end{itemize}

\section{Prime definizioni}
\begin{definition}[Sistema termodinamico]
Un \textbf{sistema termodinamico} \`e un sistema omogeneo composto da ``molti" elementi.\\
Lo \textbf{stato} di un sistema termodinamico \`e univocamente determinato da un numero contenuto di parametri\footnote{Per esempio temperatura, pressione o volume.} detti \textbf{funzioni di stato}.\\
Il numero di funzioni di stato necessarie per specificare lo stato \`e detto \textbf{numero di gradi di libert\`a}.
\end{definition}

\begin{remark}
Le funzioni di stato di un sistema non dipendono da come esso \`e venuto ad esistere; se due procedimenti portano da un particolare stato ad un altro, le differenze nelle funzioni di stato dipendono univocamente dallo stato iniziale e quello finale.
\end{remark}

\begin{remark}[Sistema ambiente]
Spesso torna comodo considerare una coppia di sistemi, uno detto semplicemente sistema e l'altro \textbf{ambiente}.
\end{remark}

\begin{definition}[Variabili estensive e intensive]
Dato un sistema termodinamico, delle variabili ad esso inerenti si dicono \textbf{estensive} se sono proporzionali alla quantit\`a di materia contenuta nel sistema e \textbf{intensive} altrimenti.
\end{definition}

\begin{example}
Il volume e l'energia sono grandezze estensive mentre la pressione e la temperatura sono intensive.
\end{example}

\begin{remark}
Il lavoro meccanico \`e dato da $W=\int \vec F\cdot \vec{d\ell}$. \`E un fatto generale che il lavoro ha la forma
\[\int(\text{intensiva})d(\text{estensiva}).\]
\end{remark}

\begin{definition}[Sistemi isolati, chiusi e aperti]
Un sistema termodinamico si dice 
\begin{itemize}
\item \textbf{isolato} se non ammette scambio con l'ambiente,
\item \textbf{chiuso} se non ammette scambio di materia con l'ambiente,
\item \textbf{aperto} se ammette scambi con l'ambiente.
\end{itemize}
\end{definition}

Per considerare pi\`u sistemi termodinamici dobbiamo considerarli come separati da una \textit{parete}.

\begin{definition}[Tipi di parete]
Una parete tra due sistemi \`e
\begin{itemize}
\item \textbf{adiabatica} se non permette scambi,
\item \textbf{diatermica} se non ammette scambi di materia,
\item \textbf{semipermeabile} se fa passare alcuni tipi di materia.
\item \textbf{permeabile}\footnote{una parete permeabile \`e come se non ci fosse} se permette ogni tipo di scambio.
\end{itemize}
\end{definition}

\section{Principio 0, Equilibrio e Temperatura empirica}
\subsection{Tipi di equilibrio}
\begin{definition}[Equilibrio]
Un sistema \`e in \textbf{equilibrio} se le sue funzioni di stato restano ``costanti" (per molto tempo rispetto alla scala temporale rilevante).\\
Un sistema \`e in \textbf{equilibrio termico} se non ci sono differenze di temperatura\footnote{definiremo la temperatura in seguito.}.\\
Un sistema \`e in \textbf{equilibrio termodinamico} se \`e in equilibrio meccanico, termico e chimico.
\end{definition}

\begin{remark}
I sistemi tendono spontaneamente ed irreversibilmente all'equilibrio termodinamico.
\end{remark}

\begin{fact}[0-esimo principio della termodinamica]
\textbf{Due sistemi in equilibrio termico con un terzo sono in equilibrio tra loro.}
\end{fact}

\begin{definition}[Equazione di stato]
Se quando un sistema \`e in equilibrio vale una equazione tra le funzioni di stato, queste si dicono \textbf{equazioni di stato}.
\end{definition}

\begin{remark}[Segno degli scambi di energia]
\underline{\textbf{NOTA BENE:}} Affermiamo per convenzione che uno scambio di energia ha segno \textit{positivo} se il sistema \textbf{\textit{acquista energia dall'ambiente}}.
\end{remark}



\subsection{Processi quasistatici}

\begin{definition}[Processi quasistatici]
Un sistema \`e \textbf{quasi in equilibrio} se \`e cos\`i vicino all'equilibrio che le equazioni di stato si possono considerare valide.\\ 
Un \textbf{processo quasistatico} \`e un processo tale per cui il sistema \`e quasi in equilibrio in ogni istante.\\
Se non sono presenti ``attriti", un processo quasistatico \`e detto \textbf{reversibile}.\\
Un processo \`e detto \textbf{totalmente reversibile} se \`e reversibile e la sua interazione con l'ambiente \`e reversibile.
\end{definition}

\begin{remark}
Un processo quasi statico va pensato come un processo molto lento; cos\`i lento da poter pensare al sistema come ``sempre in equilibrio".
\end{remark}


\begin{definition}[Principali processi quasistatici per gas]
Un processo si dice\footnote{Per le definizioni di temperatura o calore finite di leggere questo capitolo.}
\begin{itemize}
\item \textbf{isotermo} se $T$ resta costante,
\item \textbf{isobaro} se $p$ resta costante,
\item \textbf{isocore} se $V$ resta costante o
\item \textbf{adiabatico} se non avviene scambio di calore.
\end{itemize}
\end{definition}



\subsection{Temperatura empirica}

\begin{proposition}[Temperatura empirica]\label{TemperaturaEmpirica}
Ogni sistema termodinamico ammette una funzione che \`e costante in stato di equilibrio. La costante \`e detta \textbf{temperatura empirica}.
\end{proposition}
\begin{proof}
Consideriamo tre sistemi, con funzioni di stato $(x_1, y_1),\ (x_2,y_2)$ e $(x_3,y_3)$ in equilibrio tra loro. Esistono dunque equazioni di stato della forma
\[\begin{cases}
x_3=f(x_1,y_1,y_3)\\
x_3=g(x_2,y_2,y_3)
\end{cases}\]
poich\'e i sistemi 1 e 2 sono in equilibrio, se eguagliamo le due equazioni sappiamo che ci\`o che otteniamo non dipende da $y_3$, quindi
\[\begin{cases}
f(x_1,y_1,y_3)=\phi_1(x_1,y_1)\zeta(y_3)+\eta(y_3)\\
g(x_2,y_2,y_3)=\phi_2(x_2,y_2)\zeta(y_3)+\eta(y_3)
\end{cases}\]
dunque se 1 e 2 sono in equilibrio si ha che 
\[\phi_1(x_1,y_1)=\phi_2(x_2,y_2),\]
ma i due membri dipendono da insiemi di variabili disgiunti, quindi esiste $\theta_0$ tale che entrambe queste espressioni eguagliano $\theta_0$ se sono in equilibrio. Il valore $\theta_0$ \`e detto la temperatura empirica dei sistemi, i quali sono in equilibrio solo se hanno la stessa temperatura empirica.
\end{proof}

\begin{definition}[Isoterme]
Dato un sistema termodinamico e un valore $\theta_0$ di temperatura empirica, chiamiamo \textbf{isoterma a livello $\theta_0$} l'insieme degli stati del sistema la cui temperatura \`e $\theta_0$.
\end{definition}


\subsection{Definizione di temperatura tramite gas}
\begin{fact}[Punto triplo]
\emph{Considerando come sistema termodinamico dell'acqua esiste una precisa combinazione di temperatura e pressione tale per cui essa risulta in trasizione tra gli stati solido liquido e gassoso simultaneamente.\\
Questo stato si chiama \textbf{punto triplo} e i valori in questione sono una temperatura di $0.01 ^\circ \mathrm{C}$ e una pressione di $0.006\ \mathrm{atm}$.}
\end{fact}
\medskip

\noindent
A bassa pressione i gas si comportano tutti allo stesso modo\footnote{rispettano l'equazione di stato $pV=f(\theta)$}.\\
Se fissiamo il volume e la quantit\`a di materia del gas possiamo definire $\theta$ in modo tale che $p=p_0(1+\al\theta)$, cio\`e poniamo 
\[\theta=\frac1\al\frac{p-p_0}{p_0}.\] 
Se imponiamo che l'acqua congeli per $\theta=0$ e bollisca per $\theta=100$ allora si ricaviamo $1/\al=273.15$. Notiamo inoltre\footnote{l'addizione di $\al\ii$ corrisponde alla traslazione che trasforma gradi Celsius in gradi Kelvin.}
\[\frac{p_2}{p_1}=\frac{\al\ii+\theta_2}{\al\ii +\theta_1}=\frac{\theta_2'}{\theta_1'}.\]
Possiamo dunque definire la temperatura (in Kelvin) come
\[T=\lim_{p^{(PT)}\to 0}273.16 \frac{p}{p^{(PT)}}\]
dove $p^{(PT)}$ \`e la pressione del gas nel termometro quando questo sistema \`e in equilibrio con il sistema di punto triplo con l'acqua. Il limite corrisponde a prendere gas sempre pi\`u rarefatti, cio\`e a lavorare nel limite dei gas perfetti dove vale la proporzionalit\`a sopra.
\medskip

\noindent Sfruttando questa definizione possiamo costruire un termometro a gas come in figura

[FIGURA TERMOMETRO A GAS]

\noindent Quando il gas \`e alla temperatura che vogliamo misurare, misuriamo la differenza di altezza tra il livello a contatto con il gas e il livello di controllo posto a pressione atmosferica. 
Questa differenza \`e proporzionale alla differenza di pressione e questo ci permette di ricavare la temperatura se la fissiamo per quando \`e nel punto critico.


