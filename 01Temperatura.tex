\chapter{Temperatura e calore}
\noindent
La termodinamica \`e lo studio di sistemi dal punto di vista macroscopico.\\
Le massime fondamentali della termodinamica sono
\begin{itemize}
\item L'energia dell'universo \`e costante
\item L'entropia dell'universo tende ad aumentare.
\end{itemize}

\section{Prime definizioni}
\begin{definition}[Sistema termodinamico]
Un \textbf{sistema termodinamico} \`e un sistema omogeneo composto da ``molti" elementi.\\
Lo \textbf{stato} di un sistema termodinamico \`e univocamente determinato da un numero contenuto di parametri\footnote{Per esempio temperatura, pressione o volume.} detti \textbf{funzioni di stato}.\\
Il numero di funzioni di stato necessarie per specificare lo stato \`e detto \textbf{numero di gradi di libert\`a}.
\end{definition}

\begin{remark}
Le funzioni di stato di un sistema non dipendono da come esso \`e venuto ad esistere; se due procedimenti portano da un particolare stato ad un altro, le differenze nelle funzioni di stato dipendono univocamente dallo stato iniziale e quello finale.
\end{remark}

\begin{remark}[Sistema ambiente]
Spesso torna comodo considerare una coppia di sistemi, uno detto semplicemente sistema e l'altro \textbf{ambiente}.
\end{remark}

\begin{definition}[Variabili estensive e intensive]
Dato un sistema termodinamico, delle variabili ad esso inerenti si dicono \textbf{estensive} se sono proporzionali alla quantit\`a di materia contenuta nel sistema e \textbf{intensive} altrimenti.
\end{definition}

\begin{example}
Il volume e l'energia sono grandezze estensive mentre la pressione e la temperatura sono intensive.
\end{example}

\begin{definition}[Sistemi isolati, chiusi e aperti]
Un sistema termodinamico si dice 
\begin{itemize}
\item \textbf{isolato} se non ammette scambio con l'ambiente,
\item \textbf{chiuso} se non ammette scambio di materia con l'ambiente,
\item \textbf{aperto} se ammette scambi con l'ambiente.
\end{itemize}
\end{definition}

Per considerare pi\`u sistemi termodinamici dobbiamo considerarli come separati da una \textit{parete}.

\begin{definition}[Tipi di parete]
Una parete tra due sistemi \`e
\begin{itemize}
\item \textbf{adiabatica} se non permette scambi,
\item \textbf{diatermica} se non ammette scambi di materia,
\item \textbf{semipermeabile} se fa passare alcuni tipi di materia.
\item \textbf{permeabile}\footnote{una parete permeabile \`e come se non ci fosse} se permette ogni tipo di scambio.
\end{itemize}
\end{definition}

\begin{definition}[Equilibrio]
Un sistema \`e in \textbf{equilibrio} se le sue funzioni di stato restano ``costanti" (per molto tempo rispetto alla scala temporale rilevante).\\
Un sistema \`e in \textbf{equilibrio termico} se non ci sono differenze di temperatura\footnote{definiremo la temperatura in seguito.}.\\
Un sistema \`e in \textbf{equilibrio termodinamico} se \`e in equilibrio meccanico, termico e chimico.
\end{definition}

\begin{remark}
I sistemi tendono spontaneamente ed irreversibilmente all'equilibrio termodinamico.
\end{remark}

\begin{definition}[Equazione di stato]
Se quando un sistema \`e in equilibrio vale una equazione tra le funzioni di stato, queste si dicono \textbf{equazioni di stato}.
\end{definition}

\begin{definition}[Tipi di trasferimenti di energia]
Considerato un sistema termodinamico e l'ambiete definiamo le seguenti tipologie di scambi di energia:
\begin{itemize}
\item uno scambio di energia meccanica \`e detto \textbf{lavoro},
\item uno scambio di energia termica \`e detto \textbf{calore},
\item uno scambio di energia chimica \`e definito da \[\Delta E=\int \mu dn,\] dove $n$ \`e il numero di particelle coinvolte e $\mu$ \`e il \textbf{potenziale chimico}.\footnote{questa quantit\`a ha senso solo per sistemi aperti.}
\end{itemize}
Affermiamo per convenzione che uno scambio di energia ha segno \textit{positivo} se il sistema \underline{acquista energia dall'ambiente}.
\end{definition}

\begin{remark}
Il lavoro meccanico \`e dato da $W=\int \vec F\cdot \vec{d\ell}$. \`E un fatto generale che il lavoro ha la forma
\[\int(\text{intensiva})d(\text{estensiva}).\]
\end{remark}



\begin{definition}[Processi quasistatici]
Un sistema \`e \textbf{quasi in equilibrio} se \`e cos\`i vicino all'equilibrio che le equazioni di stato si possono considerare valide. Un \textbf{processo quasistatico} \`e descrivibile da una successione di variazioni infinitesime tra stati vicini all'equilibrio.\\
Se non sono presenti ``attriti", un processo quasistatico \`e detto \textbf{reversibile}.\\
Un processo \`e detto \textbf{totalmente reversibile} se \`e reversibile e la sua interazione con l'ambiente \`e reversibile.
\end{definition}

\section{Definizione di temperatura}
\begin{fact}[0-esimo principio della termodinamica]
\textbf{Due sistemi in equilibrio termico con un terzo sono in equilibrio tra loro.}
\end{fact}

\begin{proposition}[Temperatura empirica]\label{TemperaturaEmpirica}
Ogni sistema termodinamico ammette una funzione che \`e costante in stato di equilibrio. La costante \`e detta \textbf{temperatura empirica}.
\end{proposition}
\begin{proof}
Consideriamo tre sistemi, con funzioni di stato $(x_1, y_1),\ (x_2,y_2)$ e $(x_3,y_3)$ in equilibrio tra loro. Esistono dunque equazioni di stato della forma
\[\begin{cases}
x_3=f(x_1,y_1,y_3)\\
x_3=g(x_2,y_2,y_3)
\end{cases}\]
poich\'e i sistemi 1 e 2 sono in equilibrio, se eguagliamo le due equazioni sappiamo che ci\`o che otteniamo non dipende da $y_3$, quindi
\[\begin{cases}
f(x_1,y_1,y_3)=\phi_1(x_1,y_1)\zeta(y_3)+\eta(y_3)\\
g(x_2,y_2,y_3)=\phi_2(x_2,y_2)\zeta(y_3)+\eta(y_3)
\end{cases}\]
dunque se 1 e 2 sono in equilibrio si ha che 
\[\phi_1(x_1,y_1)=\phi_2(x_2,y_2),\]
ma i due membri dipendono da insiemi di variabili disgiunti, quindi esiste $\theta_0$ tale che entrambe queste espressioni eguagliano $\theta_0$ se sono in equilibrio. Il valore $\theta_0$ \`e detto la temperatura empirica dei sistemi, i quali sono in equilibrio solo se hanno la stessa temperatura empirica.
\end{proof}

\begin{definition}[Isoterme]
Dato un sistema termodinamico e un valore $\theta_0$ di temperatura empirica, chiamiamo \textbf{isoterma a livello $\theta_0$} l'insieme degli stati del sistema la cui temperatura \`e $\theta_0$.
\end{definition}


\begin{fact}[Punto triplo]
\emph{Considerando come sistema termodinamico dell'acqua esiste una precisa combinazione di temperatura e pressione tale per cui essa risulta in trasizione tra gli stati solido liquido e gassoso simultaneamente.\\
Questo stato si chiama \textbf{punto triplo} e i valori in questione sono una temperatura di $0.01 ^\circ \mathrm{C}$ e una pressione di $0.006\ \mathrm{atm}$.}
\end{fact}


\subsection{Definizione di temperatura tramite gas}
A bassa pressione i gas si comportano tutti allo stesso modo\footnote{rispettano l'equazione di stato $pV=f(\theta)$}.\\
Se fissiamo il volume e la quantit\`a di materia del gas possiamo definire $\theta$ in modo tale che $p=p_0(1+\al\theta)$, cio\`e poniamo 
\[\theta=\frac1\al\frac{p-p_0}{p_0}.\] 
Se imponiamo che l'acqua congeli per $\theta=0$ e bollisca per $\theta=100$ allora si ricaviamo $1/\al=273.15$. Notiamo inoltre\footnote{l'addizione di $\al\ii$ corrisponde alla traslazione che trasforma gradi Celsius in gradi Kelvin.}
\[\frac{p_2}{p_1}=\frac{\al\ii+\theta_2}{\al\ii +\theta_1}=\frac{\theta_2'}{\theta_1'}.\]
Possiamo dunque definire la temperatura (in Kelvin) come
\[T=\lim_{p^{(PT)}\to 0}273.16 \frac{p}{p^{(PT)}}\]
dove $p^{(PT)}$ \`e la pressione del gas nel termometro quando questo sistema \`e in equilibrio con il sistema di punto triplo con l'acqua. Il limite corrisponde a prendere gas sempre pi\`u rarefatti, cio\`e a lavorare nel limite dei gas perfetti dove vale la proporzionalit\`a sopra.
\medskip

\noindent Sfruttando questa definizione possiamo costruire un termometro a gas come in figura

[FIGURA TERMOMETRO A GAS]

\noindent Quando il gas \`e alla temperatura che vogliamo misurare misuriamo la differenza di altezza tra il livello a contatto con il gas e il livello di controllo posto a pressione atmosferica. Questa differenza \`e proporzionale alla differenza di pressione e questo ci permette di ricavare la temperatura se la fissiamo per quando \`e nel punto critico.

\section{Relazione tra parametri indipendenti ed espressioni per l'energia}

\begin{fact}[Relazione tra parametri indipendenti e espressioni per l'energia]
Il numero di parametri di un sistema indipendenti \`e pari al numero di coppie di variabili che compaiono nelle espressioni per l'energia.
\end{fact}

\begin{example}[Filo]
Un filo ha come funzioni di stato la lunghezza, la tensione e la temperatura, che indichiamo $L,\ \tau\ \text{e }T$ rispettivamente.\\
Le formule per l'energia contengono $\tau$ ed $L$ per il lavoro ($\delta W=\tau dL$) e $L$ e $T$ per il calore\footnote{$L$ appare implicitamente in quanto unica grandezza estensiva.}. Segue che il sistema filo ha due parametri indipendenti, dunque deve esistere una equazione che lega i parametri citati. In questo caso \`e la legge di Hooke ($\tau=-k(L-L_0)$).
\end{example}

\begin{example}[Fluidi]
Ragioniamo in modo simile a prima, stavolta i parametri sono volume, pressione e temperatura.
\end{example}

\section{Gas ideali}

\begin{definition}[Mole]
Una \textbf{mole di una sostanza} corrisponde a $6.02\cdot 10^{23}$ particelle di quella sostanza. La costante \`e detta \textbf{numero di Avogadro} e la indichiamo con $N_a$. 
\end{definition}

\begin{definition}[Densit\`a]
Definiamo la \textbf{denstit\`a} come
\[\rho=\frac mV.\]
\end{definition}
\begin{remark}
Il differenziale della densit\`a \`e
\[d\rho=-\frac m{V^2}dV.\]
\end{remark}

\begin{definition}[Condizioni standard]
Un gas \`e in \textbf{condizioni standard} (\textbf{STP}) se \`e alla temperatura di $0^\circ\mathrm C$ e alla pressione di $1\ \mathrm{atm}=101.3245\ \mathrm{kPa}$.
\end{definition}

\noindent Per i gas ideali valgono le seguenti leggi:
\begin{fact}[Legge di Boyle]
Se $T$ \`e costante
\[V\propto \frac1p\]
\end{fact}
\begin{fact}[Legge di Charles]
Se $p$ \`e costante
\[V\propto (1+\al T)\]
\end{fact}
\begin{fact}[Legge di Gay-Lussac]
Se $V$ \`e costante
\[p\propto T\]
\end{fact}
\begin{fact}[Legge di Avogadro]
Se $p$ e $T$ sono fissate, tutti i gas occupano lo stesso volume se consistono della stessa quantit\`a di materia, in particolare
\[V\propto n.\]
Una mole di gas in condizioni standard occupa un volume di $22.4\ell$ (litri).
\end{fact}

\noindent Combinando le leggi appena citate arriviamo alla \textbf{legge dei Gas perfetti}
\[\boxed{pV=nRT}\]
dove $p$ \`e la pressione, $V$ \`e il volume, $n$ \`e il numero di moli, $T$ \`e la temperatura e $R$ \`e la \textbf{costante fondamentale dei gas} e vale $8.314 \frac{\mathrm{J}}{\mathrm{K}\ \mathrm{mol}}$.

\begin{definition}[Costante di Boltzmann]
Definiamo la \textbf{costante di Boltzmann} $k_b$ in modo tale che 
\[R=N_a k_b.\]
\end{definition}




\section{Esempi principali di processi quasistatici}
\begin{definition}[Tipi rilevanti di processi]
Un processo si dice
\begin{itemize}
\item \textbf{isotermo} se $T$ resta costante,
\item \textbf{isobaro} se $p$ resta costante,
\item \textbf{isocore} se $V$ resta costante o
\item \textbf{adiabatico} se non avviene scambio di calore\footnote{definiremo il calore successivamente. Intuitivamente \`e lo scambio di energia che deriva da una variazione di temperatura.}.
\end{itemize}
\end{definition}

\begin{definition}[Coefficiente di espansione volumetrica]
Definiamo il \textbf{coefficiente di espansione volumetrica} come
\[\al=\frac1V\ppb TVp=-\frac mV\frac1{\rho^2}\ppb T\rho p=-\frac1\rho\ppb T\rho p.\]
L'unit\`a di misura \`e $[\al]=\mathrm{K}\ii$.
\end{definition}


\begin{definition}[Compressibilit\`a isoterma]
Definiamo la \textbf{compressibilit\`a isoterma} come
\[\beta_T=-\frac1V\ppb pVT.\]
L'unit\`a di misura \`e $[\beta_T]=\mathrm{Pa}\ii$.\\
L'inversa $k_T=1/\beta_T$ \`e detta \textbf{modulo di compressibilit\`a isoterma}.
\end{definition}

\noindent Riportiamo alcuni valori di $\al$ e $\beta_T$ per dare una intuizione sui valori tipici\footnote{il Sitall \`e materiale fatto apposta per avere coefficiente di espansione volumetrica piccolo}
\begin{center}
\begin{tabular}[ht]{|c|c|c|}
\hline 
Materiale&$\al\ [\mathrm{K}\ii]$&$\beta_T\ [\mathrm{Pa}\ii]$\\\hline
Acqua&$0.2\cdot 10^{-3}$&$4.6\cdot 10^{-10}$\\
Diamante&$3\cdot 10^{-6}$&?\\
Sitall&$\leq 10^{-7}$&?\\
Sabbia&?&$\sim10^{-8}$\\
Mercurio&$1.8\cdot 10^{-4}$&$4\cdot10^{-11}$\\
Rame&?&$7.2\cdot10^{-12}$\\
\hline
\end{tabular}
\end{center}


\begin{remark}
Non \`e necessario battezzare $\displaystyle\ppb TpV$ in quanto per la propriet\`a ciclica (\ref{ProprietaDerivateParziali})
\[\ppb TpV=-\ppb VpT\ppb TVp=\frac\al{\beta_T}.\]
\end{remark}

\begin{remark}[Relazione differenziale tra $\al$ e $\beta_T$]
Per il teorema di Schwarz si ha che
\[\pp{p\del T}{^2V}=\ppb p\al T=-\ppb T{\beta_T}p.\]
\end{remark}

\begin{proposition}[$\al$ e $\beta_T$ per gas ideali]\label{CoefficienteEspansioneECompressibilitaIsotermaPerGasIdeali}
Se il sistema in esame \`e un gas ideale valgono le seguenti identit\`a:
\[\al=\frac 1T,\qquad \beta_T=\frac 1p.\]
\end{proposition}
\begin{proof}
Segue calcolando:
\begin{align*}
\al=&\frac1V\ppb T{(nRT/p)}p=\frac{nR}{pV}=\frac1T,\\
\beta_T=&-\frac1V\ppb p{(nRT/p)}T=\frac1V nRT\frac1{p^2}=\frac1p.
\end{align*}
\end{proof}

\begin{proposition}[Differenziale della pressione]\label{DifferenzialePressione}
Si ha che
\[dp=\frac\al{\beta_T}dT-\frac1{\beta_T V}dV.\]
\end{proposition}
\begin{proof}
Osserviamo che
\[\ppb TpV\pasgnl={(\ref{ProprietaDerivateParziali})}-\ppb VpT \ppb TVp=\frac \al{\beta_T},\]
dunque ricaviamo
\[dp=\ppb TpVdT+\ppb VpT=\frac\al{\beta_T}dT-\frac1{\beta_T V}dV.\]
\end{proof}
\begin{corollary}
In una trasformazione isocora $\Delta p=\frac\al{\beta_T}\Delta T$.
\end{corollary}

\begin{remark}[Differenziale logaritmico nel volume]\label{DifferenzialeLogaritmicoNelVolume}
Spesso torner\`a comodo ricordare il seguente sviluppo differenziale
\[d\log V=\frac1VdV=\al dT-\beta_T dp\]
\end{remark}
\begin{proof}
Segue calcolando:
\[\frac1VdV=\frac1V\pa{\ppb TVp dT+\ppb pVTdp}=\al dT-\beta_T dp\]
\end{proof}


\subsection{Compressione e Lavoro}
Immagino di comprimere un sistema come in figura 

[FIGURA]

\noindent
Se spingiamo molto lentamente possiamo con buona approssimazione supporre che il processo sia quasistatico, dunque $F=pS$. Segue che
\[\boxed{\delta W=Fdx=pSdx}\]
Se il sistema in questione \`e un gas ideale allora
\[\delta W=p(-dV)=-pdV\]
Il lavoro totale per passare da uno stato $A$ ad uno stato $B$ diventa
\[W=-\int_{A}^{B} p(V,T)dV,\]
ma $p$ come cambia al variare di $V$? Dipende dal tipo di processo.

[QUALCHE GRAFICO]

\noindent
Questo mostra in particolare che il lavoro non \`e una funzione di stato.
