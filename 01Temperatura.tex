\chapter{Principi della termodinamica}
\noindent
La termodinamica \`e lo studio di sistemi dal punto di vista macroscopico.\\
Le massime fondamentali della termodinamica sono
\begin{itemize}
\item L'energia dell'universo \`e costante
\item L'entropia dell'universo tende ad aumentare.
\end{itemize}

\section{Prime definizioni}
\begin{definition}[Sistema termodinamico]
Un \textbf{sistema termodinamico} \`e un sistema omogeneo composto da ``molti" elementi.\\
Lo \textbf{stato} di un sistema termodinamico \`e univocamente determinato da un numero contenuto di parametri\footnote{Per esempio temperatura, pressione o volume.} detti \textbf{funzioni di stato}.\\
Il numero di funzioni di stato necessarie per specificare lo stato \`e detto \textbf{numero di gradi di libert\`a}.
\end{definition}

\begin{remark}
Le funzioni di stato di un sistema non dipendono da come esso \`e venuto ad esistere; se due procedimenti portano da un particolare stato ad un altro, le differenze nelle funzioni di stato dipendono univocamente dallo stato iniziale e quello finale.
\end{remark}

\begin{remark}[Sistema ambiente]
Spesso torna comodo considerare una coppia di sistemi, uno detto semplicemente sistema e l'altro \textbf{ambiente}.
\end{remark}

\begin{definition}[Variabili estensive e intensive]
Dato un sistema termodinamico, delle variabili ad esso inerenti si dicono \textbf{estensive} se sono proporzionali alla quantit\`a di materia contenuta nel sistema e \textbf{intensive} altrimenti.
\end{definition}

\begin{example}
Il volume e l'energia sono grandezze estensive mentre la pressione e la temperatura sono intensive.
\end{example}

\begin{remark}
Il lavoro meccanico \`e dato da $W=\int \vec F\cdot \vec{d\ell}$. \`E un fatto generale che il lavoro ha la forma
\[\int(\text{intensiva})d(\text{estensiva}).\]
\end{remark}

\begin{definition}[Sistemi isolati, chiusi e aperti]
Un sistema termodinamico si dice 
\begin{itemize}
\item \textbf{isolato} se non ammette scambio con l'ambiente,
\item \textbf{chiuso} se non ammette scambio di materia con l'ambiente,
\item \textbf{aperto} se ammette scambi con l'ambiente.
\end{itemize}
\end{definition}

Per considerare pi\`u sistemi termodinamici dobbiamo considerarli come separati da una \textit{parete}.

\begin{definition}[Tipi di parete]
Una parete tra due sistemi \`e
\begin{itemize}
\item \textbf{adiabatica} se non permette scambi,
\item \textbf{diatermica} se non ammette scambi di materia,
\item \textbf{semipermeabile} se fa passare alcuni tipi di materia.
\item \textbf{permeabile}\footnote{una parete permeabile \`e come se non ci fosse} se permette ogni tipo di scambio.
\end{itemize}
\end{definition}

\section{Principio 0, Equilibrio e Temperatura empirica}
\subsection{Tipi di equilibrio}
\begin{definition}[Equilibrio]
Un sistema \`e in \textbf{equilibrio} se le sue funzioni di stato restano ``costanti" (per molto tempo rispetto alla scala temporale rilevante).\\
Un sistema \`e in \textbf{equilibrio termico} se non ci sono differenze di temperatura\footnote{definiremo la temperatura in seguito.}.\\
Un sistema \`e in \textbf{equilibrio termodinamico} se \`e in equilibrio meccanico, termico e chimico.
\end{definition}

\begin{remark}
I sistemi tendono spontaneamente ed irreversibilmente all'equilibrio termodinamico.
\end{remark}

\begin{fact}[0-esimo principio della termodinamica]
\textbf{Due sistemi in equilibrio termico con un terzo sono in equilibrio tra loro.}
\end{fact}

\begin{definition}[Equazione di stato]
Se quando un sistema \`e in equilibrio vale una equazione tra le funzioni di stato, queste si dicono \textbf{equazioni di stato}.
\end{definition}

\begin{remark}[Segno degli scambi di energia]
\underline{\textbf{NOTA BENE:}} Affermiamo per convenzione che uno scambio di energia ha segno \textit{positivo} se il sistema \textbf{\textit{acquista energia dall'ambiente}}.
\end{remark}



\subsection{Processi quasistatici}

\begin{definition}[Processi quasistatici]
Un sistema \`e \textbf{quasi in equilibrio} se \`e cos\`i vicino all'equilibrio che le equazioni di stato si possono considerare valide.\\ 
Un \textbf{processo quasistatico} \`e un processo tale per cui il sistema \`e quasi in equilibrio in ogni istante.\\
Se non sono presenti ``attriti", un processo quasistatico \`e detto \textbf{reversibile}.\\
Un processo \`e detto \textbf{totalmente reversibile} se \`e reversibile e la sua interazione con l'ambiente \`e reversibile.
\end{definition}

\begin{remark}
Un processo quasi statico va pensato come un processo molto lento; cos\`i lento da poter pensare al sistema come ``sempre in equilibrio".
\end{remark}


\begin{definition}[Principali processi quasistatici per gas]
Un processo si dice\footnote{Per le definizioni di temperatura o calore finite di leggere questo capitolo.}
\begin{itemize}
\item \textbf{isotermo} se $T$ resta costante,
\item \textbf{isobaro} se $p$ resta costante,
\item \textbf{isocore} se $V$ resta costante o
\item \textbf{adiabatico} se non avviene scambio di calore.
\end{itemize}
\end{definition}



\subsection{Temperatura empirica}

\begin{proposition}[Temperatura empirica]\label{TemperaturaEmpirica}
Ogni sistema termodinamico ammette una funzione che \`e costante in stato di equilibrio. La costante \`e detta \textbf{temperatura empirica}.
\end{proposition}
\begin{proof}
Consideriamo tre sistemi, con funzioni di stato $(x_1, y_1),\ (x_2,y_2)$ e $(x_3,y_3)$ in equilibrio tra loro. Esistono dunque equazioni di stato della forma
\[\begin{cases}
x_3=f(x_1,y_1,y_3)\\
x_3=g(x_2,y_2,y_3)
\end{cases}\]
poich\'e i sistemi 1 e 2 sono in equilibrio, se eguagliamo le due equazioni sappiamo che ci\`o che otteniamo non dipende da $y_3$, quindi
\[\begin{cases}
f(x_1,y_1,y_3)=\phi_1(x_1,y_1)\zeta(y_3)+\eta(y_3)\\
g(x_2,y_2,y_3)=\phi_2(x_2,y_2)\zeta(y_3)+\eta(y_3)
\end{cases}\]
dunque se 1 e 2 sono in equilibrio si ha che 
\[\phi_1(x_1,y_1)=\phi_2(x_2,y_2),\]
ma i due membri dipendono da insiemi di variabili disgiunti, quindi esiste $\theta_0$ tale che entrambe queste espressioni eguagliano $\theta_0$ se sono in equilibrio. Il valore $\theta_0$ \`e detto la temperatura empirica dei sistemi, i quali sono in equilibrio solo se hanno la stessa temperatura empirica.
\end{proof}

\begin{definition}[Isoterme]
Dato un sistema termodinamico e un valore $\theta_0$ di temperatura empirica, chiamiamo \textbf{isoterma a livello $\theta_0$} l'insieme degli stati del sistema la cui temperatura \`e $\theta_0$.
\end{definition}


\subsection{Definizione di temperatura tramite gas}
\begin{fact}[Punto triplo]
\emph{Considerando come sistema termodinamico dell'acqua esiste una precisa combinazione di temperatura e pressione tale per cui essa risulta in trasizione tra gli stati solido liquido e gassoso simultaneamente.\\
Questo stato si chiama \textbf{punto triplo} e i valori in questione sono una temperatura di $0.01 ^\circ \mathrm{C}$ e una pressione di $0.006\ \mathrm{atm}$.}
\end{fact}
\medskip

\noindent
A bassa pressione i gas si comportano tutti allo stesso modo\footnote{rispettano l'equazione di stato $pV=f(\theta)$}.\\
Se fissiamo il volume e la quantit\`a di materia del gas possiamo definire $\theta$ in modo tale che $p=p_0(1+\al\theta)$, cio\`e poniamo 
\[\theta=\frac1\al\frac{p-p_0}{p_0}.\] 
Se imponiamo che l'acqua congeli per $\theta=0$ e bollisca per $\theta=100$ allora si ricaviamo $1/\al=273.15$. Notiamo inoltre\footnote{l'addizione di $\al\ii$ corrisponde alla traslazione che trasforma gradi Celsius in gradi Kelvin.}
\[\frac{p_2}{p_1}=\frac{\al\ii+\theta_2}{\al\ii +\theta_1}=\frac{\theta_2'}{\theta_1'}.\]
Possiamo dunque definire la temperatura (in Kelvin) come
\[T=\lim_{p^{(PT)}\to 0}273.16 \frac{p}{p^{(PT)}}\]
dove $p^{(PT)}$ \`e la pressione del gas nel termometro quando questo sistema \`e in equilibrio con il sistema di punto triplo con l'acqua. Il limite corrisponde a prendere gas sempre pi\`u rarefatti, cio\`e a lavorare nel limite dei gas perfetti dove vale la proporzionalit\`a sopra.
\medskip

\noindent Sfruttando questa definizione possiamo costruire un termometro a gas come in figura

[FIGURA TERMOMETRO A GAS]

\noindent Quando il gas \`e alla temperatura che vogliamo misurare, misuriamo la differenza di altezza tra il livello a contatto con il gas e il livello di controllo posto a pressione atmosferica. 
Questa differenza \`e proporzionale alla differenza di pressione e questo ci permette di ricavare la temperatura se la fissiamo per quando \`e nel punto critico.

\section{Primo principio e Definizione di calore}

\begin{fact}[Primo principio della termodinamica]
\textbf{L'energia interna di un sistema di conserva.}
\end{fact}

\begin{definition}[Calore]
Il \textbf{calore} \`e la differenza tra la variazione di energia interna e il lavoro compiuto su un sistema termodinamico, esplicitamente
\[\boxed{\Delta U=Q+W}\]
\end{definition}

\begin{remark}
Il calore e il lavoro non sono funzioni di stato, ma la loro somma s\`i.
\end{remark}

\begin{remark}[Primo principio in forma differenziale]
Scrivendo il primo principio in termini di infinitesimi troviamo
\[dU=\delta Q+\delta W,\]
in particolare per i gas ideali vale
\[dU=\delta Q-pdV.\]
\end{remark}


\begin{definition}[Caloria]
Una \textbf{caloria} \`e la quantit\`a di calore necessaria per far variare la temperatura di un grammo di acqua da $14.5^\circ\mathrm{C}$ a $15.5^\circ\mathrm{C}$.\\
In Joule si ha che
\[\boxed{1\ \mathrm{cal}=4.186\ \mathrm{J}}\]
\end{definition}


\begin{remark}
In una trasformazione adiabatica, il lavoro \`e dato dalla differenza di energia interna.
\end{remark}
\begin{example}[Coppia di sistemi dentro un contenitore adiabatico]
Consideriamo due sistemi $A$ e $B$ dentro un contenitore adiabatico. Per il primo principio
\[0=\Delta U=\Delta U_A+\Delta U_B=Q_A+Q_B+\under{=W}{W_A+W_B}.\]
I trasferimenti di calore possono avvenire solo tra $A$ e $B$, quindi $Q_A+Q_B=0$ e $W=0$. Quanto scritto \`e una ``legge di conservazione del calore" in questo tipo di sistema.
\end{example}

\section{Secondo principio e cicli}
\subsection{Enunciati del secondo principio}
\begin{fact}[Secondo principio della termodinamica, formulazione di Kelvin]
\textbf{Non esiste un processo che traformi \ul{interamente} calore in lavoro.}
\end{fact}

\begin{fact}[Secondo principio della termodinamica, formulazione di Clausius]
\textbf{Non esiste un processo il cui \ul{unico risulato} sia trasferire calore da una sorgente pi\`u fredda ad una pi\`u calda.}
\end{fact}

\begin{proposition}
Le due formulazioni del secondo principio sono equivalenti.
\end{proposition}
\begin{proof}
Mostrimo che le loro negazioni sono equivalenti:
\setlength{\leftmargini}{0cm}
\begin{itemize}
\item[$\boxed{\neg K\implies \neg C}$] Consideriamo il diagramma

[DIAGRAMMA]

Notiamo che $\abs{Q'}+\abs{W}>\max{\cpa{\abs{Q'},\abs{W}}}=\max{\cpa{\abs{Q'},\abs{Q}}}$. Considerando ora il sistema di due macchine come un insieme troviamo una macchina che trasferisce un calore $\abs{Q'}$ dala sorgente fredda alla sorgente calda, negando Clausius.
\item[$\boxed{\neg C\implies \neg K}$] Procediamo analogamente a prima

[DIAGRAMMA]

e leggendo questo diagramma come un insieme la macchina avrebbe preso del calore $\abs Q-\abs{Q'}$ dalla sorgente $T_L$ e lo ha trasformato interamente in lavoro, negando Kelvin.
\end{itemize}
\setlength{\leftmargini}{0.5cm}
\end{proof}

\subsection{Processo ciclico}
\begin{definition}[Processo ciclico]
Un processo \`e \textbf{ciclico} se lo stato iniziale e finale sono lo stesso. Se qualcosa realizza un processo ciclico \`e detto \textbf{motore}.
\end{definition}

\begin{remark}[Diagramma di una macchina a due sorgenti]
Spesso torna comodo fare diagrammi come in figura

[DIAGRAMMA]
\end{remark}

\begin{remark}
Per un processo ciclico, $\Delta U=0$, dunque $Q=-W$.
\[-W=Q=Q_H+Q_L\]
dove $Q_H$ \`e il calore che il sistema acquista da una sorgente calda e $Q_L$ \`e il calore che acquista da una sorgente fredda\footnote{$Q_L$ \`e negativo}. Notiamo che
\[\abs W=\abs{Q_H}-\abs{Q_L}.\]
\end{remark}

\begin{definition}[Efficienza]
L'\textbf{efficienza} di un processo ciclico \`e data da\footnote{Intuitivamente l'efficienza \`e una misura di quanto lavoro riesco a realizzare in proporzione a quanto calore abbiamo dovuto inserire nel sistema. L'altra forma ci dice che l'efficienza \`e una coversione perfetta eccetto per il calore che viene disperso senza diventare lavoro ($Q_L$).}
\[\eta=\frac{\abs W}{\abs{Q_H}}=1-\frac{\abs{Q_L}}{\abs{Q_H}}.\]
\end{definition}

\begin{definition}[Frigorifero e coefficiente di prestazione]
Un \textbf{frigorifero} \`e un motore che trasferisce calore da una sorgente fredda ad una calda. Il suo \textbf{coefficiente di prestazione} \`e dato da
\[COP=\frac{\abs{Q_L}}{\abs{W}}=\frac{1-\eta}\eta.\]
\end{definition}

\begin{definition}[Pompa di calore]
Una \textbf{pompa di calore} \`e una macchina volta a trasformare lavoro in calore verso la sorgente calda. La sua efficienza \`e quindi l'inversa di quella di un motore standard:
\[\frac{\abs{Q_H}}{\abs{W}}=\frac1\eta.\]
\end{definition}

\begin{theorem}[di Carnot]\label{TeoremaDiCarnot}
Un ciclo reversibile \`e il pi\`u efficiente che lavori tra due sorgenti $\theta_H$ e $\theta_L$.
\end{theorem}
\begin{proof}
Consideriamo due cicli $S$ ed $S'$ di cui $S$ reversibile. Per il primo principio $-W=\abs{Q_H}-\abs{Q_L}$ e $-W'=\abs{Q_H'}-\abs{Q_L'}$.\\
Con precisione arbitraria, siano $N$ ed $N'$ interi positivi tali che
\[\frac{\abs{Q_H}}{\abs{Q_H'}}\approx \frac{N'}{N}.\]
Facendo fare $N'$ cicli a $S'$ ed $N$ cicli reversibili \textit{al contrario}\footnote{qu\`i usiamo la reversibilit\`a. Se prima il sistema trasformava calore in lavoro con qualche perdita di calore ora il sistema riceve lavoro e un po' di calore per fornire calore alla sorgente calda} a $S$ troviamo
\[-W_{tot}=N'(-W')-N(-W)=N'(\abs{Q_H'}-\abs{Q_L'})-N(\abs{Q_H}-\abs{Q_L})\]
\[Q_{H,tot}=N'\abs{Q_H'}-N\abs{Q_H}\]
\[-Q_{L,tot}=N'\abs{Q_L'}-N\abs{Q_L}.\]
Per il primo principio, facendo lavorare in parallelo le due macchine
\[-W_{tot}=Q_{H,tot}+Q_{L,tot}.\]
Scegliendo $N$ ed $N'$ arbitrariamente grandi possiamo approssimare $Q_{H,tot}\approx 0$, e quindi
\[-W_{tot}\approx Q_{L,tot}.\]
Per la formulazione di Kelvin del secondo principio si ha che $-W_{tot}\leq0$\footnote{se cos\`i non fosse la macchina composta starebbe convertendo il calore $\abs{Q_{L,tot}}$ in lavoro sull'esterno $\abs{W_{tot}}$, contraddicendo il secondo principio.}, quindi $Q_{L,tot}\leq 0$, cio\`e
\[N'\abs{Q_L'}-N\abs{Q_L}\geq 0\coimplies \frac{N'}N\geq \frac{\abs{Q_L}}{\abs{Q_L'}}.\]
Passando al limite negli $N$ e $N'$ si ha che
\[\frac{\abs{Q_L}}{\abs{Q_H}}\leq \frac{\abs{Q_L'}}{\abs{Q_H'}}\implies \eta=1-\frac{\abs{Q_L}}{\abs{Q_H}}\geq 1-\frac{\abs{Q_L'}}{\abs{Q_H'}}=\eta'.\]
\end{proof}

\begin{corollary}[I cicli reversibili hanno la stessa efficienza]\label{CicliReversibiliHannoLaStessaEfficienza}
Tutti i cicli reversibili hanno la stessa efficienza.
\end{corollary}
\begin{proof}
Applicando il teorema abbiamo le due disuguaglianze scambiando i ruoli tra i due cicli.
\end{proof}

\subsection{Definizione della temperatura tramite cicli reversibili}
Il teorema di Carnot (\ref{TeoremaDiCarnot}) ci suggerisce un modo per ridefinire la temperatura in termini della temperatura empirica senza bisogno di ricorrere ai gas:
\medskip

\noindent
Per il teorema di Carnot esiste $f$ tale che dopo un ciclo reversibile
\[\frac{\abs{Q_H}}{\abs{Q_L}}=f(\theta_L,\theta_H).\]
Collegando due tali processi facendo s\`i che il calore rilasciato dal primo sia quello assorbito dal secondo ricaviamo le equazioni
\[\frac{\abs{Q_3}}{\abs{Q_2}}=f(\theta_2,\theta_3),\quad\frac{\abs{Q_2}}{\abs{Q_1}}=f(\theta_1,\theta_2),\quad \frac{\abs{Q_3}}{\abs{Q_1}}=f(\theta_1,\theta_3),\]
dove $\theta_1\leq \theta_2\leq \theta_3$.\\
Segue dunque l'identit\`a
\[f(\theta_1,\theta_2)=\frac{f(\theta_1,\theta_3)}{f(\theta_2,\theta_3)}.\]
Derivando rispetto a $\theta_3$ ricaviamo
\begin{gather*}
0=\frac1{f(\theta_2,\theta_3)}\pp{\theta_3}{f}(\theta_1,\theta_3)-\frac{f(\theta_1,\theta_3)}{(f(\theta_2,\theta_3))^2}\pp{\theta_3}f(\theta_2,\theta_3)\\
\frac1{f(\theta_1,\theta_3)}\pp{\theta_3}{f}(\theta_1,\theta_3)=\frac1{f(\theta_2,\theta_3)}\pp{\theta_3}{f}(\theta_2,\theta_3).
\end{gather*}
Abbiamo dunque mostrato che $\frac1{f(\theta_1,\theta_3)}\pp{\theta_3}f(\theta_1,\theta_3)$ non dipende da $\theta_1$, cio\`e
\begin{gather*}
\pp{\theta_3}{}(\log(f(\theta_1,\theta_3)))=A(\theta_3)\\
\log(f(\theta_1,\theta_3))=B(\theta_3)+C(\theta_1),
\end{gather*}
dove $B(\theta_3)$ \`e una primitiva di $A(\theta_3)$.\\
Notiamo ora che $f(\theta,\theta)=1$ in quanto tanto calore viene rilasciato quanto viene assorbito se le sorgenti sono alla stessa temperatura.\\ 
Segue che $\log(f(\theta,\theta))=0$, cio\`e $B(\theta)=-C(\theta)$.
\[\log(f(\theta_1,\theta_3))=B(\theta_3)-B(\theta_1)\implies f(\theta_1,\theta_3)\pasgnlmath={g(\theta)=e^{B(\theta)}}\frac{g(\theta_3)}{g(\theta_1)}.\]
Possiamo dunque definire la \textbf{temperatura assoluta} come
\[T=g(\theta).\]
Tutto ci\`o che abbiamo detto fin'ora in termini della temperatura definita tramite gas continua ad essere valido per la temperatura assoluta.


