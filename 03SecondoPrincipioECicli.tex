\chapter{Secondo principio e cicli}

\section{Processo ciclico}
\begin{definition}[Processo ciclico]
Un processo \`e \textbf{ciclico} se lo stato iniziale e finale sono lo stesso. Se qualcosa realizza un processo ciclico \`e detto \textbf{motore}.
\end{definition}

\begin{remark}[Diagramma di una macchina a due sorgenti]
Spesso torna comodo fare diagrammi come in figura

[DIAGRAMMA]
\end{remark}

\begin{remark}
Per un processo ciclico, $\Delta U=0$, dunque $Q=-W$.
\[-W=Q=Q_H+Q_L\]
dove $Q_H$ \`e il calore che il sistema acquista da una sorgente calda e $Q_L$ \`e il calore che acquista da una sorgente fredda\footnote{$Q_L$ \`e negativo}. Notiamo che
\[\abs W=\abs{Q_H}-\abs{Q_L}.\]
\end{remark}

\begin{definition}[Efficienza]
L'\textbf{efficienza} di un processo ciclico \`e data da\footnote{Intuitivamente l'efficienza \`e una misura di quanto lavoro riesco a realizzare in proporzione a quanto calore abbiamo dovuto inserire nel sistema. L'altra forma ci dice che l'efficienza \`e una coversione perfetta eccetto per il calore che viene disperso senza diventare lavoro ($Q_L$).}
\[\eta=\frac{\abs W}{\abs{Q_H}}=1-\frac{\abs{Q_L}}{\abs{Q_H}}.\]
\end{definition}

\begin{definition}[Frigorifero e coefficiente di prestazione]
Un \textbf{frigorifero} \`e un motore che trasferisce calore da una sorgente fredda ad una calda. Il suo \textbf{coefficiente di prestazione} \`e dato da
\[COP=\frac{\abs{Q_L}}{\abs{W}}=\frac{1-\eta}\eta.\]
\end{definition}

\begin{definition}[Pompa di calore]
Una \textbf{pompa di calore} \`e una macchina volta a trasformare lavoro in calore verso la sorgente calda. La sua efficienza \`e quindi l'inversa di quella di un motore standard:
\[\frac{\abs{Q_H}}{\abs{W}}=\frac1\eta.\]
\end{definition}


\section{Enunciato del secondo principio}
\begin{fact}[Secondo principio della termodinamica, formulazione di Kelvin]
\textbf{Non esiste un processo che traformi \ul{interamente} calore in lavoro.}
\end{fact}

\begin{fact}[Secondo principio della termodinamica, formulazione di Clausius]
\textbf{Non esiste un processo il cui \ul{unico risulato} sia trasferire calore da una sorgente pi\`u fredda ad una pi\`u calda.}
\end{fact}

\begin{proposition}
Le due formulazioni del secondo principio sono equivalenti.
\end{proposition}
\begin{proof}
Mostrimo che le loro negazioni sono equivalenti:
\setlength{\leftmargini}{0cm}
\begin{itemize}
\item[$\boxed{\neg K\implies \neg C}$] Consideriamo il diagramma

[DIAGRAMMA]

Notiamo che $\abs{Q'}+\abs{W}>\max{\cpa{\abs{Q'},\abs{W}}}=\max{\cpa{\abs{Q'},\abs{Q}}}$. Considerando ora il sistema di due macchine come un insieme troviamo una macchina che trasferisce un calore $\abs{Q'}$ dala sorgente fredda alla sorgente calda, negando Clausius.
\item[$\boxed{\neg C\implies \neg K}$] Procediamo analogamente a prima

[DIAGRAMMA]

e leggendo questo diagramma come un insieme la macchina avrebbe preso del calore $\abs Q-\abs{Q'}$ dalla sorgente $T_L$ e lo ha trasformato interamente in lavoro, negando Kelvin.
\end{itemize}
\setlength{\leftmargini}{0.5cm}
\end{proof}

\section{Ciclo di Carnot}
Per cercare di sprecare meno calore possibile vogliamo una piccola differenza di temperatura

[IMMAGINE]

\noindent
Il ciclo \`e composto da isoterma a temperatura $T_H$, espansione adiabatica da $T_H$ a $T_L$, isoterma a temperatura $T_L$ e poi una compressione adiabatica per tornare a $T_H$.

\begin{remark}
Gli unici scambi di calore avvengono lungo l'isoterma, che ha senso solo a regime quasistatico (dato che il calore \`e uno scambio derivante da una differenza di energia).
\end{remark}

\begin{fact}
Il ciclo di Carnot \`e l'unico ciclo che effettua scambi in modo reversibile tra due sorgenti.
\end{fact}




\begin{proposition}[Efficienza del ciclo di Carnot]\label{EfficienzaCicloCarnot}
L'efficienza di un ciclo di Carnot tra le temperature $T_H$ e $T_L$ \`e data da
\[\eta=1-\frac{T_L}{T_H}.\]
\end{proposition}
\begin{proof}
Calcoliamo che quantit\`a coinvolte:
\[\abs{Q_H}=Q_{AB}\pasgnl={isoterma.}-W_{AB}=\int_A^BpdV=nRT_H\log\pa{\frac{V_B}{V_A}}>0\]
\[\abs{Q_L}=-Q_{CD}\pasgnl={isoterma.}W_{CD}=-\int_C^DpdV=nRT_L\log\pa{\frac{V_C}{V_D}},\]
\[\eta=1-\frac{\abs{Q_L}}{\abs{Q_H}}=1-\frac{T_L\log(V_C/V_D)}{T_H\log(V_B/V_A)}=1-\frac{T_L}{T_H},\]
dove nell'ultimo conto abbiamo usato le equazioni per le adiabatiche:
\[\pa{\frac{V_B}{V_C}}^{\gamma-1}=\frac{T_L}{T_H},\quad \pa{\frac{V_D}{V_A}}^{\gamma-1}=\frac{T_H}{T_L}\implies \frac{V_B}{V_A}=\frac{V_C}{V_D}.\]
\end{proof}

\begin{theorem}[di Carnot]\label{TeoremaDiCarnot}
Un ciclo reversibile \`e il pi\`u efficiente che lavori tra due sorgenti $T_H$ e $T_L$.
\end{theorem}
\begin{proof}
Consideriamo due cicli $S$ ed $S'$, uno reversibile e uno non reversibile. Per il primo principio $-W=\abs{Q_H}-\abs{Q_L}$ e $-W'=\abs{Q_H'}-\abs{Q_L'}$.

Con precisione arbitraria, siano $N$ ed $N'$ interi positivi tali che
\[\frac{\abs{Q_H}}{\abs{Q_H'}}\approx \frac{N'}{N}.\]
Facendo fare $N'$ cicli a $S'$ ed $N$ cicli reversibili \textit{al contrario}\footnote{qu\`i usiamo la reversibilit\`a. Se prima il sistema trasformava calore in lavoro con qualche perdita di calore ora il sistema riceve lavoro e un po' di calore per fornire calore alla sorgente calda} a $S$ troviamo
\[-W_{tot}=N'(-W')-N(-W)=N'(\abs{Q_H'}-\abs{Q_L'})-N(\abs{Q_H}-\abs{Q_L})\]
\[Q_{H,tot}=N'\abs{Q_H'}-N\abs{Q_H}\]
\[-Q_{L,tot}=N'\abs{Q_L'}-N\abs{Q_L}.\]
Per il primo principio, facendo lavorare in parallelo le due macchine
\[-W_{tot}=Q_{H,tot}+Q_{L,tot}.\]
Grazie alla scelta di $N$ ed $N'$ possiamo approssimare $Q_{H,tot}\approx 0$, quindi
\[-W_{tot}\approx Q_{L,tot}.\]
Per la formulazione di Kelvin del secondo principio si ha che $-W_{tot}\leq0$, quindi $Q_{L,tot}\leq 0$, cio\`e
\[N'\abs{Q_L'}-N\abs{Q_L}\geq 0\coimplies \frac{N'}N\geq \frac{\abs{Q_L}}{\abs{Q_L'}}.\]
Passando al limite negli $N$ e $N'$ si ha che
\[\frac{\abs{Q_L}}{\abs{Q_H}}\leq \frac{\abs{Q_L'}}{\abs{Q_H'}}\implies \eta=1-\frac{\abs{Q_L}}{\abs{Q_H}}\geq 1-\frac{\abs{Q_L'}}{\abs{Q_H'}}=\eta'.\]
\end{proof}

\begin{corollary}[I cicli reversibili hanno la stessa efficienza]\label{CicliReversibiliHannoLaStessaEfficienza}
Tutti i cicli reversibili hanno la stessa efficienza.
\end{corollary}
\begin{proof}
Applicando il teorema abbiamo le due disuguaglianze scambiando i ruoli tra i due cicli.
\end{proof}

\begin{remark}[Efficienza massima]
Poich\'e il ciclo di Carnot \`e reversibile, per il teorema di Carnot (\ref{TeoremaDiCarnot}) il valore
\[1-\frac{T_L}{T_H}\]
\`e la massima efficienza possibile per un qualsiasi ciclo.
\end{remark}

\begin{remark}[Coefficiente di prestazione massimo]
Per quanto detto il coefficente di prestazione massimo \`e
\[\frac{1-\eta_{Carnot}}{\eta_{Carnot}}= \frac{T_L}{T_H-T_L}.\]
Se $T_L=4^\circ\mathrm{C}$ e $T_H=20^\circ\mathrm{C}$ (caso tipico del frigorifero casalingo) allora $COP_{max}\approx 17.3$. Tipicamente $COP\approx 4$.
\end{remark}
\begin{remark}[Massima efficienza di una pompa di calore]
Per una pompa di calore, la massima efficienza \`e data da
\[\frac{T_H}{T_H-T_L}.\]
\end{remark}

\subsection{Definizione della temperatura tramite ciclo di Carnot}

[COME FATTA A LEZIONE NON HA SENSO PERCH\'E IL CALORE NON \`E STATO DEFINITO]

Appurata la validit\`a del teorema di Carnot (\ref{TeoremaDiCarnot}) possiamo ridefinire la temperatura in modo universale come segue:

Per il teorema di Carnot esiste $f$ tale che dopo un ciclo reversibile
\[\frac{\abs{Q_H}}{\abs{Q_L}}=f(\theta_L,\theta_H).\]

Collegando due tali processi facendo s\`i che il calore rilasciato dal primo sia quello assorbito dal secondo ricaviamo le equazioni
\[\frac{\abs{Q_3}}{\abs{Q_2}}=f(\theta_2,\theta_3),\quad\frac{\abs{Q_2}}{\abs{Q_1}}=f(\theta_1,\theta_2),\quad \frac{\abs{Q_3}}{\abs{Q_1}}=f(\theta_1,\theta_3),\]
dove $\theta_1\leq \theta_2\leq \theta3$.\\
Segue dunque l'identit\`a
\[f(\theta_1,\theta_2)=\frac{f(\theta_1,\theta_3)}{f(\theta_2,\theta_3)}.\]
Derivando rispetto a $\theta_3$ ricaviamo
\begin{gather*}
0=\frac1{f(\theta_2,\theta_3)}\pp{\theta_3}{f}(\theta_1,\theta_3)-\frac{f(\theta_1,\theta_3)}{(f(\theta_2,\theta_3))^2}\pp{\theta_3}f(\theta_2,\theta_3)\\
\frac1{f(\theta_1,\theta_3)}\pp{\theta_3}{f}(\theta_1,\theta_3)=\frac1{f(\theta_2,\theta_3)}\pp{\theta_3}{f}(\theta_2,\theta_3).
\end{gather*}
Abbiamo dunque mostrato che $\frac1{f(\theta_1,\theta_3)}\pp{\theta_3}f(\theta_1,\theta_3)$ non dipende da $\theta_1$, cio\`e
\begin{gather*}
\pp{\theta_3}{}(\log(f(\theta_1,\theta_3)))=A(\theta_3)\\
\log(f(\theta_1,\theta_3))=B(\theta_3)+C(\theta_1),
\end{gather*}
dove $B(\theta_3)$ \`e una primitiva di $A(\theta_3)$.\\
Notiamo ora che $f(\theta,\theta)=1$ in quanto tanto calore viene rilasciato quanto viene assorbito. Segue che $\log(f(\theta,\theta))=0$, cio\`e $B(\theta)=-C(\theta)$.
\[\log(f(\theta_1,\theta_3))=B(\theta_3)-B(\theta_1)\implies f(\theta_1,\theta_3)\pasgnlmath={g(\theta)=e^{B(\theta)}}\frac{g(\theta_3)}{g(\theta_1)}.\]
Possiamo dunque definire la \textbf{temperatura assoluta} come
\[T=g(\theta).\]