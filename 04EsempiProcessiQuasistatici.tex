\chapter{Esempi principali di processi quasistatici per gas}

\section{Coefficiente di espansione volumetrica e compressibilit\`a isoterma}
\begin{definition}[Coefficiente di espansione volumetrica]
Definiamo il \textbf{coefficiente di espansione volumetrica} come
\[\al=\frac1V\ppb TVp=-\frac mV\frac1{\rho^2}\ppb T\rho p=-\frac1\rho\ppb T\rho p.\]
L'unit\`a di misura \`e $[\al]=\mathrm{K}\ii$.
\end{definition}


\begin{definition}[Compressibilit\`a isoterma]
Definiamo la \textbf{compressibilit\`a isoterma} come
\[\beta_T=-\frac1V\ppb pVT.\]
L'unit\`a di misura \`e $[\beta_T]=\mathrm{Pa}\ii$.\\
L'inversa $k_T=1/\beta_T$ \`e detta \textbf{modulo di compressibilit\`a isoterma}.
\end{definition}

\noindent Riportiamo alcuni valori di $\al$ e $\beta_T$ per dare una intuizione sui valori tipici\footnote{il Sitall \`e materiale fatto apposta per avere coefficiente di espansione volumetrica piccolo}
\begin{center}
\begin{tabular}[ht]{|c|c|c|}
\hline 
Materiale&$\al\ [\mathrm{K}\ii]$&$\beta_T\ [\mathrm{Pa}\ii]$\\\hline
Acqua&$0.2\cdot 10^{-3}$&$4.6\cdot 10^{-10}$\\
Diamante&$3\cdot 10^{-6}$&?\\
Sitall&$\leq 10^{-7}$&?\\
Sabbia&?&$\sim10^{-8}$\\
Mercurio&$1.8\cdot 10^{-4}$&$4\cdot10^{-11}$\\
Rame&?&$7.2\cdot10^{-12}$\\
\hline
\end{tabular}
\end{center}


\begin{remark}
Non \`e necessario battezzare $\displaystyle\ppb TpV$ in quanto per la propriet\`a ciclica (\ref{ProprietaDerivateParziali})
\[\ppb TpV=-\ppb VpT\ppb TVp=\frac\al{\beta_T}.\]
\end{remark}

\begin{remark}[Relazione differenziale tra $\al$ e $\beta_T$]
Per il teorema di Schwarz si ha che
\[\pp{p\del T}{^2V}=\ppb p\al T=-\ppb T{\beta_T}p.\]
\end{remark}

\begin{proposition}[Differenziale della pressione]\label{DifferenzialePressione}
Si ha che
\[dp=\frac\al{\beta_T}dT-\frac1{\beta_T V}dV.\]
\end{proposition}
\begin{proof}
Osserviamo che
\[\ppb TpV\pasgnl={(\ref{ProprietaDerivateParziali})}-\ppb VpT \ppb TVp=\frac \al{\beta_T},\]
dunque ricaviamo
\[dp=\ppb TpVdT+\ppb VpT=\frac\al{\beta_T}dT-\frac1{\beta_T V}dV.\]
\end{proof}
\begin{corollary}
In una trasformazione isocora $\Delta p=\frac\al{\beta_T}\Delta T$.
\end{corollary}

\begin{remark}[Differenziale logaritmico nel volume]\label{DifferenzialeLogaritmicoNelVolume}
Spesso torner\`a comodo ricordare il seguente sviluppo differenziale
\[d\log V=\frac1VdV=\al dT-\beta_T dp\]
\end{remark}
\begin{proof}
Segue calcolando:
\[\frac1VdV=\frac1V\pa{\ppb TVp dT+\ppb pVTdp}=\al dT-\beta_T dp\]
\end{proof}

\section{Lavoro per gas}
Immagino di comprimere un sistema come in figura 

[FIGURA]

\noindent
Se spingiamo molto lentamente possiamo con buona approssimazione supporre che il processo sia quasistatico, dunque $F=pS$. Segue che
\[\boxed{\delta W=Fdx=pSdx}\]
Se il sistema in questione \`e un gas ideale allora
\[\delta W=p(-dV)=-pdV\]
Il lavoro totale per passare da uno stato $A$ ad uno stato $B$ diventa
\[W=-\int_{A}^{B} p(V,T)dV,\]
ma $p$ come cambia al variare di $V$? Dipende dal tipo di processo.

[QUALCHE GRAFICO]

\noindent
Questo mostra in particolare che il lavoro non \`e una funzione di stato.



\section{Processi politropici}
Possiamo generalizzare i quattro tipi di processi citati nella seguente classe:
\begin{definition}[Processo politropico]
Un processo \`e \textbf{politropico} se la capacit\`a termica \`e costante.
\end{definition}

\begin{proposition}[Curve per processi politropici]\label{CurveProcessiPolitropici}
Considerando un processo politropicorelativo ad un gas ideale e definiamo
\[\delta=\frac{C_p-C}{C_V-C},\]
allora seguendo questo processo si ha che $pV^\delta=cost.$.
\end{proposition}
\begin{proof}
Poich\'e $\delta Q=CdT=C_VdT+pdV=C_pdT-Vdp$ ricaviamo che
\[-\frac V{C-C_p}dp=dT=\frac p{C-C_V}dV,\]
da cui
\[-\frac Vp\dd Vp=\frac{C_p-C}{C_V-C}=\delta.\]
Questa espressione restituisce una equazione differenziale
\[-\frac{dp}p=\delta\frac{dV}V,\]
la cui soluzioni hanno la forma voluta.
\end{proof}

\noindent
Possiamo interpretare processi isocori, isobari, isotermi e adiabatici come processi politropici:
\begin{center}
\begin{tabular}[ht]{|c||c|c|c|c|}
\hline
Processo & Isocoro & Isobaro & Isotermo & Adiabatico\\\hline&&&&\\
$\delta$ & $\infty$ & $0$ & $1$ & $\displaystyle\gamma=\frac{c_p}{c_V}$\\ &&&&\\\hline
\end{tabular}
\end{center}


