\chapter{Meccanica Quantistica}

\section{Operatori autoaggiunti}
\begin{definition}[Stato]
Uno \textbf{stato} \`e una funzione d'onda che per ogni $t$ \`e un vettore unitario nello spazio di Hilbert dato da funzioni $L^2$ con prodotto interno
\[\ps{\phi,\psi}=\int\phi^\ast(x)\psi(x)dx.\]
\end{definition}
\begin{remark}
Il prodotto \`e hermitiano, cio\`e $\ps{\phi,\psi}=\ps{\psi,\phi}^\ast$.
\end{remark}
\begin{remark}
La condizione che uno stato abbia norma unitaria corrisponde al fatto che $\abs{\psi}^2$ \`e una densit\`a di probabilit\`a, infatti
\[\ps{\psi,\psi}=\int \abs{\psi}^2dx.\]
\end{remark}

\begin{definition}[Operatore autoaggiunto]
Un \textbf{operatore} \`e una funzione lineare nello spazio di Hilbert descritto sopra.\\
Se $A$ \`e un operatore, il suo \textbf{aggiunto} \`e l'operatore $A^\dag$ definito da
\[\ps{\phi,A\psi}=\ps{A^\dag\phi,\psi}.\]
Un operatore \`e \textbf{autoaggiunto} (o \textbf{hermitiano}) se $A=A^\dag$.
\end{definition}

\noindent In questo formalismo, gli operatori autoaggiunti corrispondono a \textbf{grandezze fisiche} e i loro autovalori ai possibili esiti di una misura della data grandezza.\\
Un'autofunzione corrisponde dunque ad uno stato per cui la misura della grandezza per cui \`e un autostato ha come esito certo l'autovalore relativo.

\begin{proposition}[Autovalori per autoaggiunti sono reali]
Se $\la$ \`e un autovalore per $A$ autoaggiunto allora $\la\in \R$.
\end{proposition}
\begin{proof}
Sia $\phi_A^{(\la)}$ un'autofunzione per $A$ relativa a $\la$. Osserviamo che
\begin{align*}
\la\ps{\phi_A^{(\la)},\phi^{(\la)}_A}=&\ps{\phi_A^{(\la)},A\phi_A^{(\la)}}=\ps{A^\dag \phi_A^{(\la)},\phi_A^{(\la)}}\pasgnlmath={A=A^\dag}\\
=&\ps{A\phi_A^{(\la)},\phi_A^{(\la)}}=\ps{\phi_A^{(\la)},A\phi_A^{(\la)}}^\ast=\\
=&\la^\ast\ps{\phi_A^{(\la)},\phi_A^{(\la)}},
\end{align*}
cio\`e $\la=\la^\ast$ e quindi $\la\in\R$.
\end{proof}



\begin{proposition}
Se $\la_1\neq \la_2$ autovalori per $A$ autoaggiunto allora $\ps{\phi_A^{(\la_1)},\phi_A^{(\la_2)}}=0$.
\end{proposition}
\begin{proof}
Segue dalla seguente catena di uguaglianze
\begin{align*}
\la_1\ps{\phi_A^{(\la_2)},\phi_A^{(\la_1)}}=&{\phi_A^{(\la_2)},A\phi_A^{(\la_1)}}=\ps{A\phi_A^{(\la_2)},\phi_A^{(\la_1)}}=\\
=&\ps{\phi_A^{(\la_1)},A\phi_A^{(\la_2)}}^\ast=\la_2^\ast \ps{\phi_A^{(\la_1)},\phi_A^{(\la_2)}}^\ast\pasgnl={$\la_2$ reale}\\
=&\la_2\ps{\phi_A^{(\la_2)},\phi_A^{(\la_1)}}.
\end{align*}
\end{proof}

\noindent Se $A$ autoaggiunto ammette una base di autofunzioni normalizzate $\phi_A^{(\la)}$ allora
\[\psi(x)=\int c(\la)\phi_A^{(\la)}(x)d\la,\quad\text{dove }c(\la)=\ps{\phi_A^{(\la)},\psi}.\]

\begin{definition}[Media di un operatore]
Fissiato uno stato $\psi$, definiamo il \textbf{valore medio} dell'operatore autoaggiunto $A$ come
\begin{align*}
\ps{A}=\ps{\psi,A\psi}=&\int \psi^\ast A\psi dx=\\
=&\iiint \la c(\mu)^\ast c(\la) (\phi_A^{(\mu)})^\ast\phi_A^{(\la)} d\mu d\la dx=\\
=&\iint  \la c(\mu)^\ast c(\la) \ps{\phi_A^{(\mu)},\phi_A^{(\la)}} d\mu d\la=\\
=&\int \la\abs{c(\la)}^2d\la.
\end{align*}
In particolare $\abs{c(\la)}^2=\abs{\ps{\phi_A^{(\la)},\psi}}^2$ \`e la \textbf{densit\`a di probabilit\`a di misurare $\la$}.
\end{definition}

\begin{definition}[Varianza di un operatore]
Fissiato uno stato $\psi$, definiamo la \textbf{varianza} dell'operatore autoaggiunto $A$ come
\[\sigma^2_A=\ps{(A-\ps A)^2}.\]
\end{definition}

\noindent Ma quando effettivamente misuro cosa succede alla funzione d'onda?\\
Avviene il \textbf{collasso della funzione d'onda}, cio\`e dopo la misura, se abbiamo misurato il valore $\la$ allora la funzione d'onda DIVENTA\footnote{ci sono alcuni problemi quando abbiamo un autospazio di dimensione maggiore ad 1 ma per il momento ignoriamo questi problemi.} $\phi_A^{(\la)}$.

\section{Operatori principali}

\subsection{Operatore Hamiltoniano}
\begin{definition}[Operatore Hamiltoniano]
Definiamo l'\textbf{operatore Hamiltoniano} come
\[H=-\frac{\hbar^2}{2m}\pps[2]x{}+V(x).\]
\end{definition}
\begin{remark}
L'equazione di Schr\"odinger indipendente dal tempo si scrive $H\psi=E\psi$.
\end{remark}

\begin{remark}
L'equazione di Schr\"odinger si scrive
\[i\hbar\dd t\psi=H\psi.\]
\end{remark}

\noindent Per quanto detto nel capitolo precedente, le autofunzioni di $H$ sono gli stati stazionari
\[\psi(x,t)=e^{-iEt/\hbar} f(x)\]
Sia $\psi^{(E)}$ un tale stato, allora
\[\ps{\psi^{(E)},H\psi^{(E)}}=\ps{\psi^{(E)},E\psi^{(E)}}=E\ps{\psi^{(E)},\psi^{(E)}}=E.\]
Similmente
\[\ps{\psi^{(E)},H^2\psi^{(E)}}=E^2,\]
quindi la varianza per l'autofunzione \`e
\[\sigma_H^2=\ps{(H-\ps H)^2}=\ps{H^2}-2\ps H^2+\ps H^2=E^2-2E^2+E^2=0.\]

\begin{remark}[Conservazione dell'energia]
Consideriamo $\psi=\sum_n c_n\psi_n$ con $\psi_n=\psi^{(E_n)}=f_n(x)e^{-iE_nt/\hbar}$. Osserviamo che 
\[\ps H=\sum_m E_me^{i(E_m-E_m) t/\hbar}\abs{c_m}^2=\sum_m E_m\abs{c_m}^2,\]
in particolare questa media non dipende dal tempo. Questa \`e la versione quantistica della \textbf{conservazione dell'energia}.
\end{remark}


\subsection{Operatore Posizione}

\begin{definition}[Operatore Posizione]
Definiamo l'\textbf{operatore posizione} $X$ come
\[X\psi(x,t)=x\psi(x,t).\]
\end{definition}

\begin{remark}
La posizione media \`e data da
\[\ps{X}=\ps{\psi,X\psi}=\int x\abs{\psi(x,t)}^2dx.\]
\end{remark}

\noindent Chi sono le autofunzioni?
\[X\phi_X^{(\la)}(x)=
\begin{cases}
\la \phi_X^{(\la)}(x) &\text{perch\'e autofunzione}\\
x\phi_X^{(\la)}(x) &\text{per definizione}
\end{cases}\]
Quindi serve una funzione che in un qualche modo \`e nulla ovunque eccetto in $\la$ ma tale che integrando il modulo viene 1. Queste condizioni NON sono verificate da una funzione ma spesso in fisica torna comodo considerare un oggetto con propriet\`a simili

\begin{definition}[Delta di Dirac]
La \textbf{delta di Dirac} \`e la misura definita da
\[\delta(x-\la)dx:\funcDef{\powerset(\R)}{\R_{\geq0}}{E}{\chi_E(\la)},\]
dove $\chi_E$ \`e la funzione caratteristica di $E$.
\end{definition}
\begin{remark}
\`E possibile verificare che
\[\int f(x)\delta(x-\la)dx=f(\la).\]
\end{remark}
\noindent Intuitivamente pu\`o essere utile immaginarsi la delt\`a di Dirac come una funzione
\[\delta(x)=\begin{cases}
0 &x\neq 0\\
\infty &x=0
\end{cases}\]

\subsection{Operatore Quantit\`a di moto}
\noindent Avendo una posizione media proviamo a trovare una velocit\`a\footnote{assumiamo potenziale nullo}:
\begin{align*}
\dd t{}\ps{X}=&\dd t{}\ps{\psi,X\psi}=\int x\dd t{}\pa{\psi^\ast\psi}dx=\\
=&\frac{i\hbar}{2m}\int x\pa{-\pps[2] x{\psi^\ast}\psi+\pps[2] x{\psi}\psi^\ast}dx=\\
=&\frac{i\hbar}{2m}\int x\pp x{}\pa{-\pp x{\psi^\ast}\psi+\pp x{\psi}\psi^\ast}dx=\\
=&0-\frac{i\hbar}{2m}\int-\pp x{\psi^\ast}\psi+\pp x{\psi}\psi^\ast dx=\\
=&-\frac{i\hbar}{m}\int\pp x{\psi}\psi^\ast dx.
\end{align*}
\begin{definition}[Operatore quantit\`a di moto]
Definiamo l'\textbf{operatore quantit\`a di moto} come
\[P_x=-i\hbar \pp x{}.\]
\end{definition}
\begin{remark}
Se il potenziale \`e nullo allora 
\[\ps{P_x}=-i\hbar\int \psi^\ast\pp x\psi dx=m\dd t{}\ps{X}.\]
\end{remark}
\begin{remark}
$P_x$ \`e effettivamente un operatore autoaggiunto.
\end{remark}
\begin{proof}
Segue calcolando:
\begin{align*}
\ps{\psi,-i\hbar\pp x\psi}=&\int-i\hbar\psi^\ast\pp x\psi dx=\int-i\hbar\psi^\ast\pp x\psi dx=\\
=&\cancelto{0}{\rbar{-i\hbar\psi^\ast\psi}^\infty_{-\infty}}-\int -i\hbar\pp x{\psi^\ast}\psi dx=\\
=&-\int -\pa{-i\hbar\pp x{\psi}}^\ast\psi dx=\\
=&\ps{-i\hbar\pp x\psi,\psi}.
\end{align*}
\end{proof}

\begin{remark}
Le autofunzioni dell'operatore impulso sono
\[\phi^{(k)}_{P_x}=Ae^{ikx},\]
dove \underline{l'autovalore \`e $\hbar k$}. 
Queste funzioni non sono normalizzabili ma si possono comunque usare come una base.
\end{remark}
\noindent Se $\psi=\int c(k)e^{ikx}dx$ allora
\[P_x\psi=\int c(k)\hbar k e^{ikx}dx,\]
da cui
\begin{align*}
\ps{P_x}=&\int\pa{\int c(k')^\ast e^{ik'x}dk'}^\ast\pa{\int \hbar kc(k)^\ast e^{ikx}dk}dx=\\
=&\iint \int c(k')^\ast c(k) \hbar k e^{i(k-k')x}dx\ dk'dk\pasgnl={intuitivamente}\\
=&\iint c(k')^\ast c(k) \hbar k\delta(k'-k)dk'dk=\\
=&\int \hbar k\abs{c(k)}^2dk.
\end{align*}

\section{Operatori incompatibili e Principio di indeterminazione}

\begin{remark}
Osserviamo che $X$ e $P_x$ non commutano, infatti
\begin{align*}
P_xX\psi=P_xx\psi=-i\hbar \psi -i\hbar x\pp x\psi=-i\hbar \psi+xP_x\psi=-i\hbar \psi+XP_x\psi
\end{align*}
In particolare il commutatore vale
\[[X,P_x]=XP_x-P_xX=i\hbar.\]
\end{remark}

\begin{definition}[Operatori incompatibili]
Due operatori autoaggiunti sono \textbf{incompatibili} se non ammettono autofunzioni comuni.
\end{definition}

\begin{proposition}
Se $A$ e $B$ sono operatori autoaggiunti allora 
\[[A,B]\neq 0\implies A,B \text{ incompatibili.}\]
\end{proposition}
\begin{proof}
Se per assurdo $\phi$ \`e un'autofunzione che hanno in comune ($A\phi=\al \phi$ e $B\phi=\beta\phi$) allora
\[0\neq [A,B]\phi=AB\phi-BA\phi=\beta A\phi-\al B\phi=(\al\beta-\al\beta)\phi=0.\]
\end{proof}

\begin{remark}
Poich\'e $[X,P_x]=i\hbar\neq 0$, posizione e quantit\`a di moto sono incompatibili.
\end{remark}



\begin{proposition}
Sia $A$ un operatore, allora
$\dd t{}\ps{A}=\frac1{i\hbar}\ps{\psi,[A,H]\psi}$.
\end{proposition}
\begin{proof}
\`E il seguente conto
\begin{align*}
\dd t{}\ps{\psi,A\psi}=&\ps{\dd t\psi,A\psi}+\ps{\psi,A\dd t\psi}\pasgnl={(\ref{EquazioneSchrodinger})}\\
=&-\frac1{i\hbar}\ps{H\psi,A\psi}+\frac1{i\hbar}\ps{\psi,AH\psi}=\\
=&\frac1{i\hbar}\ps{\psi,AH\psi}+\frac1{i\hbar}\ps{\psi,-HA\psi}=\\
=&\frac1{i\hbar}\ps{\psi,[A,H]\psi}.
\end{align*}
\end{proof}
\begin{corollary}
Se la media del valore di $A$ dipende dal tempo allora $A$ \`e incompatibile con $H$.
\end{corollary}


\begin{theorem}[Principio di indeterminazione di Heisenberg]
Se $A$ e $B$ sono operatori autoaggiunti allora
\[\sigma_A^2\sigma_B^2\geq\pa{\frac{\ps{[A,B]}}{2i}}^2.\]
\end{theorem}
\begin{proof}
Siano $\al=(A-\ps{A})\psi$ e $\beta=(B-\ps{B})\psi$, in modo tale che 
\[\sigma_A^2=\ps{(A-\ps{A})^2}=\ps{(A-\ps{A})\psi,(A-\ps{A})\psi}=\ps{\al,\al}\]
e similmente $\sigma_B^2=\ps{\beta,\beta}$.\medskip

\noindent
Per Cauchy-Schwarz
\[\sigma_A^2\sigma_B^2=\ps{\al,\al}\ps{\beta,\beta}\geq\abs{\ps{\al,\beta}}^2,\]
inoltre, notando che
\[\abs{z}^2\geq (\Imag(z))^2=\pa{\frac{z-z^\ast}{2i}}^2,\]
troviamo
\[\sigma_A^2\sigma_B^2\geq\abs{\ps{\al,\beta}}^2=\pa{\frac1{2i}(\ps{\al,\beta}-\ps{\beta,\al})}^2.\]
Sviluppiamo i prodotti scalari coinvolti:
\begin{align*}
\ps{\al,\beta}=&\ps{(A-\ps A)\psi,(B-\ps B)\psi}=\\
=&\ps{\psi,AB\psi}-\ps A\ps{\psi,B\psi}-\ps B\ps{\psi,A\psi}+\ps A\ps B\cancelto{1}{\ps{\psi,\psi}}\ \ \ =\\
=&\ps{AB}-\ps A\ps B.
\end{align*}
Segue che
\[\ps{\al,\beta}-\ps{\beta,\al}=\ps{[A,B]}\]
e quindi troviamo la tesi
\[\sigma_A^2\sigma_B^2\geq\pa{\frac{\ps{[A,B]}}{2i}}^2.\]
\end{proof}
\begin{corollary}
Nel caso $A=X$ e $B=P_x$ troviamo
\[\sigma_X\sigma_{P_x}\geq\frac\hbar2.\]
\end{corollary}