\chapter{Meccanica Quantistica}

\begin{definition}[Stato]
Uno \textbf{stato} \`e un vettore unitario in uno spazio di Hilbert di classe $L^2$.\\
Date due funzioni $\phi$ e $\psi$ definiamo
\[\ps{\phi,\psi}=\int\phi^\ast(x)\psi(x)dx.\]
\end{definition}
\begin{remark}
$\ps{\phi,\psi}=\ps{\psi,\phi}^\ast$. $\ps{\psi,\psi}=\int\abs{\psi}^2dx$. Se $\psi$ \`e una funzione d'onda allora $\ps{\psi,\psi}=1$. 
\end{remark}

Ricordiamo che se $A$ \`e un operatore, il suo aggiunto $A^\dag$ \`e l'operatore tale che
\[\ps{\phi,A\psi}=\ps{A^\dag\phi,\psi}.\]
Un operatore \`e \textbf{autoaggiunto} se $A=A^\dag$.

\begin{definition}[Grandezze fisiche]
Le grandezze fisiche corrispondono a operatori autoaggiunti (o hermitiani).\\
Per ogni operatore $A$, gli unici valori misurabili sono gli autovalori di $A$.\\
Un'autofunzione corrisponde ad uno stato dove la misura restituisce come valore l'autovalore corrispondente con certezza.
\end{definition}

\begin{remark}[Autovalori per autoaggiunti sono reali]
Se $\phi_A^{(\la)}$ \`e un'autofunzione per l'operatore $A$ relativa a $\la$ allora
\[\la\ps{\phi_A^{(\la)},\phi^{(\la)}_A}=\ps{\phi_A^{(\la)},A\phi_A^{(\la)}}=\ps{A^\dag \phi_A^{(\la)},\phi_A^{(\la)}}.\]
Se $A$ \`e autoaggiunto allora l'uguaglianza prosegue con
\[=\ps{A\phi_A^{(\la)},\phi_A^{(\la)}}=\ps{\phi_A^{(\la)},\phi_A^{(\la)}}^\ast=\la^\ast\ps{\phi_A^{(\la)},\phi_A^{(\la)}},\]
cio\`e $\la\in\R$.
\end{remark}

\begin{proposition}
Se $\la_1\neq \la_2$ autovalori per $A$ autoaggiunto allora $\ps{\phi_A^{(\la_1)},\phi_A^{(\la_2)}}=0$.
\end{proposition}
\begin{proof}
Segue dalla seguente catena di uguaglianze
\begin{align*}
\la_1\ps{\phi_A^{(\la_2)},\phi_A^{(\la_1)}}=&{\phi_A^{(\la_2)},A\phi_A^{(\la_1)}}=\ps{A\phi_A^{(\la_2)},\phi_A^{(\la_1)}}=\\
=&\ps{\phi_A^{(\la_1)},A\phi_A^{(\la_2)}}^\ast=\la_2^\ast \ps{\phi_A^{(\la_1)},\phi_A^{(\la_2)}}^\ast\pasgnl={$\la_2$ reale}\\
=&\la_2\ps{\phi_A^{(\la_2)},\phi_A^{(\la_1)}}.
\end{align*}
\end{proof}

\noindent Se $A$ autoaggiunto ammette una base di autofunzioni normalizzate $\phi_A^{(\la)}$ allora
\[\psi(x)=\int c(\la)\phi_A^{(\la)}(x)d\la,\quad\text{dove }c(\la)=\ps{\phi_A^{(\la)},\psi}.\]

\begin{remark}
Osserviamo che $\ps{\phi,A\psi}=\int \la \abs{c(\la)}^2$, che \`e il valore medio che $A$ assume per la funzione d'onda $\psi$. La funzione $\abs{c(\la)}^2$ \`e dunque la \textbf{densit\`a di probabilit\`a di misurare $\la$}.
\end{remark}

\noindent E quando effettivamente misuro cosa succede?\\
Avviene il \textbf{collasso della funzione d'onda}, cio\`e dopo la misura, se abbiamo misurato il valore $\la$ allora la funzione d'onda DIVENTA\footnote{ci sono alcuni problemi quando abbiamo un autospazio di dimensione maggiore ad 1 ma per il momento ignoriamo questi problemi.} $\phi_A^{(\la)}$.

\section{Operatori principali}

\subsection{Operatore Hamiltoniamno}
\begin{definition}[Operatore Hamiltoniano]
Definiamo l'\textbf{operatore Hamiltoniano} come
\[H=-\frac{\hbar^2}{2m}\pps[2]x{}+V(x).\]
\end{definition}
\begin{remark}
L'equazione di Schr\"odinger indipendente dal tempo si scrive $H\psi=E\psi$.
\end{remark}

\begin{remark}
L'equazione di Schr\"odinger si scrive
\[i\hbar\dd t\psi=H\psi.\]
\end{remark}

\noindent Le autofunzioni di $H$ sono gli stati stazionari
\[\psi(x,t)=e^{-iEt/\hbar} f(x)\]
Sia $\psi^{(E)}$ un tale stato
\[\ps{\psi^{(E)},H\psi^{(E)}}=\ps{\psi^{(E)},E\psi^{(E)}}=E\ps{\psi^{(E)},\psi^{(E)}}=E.\]
Osserviamo anche che per motivi analoghi
\[\ps{\psi^{(E)},H^2\psi^{(E)}}=E^2\]
dunque la varianza per l'autofunzione \`e
\[\sigma_H^2=\ps{H^2}-\ps{H}^2=E^2-E^2=0.\]
Consideriamo ora $\psi=\sum_n c_n\psi_n$ con $\psi_n=\psi^{(E_n)}=f_n(x)e^{-iE_nt/\hbar}$. Poich\'e deve valere la normalizzazione
\[1\pasgnl={indi. da $t$}\int\abs{\psi(x,0)}^2dx=\sum_n\sum_m c_n^\ast c_m \int f_n^\ast f_m dx=\sum_n\abs{c_n}^2.\]
Segue che
\begin{align*}
\ps{H}=&\ps{\psi,H\psi}=\sum_n\sum_m c_n^\ast c_m\int f_n^\ast E_m f_m e^{i(E_n-E_m) t/\hbar}dx=\\
=&\sum_n\sum_m c_n^\ast c_m E_me^{i(E_n-E_m) t/\hbar}\delta_{nm} dx=\\
=&\sum_m E_m\abs{c_m}^2,
\end{align*}
osserviamo in particolare che anche questa media non dipende dal tempo. Questa \`e la versione quantistica della \textbf{conservazione dell'energia}.

\subsection{Operatore Posizione}

\begin{definition}[Operatore Posizione]
Definiamo l'\textbf{operatore posizione} $X$ come
\[X\psi(x,t)=x\psi(x,t).\]
\end{definition}
Osserviamo che
\[\ps{X}=\ps{\psi,X\psi}=\int x\abs{\psi(x,t)}^2dx,\]
cio\`e la posizione media.\bigskip

\noindent Chi sono le autofunzioni?
\[X\phi_X^{(\la)}(x)=
\begin{cases}
\la \phi_X^{(\la)}(x) &\text{perch\'e autofunzione}\\
x\phi_X^{(\la)}(x) &\text{per definizione}
\end{cases}\]
Questa non \`e davvero una funzione, \`e un limite di distribuzioni e si chiama \textbf{delta di Dirac}. Moralmente \`e $\delta(x-\la)$ per 
\[\delta(x)=\begin{cases}
0 &x\neq 0\\
\infty &x=0
\end{cases}\]
La vera propriet\`a che definisce questo oggetto \`e che
\[\int f(x)\delta(x-\la)dx=f(\la).\]

\subsection{Operatore Quantit\`a di moto}
\noindent Avendo una posizione media proviamo a trovare una velocit\`a\footnote{assumiamo potenziale nullo}:
\begin{align*}
\dd t{}\ps{X}=&\dd t{}\ps{\psi,X\psi}=\int x\dd t{}\pa{\psi^\ast\psi}dx=\\
=&\frac{i\hbar}{2m}\int x\pa{-\pps[2] x{\psi^\ast}\psi+\pps[2] x{\psi}\psi^\ast}dx=\\
=&\frac{i\hbar}{2m}\int x\pp x{}\pa{-\pp x{\psi^\ast}\psi+\pp x{\psi}\psi^\ast}dx=\\
=&0-\frac{i\hbar}{2m}\int-\pp x{\psi^\ast}\psi+\pp x{\psi}\psi^\ast dx=\\
=&-\frac{i\hbar}{m}\int\pp x{\psi}\psi^\ast dx.
\end{align*}
\begin{definition}[Operatore quantit\`a di moto]
Definiamo l\textbf{operatore quantit\`a di moto} come
\[P_x=-i\hbar \pp x{}.\]
\end{definition}
\begin{remark}
$\ps{P_x}=-i\hbar\int \psi^\ast\pp x\psi dx=m\dd t{}\ps{X}$.
\end{remark}
\begin{remark}
$P_x$ \`e effettivamente un operatore autoaggiunto perch\'e
\[\ps{\psi,\pp x\psi}=\int\psi^\ast\pp x\psi dx=\rbar{\psi^\ast\psi}^\infty_{-\infty}-\int \pp x{\phi^\ast}\psi dx=-\ps{\pp x\phi,\psi}\]
e quel segno viene compensato dall'unit\`a immaginaria nella definizione di $P_x$.
\end{remark}

Le autofunzioni dell'operatore impulso sono
\[\phi^{(k)}_{P_x}=Ae^{ikx},\]
dove l'autovalore \`e $\hbar k$.\\
Queste funzioni non sono normalizzabili ma si possono comunque usare come una base.\\
Se $\psi=\int c(k)e^{ikx}dx$ allora
\[P_x\psi=\int c(k)\hbar k e^{ikx}dx,\]
da cui
\begin{align*}
\ps{P_x}=&\int\pa{\int c(k)^\ast e^{-ikx}dk}\pa{\int \hbar k'c(k')^\ast e^{-ik'x}dk'}dx=\\
=&\iint \int c(k)^\ast c(k') \hbar k' e^{i(k'-k)x}dx\ dkdk'\pasgnl={intuitivamente}\\
=&\iint\delta(k-k')c(k)^\ast c(k') \hbar k'dkdk'=\\
=&\int \hbar k\abs{c(k)}^2dk
\end{align*}

\section{Operatori incompatibili}

\begin{remark}
Osserviamo che $X$ e $P_x$ non commutano
\begin{align*}
P_xX\psi=P_xx\psi=-i\hbar \psi -i\hbar x\pp x\psi=-i\hbar \psi+xP_x\psi=-i\hbar \psi+XP_x\psi
\end{align*}
In particolare il commutatore vale
\[[X,P_x]=XP_x-P_xX=i\hbar.\]
\end{remark}

\begin{remark}
Poich\'e $[X,P_x]\neq 0$, non \`e possibile trovare un'autofunzione comune a $X$ e $P_x$, cio\`e posizione e quantit\`a di moto sono imcompatibili (non simultaneamente conoscibili).
\end{remark}
\begin{proof}
Se $\phi$ \`e un autofunzione di $X$ e $P_x$ (con posizione $x$ e quantit\`a di moto $\la$) allora
\[i\hbar \phi=[X,P_x]\phi=(x\la-\la x)\phi=0,\]
assurdo.
\end{proof}



\begin{proposition}
Sia $A$ un operatore, allora
$\dd t{}\ps{A}=\frac1{i\hbar}\ps{\psi,[A,H]\psi}$.
\end{proposition}
\begin{proof}
\`E il seguente conto
\begin{align*}
\dd t{}\ps{\psi,A\psi}=&\ps{\dd t\psi,A\psi}+\ps{\psi,A\dd t\psi}=\\
=&-\frac1{i\hbar}\ps{H\psi,A\psi}+\frac1{i\hbar}\ps{\psi,AH\psi}=\\
=&\frac1{i\hbar}\ps{\psi,[A,H]\psi}.
\end{align*}
\end{proof}
\begin{corollary}
Se la media del valore di $A$ dipende dal tempo allora $A$ \`e incompatibile con $H$.
\end{corollary}


\begin{theorem}[Principio di indeterminazione di Heisenberg]
Se $A$ e $B$ sono operatori autoaggiunti allora
\[\sigma_A^2\sigma_B^2\geq\pa{\frac{\ps{[A,B]}}{2i}}^2.\]
\end{theorem}
\begin{proof}
Siano $\al=(A-\ps{A})\psi$ e $\beta=(B-\ps{B})\psi$, in modo tale che 
\[\sigma_A^2=\ps{(A-\ps{A})^2}=\ps{(A-\ps{A})\psi,(A-\ps{A})\psi}\]
e similmente $\sigma_B^2=\ps{\beta,\beta}$.\\
Per Cauchy-Schwarz
\[\sigma_A^2\sigma_B^2=\ps{\al,\al}\ps{\beta,\beta}\geq\abs{\ps{\al,\beta}}^2,\]
inoltre, notando
\[\abs{z}^2\geq \Imag^2z=\pa{\frac{z-z^\ast}{2i}}^2,\]
se $z=\spa{\al,\beta}$ troviamo
\[\sigma_A^2\sigma_B^2\geq\pa{\frac1{2i}(\ps{\al,\beta}-\ps{\beta,\al})}^2.\]
Sviluppiamo i due prodotti scalari coinvolti:
\begin{align*}
\ps{\al,\beta}=&\ps{(A-\ps A)\psi,(B-\ps B)\psi}=\\
=&\ps{\psi,AB\psi}-\ps A\ps{\psi,B\psi}-\ps B\ps{\psi,A\psi}+\ps A\ps B\ps{\psi,\psi}=\\
=&\ps{AB}-\ps A\ps B.
\end{align*}
Segue che
\[\ps{\al,\beta}-\ps{\beta,\al}=\ps{[A,B]}\]
e quindi
\[\sigma_A^2\sigma_B^2\geq\pa{\frac{\ps{[A,B]}}{2i}}^2.\]
\end{proof}
\begin{corollary}
Nel caso $A=X$ e $B=P_x$ troviamo
\[\sigma_X\sigma_{P_x}\geq\frac\hbar2.\]
\end{corollary}