\part{Appendice}
\chapter{Richiami matematici}
\section{Derivate parziali e Jacobiane}
Da una relazione $f(x,y,z)=0$ possiamo ricavare $x=x(y,z)$ e $y=y(x,z)$.\\
Possiamo dunque sviluppare i differenziali
\begin{align*}
dx=&\ppb yxzdy+\ppb zxydz\\
dy=&\ppb xyzdx+\ppb zyxdz.
\end{align*}

\begin{proposition}[Propriet\`a delle derivate parziali]\label{ProprietaDerivateParziali}
Valgono le seguenti propriet\`a, dette \textbf{dell'inversa} e \textbf{ciclicit\`a} rispettivamente:
\[\ppb yxz=\pa{\ppb xyz}\ii,\qquad \ppb yxz\ppb zyx\ppb xzy=-1.\]
\end{proposition}
\begin{proof}
Considerando le espressioni date sopra e sostituiendo $dy$ dentro lo sviluppo di $dx$ ricaviamo l'equazione
\[\pa{1-\ppb yxz\ppb xyz}dx=\pa{\ppb yxz\ppb zyx+\ppb zxy}dz.\]
Se fissiamo $z$ il membro di sinistra non cambia, mentre quello di destra risulta nullo ($dz=0$). Poich\'e questo \`e vero anche per $dx\neq 0$ necessariamente ricaviamo
\[1=\ppb yxz\ppb xyz\]
che \`e la propriet\`a dell'inversa.\medskip

\noindent Avendo mostrato questo ricaviamo che il membro di sinistra \`e sempre nullo, anche per $dz\neq 0$, quindi segue l'equazione
\[\ppb yxz\ppb zyx+\ppb zxy=0,\]
la quale corrisponde alla propriet\`a di ciclicit\`a.
\end{proof}

\noindent Consideriamo le seguenti relazioni
\[\begin{cases}
x=x(u,v)\\
y=y(u,v)
\end{cases}.\]
Poniamo
\[\pp{(u,v)}{(x,y)}=det{\mat{\displaystyle\pp ux &\displaystyle\pp vx\\\\ \displaystyle\pp uy & \displaystyle\pp vy}}.\]
\begin{remark}[Jacobiane notevoli]\label{JacobianeNotevoli}
Si ha che
\[\pp{(x,y)}{(x,y)}=1,\quad \pp{(u,v)}{(x,x)}=0,\quad \pp{(u,v)}{(x,y)}=-\pp{(u,v)}{(y,x)}=\pp{(u,v)}{(-x,y)}.\]
Inoltre
\[\pp{(u,y)}{(x,y)}=\ppb uxy,\quad \pp{(u,v)}{(x,u)}=\pp{(r,s)}{(x,u)}\pp{(u,v)}{(r,s)},\quad \pp{(u,v)}{(x,y)}=\pa{\pp{(x,y)}{(u,v)}}\ii.\]
\end{remark}

\section{Differenziali esatti}
Ricordiamo che una forma $\sum A_i dx_i$ \`e chiusa quando per ogni coppia $i,j$
\[\pp {x_i}{A_j}=\pp{x_j}{A_i}.\]
Se il dominio \`e semplicemente connesso allora questa condizione caratterizza anche le forme esatte.

\begin{proposition}[Esattezza tramite Pfaff]\label{EsattezzaPfaff}
Sia $\sum_i A_idx_i$ una forma. Se l'equazione di Pfaff
\[\sum_i A_idx_i=0\]
\`e integrabile\footnote{cio\`e i punti che la verificano sono descrivibili tramite una equazione $F(x_1,\cdots, x_n)=cost.$} allora la forma \`e chiusa ed esiste $u(x_1,\cdots, x_n)$ tale che $\sum uA_i dx_i$ \`e esatta.
\end{proposition}
\begin{proof}
Sia $\cpa{F=0}$ l'equazione del luogo dove vale l'equazione Pfaff. Segue che
\[dF=\sum_i\pp{x_i}F dx_i=0=\sum_i A_idx_i,\]
esiste dunque $u$ tale che
\[\pp{x_i}F=u(x_1,\cdots, x_n)A_i,\]
da cui derivando
\[\pp{x_i}{}(uA_j)=\pp{x_i\del x_j}{^2F}=\pp{x_j}{}(uA_i),\]
cio\`e $\sum_i uA_idx_i$ \`e chiusa.
\end{proof}

\begin{fact}[Condizione di integrabilit\`a]
Se per ogni terna $i,\ j,\ k$ di indici distinti vale
\[A_k\pa{\pp {x_i}{A_j}-\pp {x_j}{A_i}}+A_j\pa{\pp {x_k}{A_i}-\pp {x_i}{A_k}}+A_i\pa{\pp {x_j}{A_k}-\pp {x_k}{A_j}}=0\]
allora $\sum A_i dx_i=0$ \`e integrabile.
\end{fact}

\begin{remark}\label{IntegrabilitaPuntiRaggiungibili}
Un sistema \`e integrabile se nell'intorno di un punto $P$ esistono infiniti punti non raggiungibili tramite percorsi su cui vale $\sum_i A_idx_i=0$ (ipersuperficie ha dimensione minore dello spazio ambiente).\medskip

\noindent
Segue che se posso raggiungere ogni punto rispettando l'equazione allora non avevamo integrabilit\`a
\end{remark}


\begin{example}
Consideriamo l'equazione $ydx+dy+dz=0$ e passiamo da $(0,0,0)$ a $(a,b,c)$:
\begin{itemize}
\item Variamo solo la $x$ ($dy=dz=0$). Poich\'e $y=0$ e non cambia l'equazione \`e rispettata e passiamo da $(0,0,0)$ a $(x_0,0,0)$ per un qualsiasi $x_0$.
\item Fissiamo $x=x_0$ ($dx=0$) e muoviamoci in modo tale che $dy=dz$ (lungo una diagonale) e cos\`i passiamo da $(x_0,0,0)$ a $(x_0,b,-b)$.
\item Fissiamo $y=b$ ($dy=0$) e muoviamoci in modo che $dz=-bdx$. Passiamo da $(x_0+(a-x_0),b,-b+(a-x_0)(-b))$. Se avevamo scelto $x_0$ in modo tale che $-b+(a-x_0)(-b)=c$ allora abbiamo finito.
\end{itemize}
\end{example}




\section{Trasformazione di Legendre}
Sia $F(x,y)$ con $dF=udx+vdy$\footnote{cio\`e $u=\ppb xFy$ e $v=\ppb yFx$.} e supponiamo di voler riformulare l'espressione in termini di $u$ e $y$. Definiamo 
\[G(u,y)=F-ux\]
e notiamo che
\[dG=dF-udx-xdu=\cancel{udx}+vdy-\cancel{udx}-xdu=vdy-xdu,\]
cio\`e $G$ effettivamente dipende esplicitamente solo da $u$ e $y$.


\chapter{Formulario}
A questo link potete trovare un formulario per il corso redatto l'anno scorso (22/23) da Andrea Rocca e Alessio Sgubin:
\begin{center}
\url{https://poisson.phc.dm.unipi.it/~rocca/materiali/Fisica3.pdf}
\end{center}