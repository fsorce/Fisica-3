\chapter{Teoria cinetica dei gas}
Nella realt\`a i gas sono composti da tante particelle. Imponiamo alcune condizioni:
\setlength{\leftmargini}{0cm}
\begin{itemize}
\item[$\boxed{\text{Isotropo}}$] Le velocit\`a delle particelle sono equamente distribuite in ogni direzione.
\item[$\boxed{\text{Omogeneo}}$] Le particelle sono equamente distribuite.
\end{itemize}
\setlength{\leftmargini}{0.5cm}
Sia $dn(v)$ il numero di particelle con una data velocit\`a.
\begin{remark}
Se $N$ \`e il numero totale di particelle
\[N=\int dn(v)=\int_0^\infty \dd vndv\]
dove $\dd vn$ \`e in un qualche modo la ``densit\`a delle particelle di una data velocit\`a".
\end{remark}
\begin{remark}
Sia $\vec v$ una qualche velocit\`a.
\[dn(\vec v)\pasgnl={isotropia}dn(v)\frac{d\Omega}{4\pi},\]
dove $d\Omega$ \`e l'\textbf{angolo solido}, cio\`e l'area della ambiguit\`a sulla direzione voluta sulla sfera di raggio 1\footnote{$d\Omega=\sin\theta d\theta d\phi$.}.
\end{remark}

\begin{remark}
Per omogeneit\`a il numero di particelle in un volumetto \`e
\[dn=\frac NVdV=\frac NV dAvdt\cos\theta.\]
\end{remark}

\begin{remark}
L'impulso trasferito alla parete dall'impatto di una particella \`e $\abs{\Delta \vec p}=2mv\cos \theta$.
\end{remark}
\noindent Appurate queste equazioni possiamo scrivere il differenziale della pressione come segue:
\begin{align*}
d^2p=&\frac{dF}{dA}=\frac{\abs{d\vec p}/dt}{dA}=\frac1{dA}\frac1{dt}\abs{d\vec p}_{singola}dn dn(\vec v)\\
=&\frac1{dA}\frac1{dt}{2mv\cos \theta}\quad{\frac{N}V dAvdt\cos \theta}\quad{dn(v)\frac{d\Omega}{4\pi}}=\\
=&N\frac{2mv^2\cos^2\theta}Vdn(v)\frac{d\Omega}{4\pi}.
\end{align*}
Facendo la media su tutte le direzioni troviamo il vero differenziale della pressione:
\begin{align*}
dp=&\int_\Omega d^2p=\frac{mv^2}{2\pi}\frac NVdn(v)\int_0^{2\pi}d\phi\int_0^{\pi/2}\cos^2\theta\sin\theta d\theta=\\
=&\frac13mv^2\frac NVdn(v).
\end{align*}
Integrando ora sui possibili moduli delle velocit\`a troviamo la pressione:
\[p=\frac13m\frac NV\under{\doteqdot \ps{v^2}}{\int_0^\infty v^2dn(v)}.\]

\begin{definition}[Enercia cinetica media]
Definiamo l'\textbf{energia cinetica media} come \[\ps{E_K}=\frac12mN\ps{v^2}.\]
\end{definition}
\begin{remark}
Vale la relazione
\[\boxed{\frac12m\ps{v^2}=\frac32 k_b T}\]
\end{remark}
\begin{proof}
Osserviamo che
\[pV=\frac13 mN\ps{v^2},\]
dunque
\[nRT=pV=\frac23\ps{E_K},\]
cio\`e
\[T=\frac 23\frac{\ps{E_K}}{nR}=\frac 23N_a\frac{\ps{E_K}}{NR}=\frac 23\frac{\ps{E_K}}{Nk_b}\]
In conclusione
\[{\frac12m\ps{v^2}=\frac32 k_b T}.\]
\end{proof}

\noindent
Consideriamo ora l'energia interna di questo sistema\footnote{affermare che $U=\ps{E_K}$ corrisponde ad assumere che il gas sia monoatomico.}:
\[U=\ps{E_K}=\frac32nRT=C_VT.\]
In generale $U=E_K+E_P$ per una qualche energia potenziale $E_P$. Per piccoli spostamenti $E_P=(E_P)_0+\frac12kx^2$\ \footnote{regime ragionevole per il tipo di forze che agisce all'interno di materiali.}. Nel caso biatomico per esempio $E_P=\frac12I\omega^2$.
\setlength{\leftmargini}{0cm}
\begin{itemize}
\item[$\boxed{\text{Solido}}$] 6 gradi di libert\`a: 3 potenziali (forze elastiche) e 3 cinetiche.
\item[$\boxed{\text{Gas perf. mono.}}$] 3 gradi di libert\`a, tutti cinetici.
\item[$\boxed{\text{Gas perf. bi.}}$] 5 gradi di libert\`a: 3 cinetici e 2 dalla rotazione \footnote{la rotazione lungo l'asse che congiunge le particelle \`e irrilevante}.
\end{itemize}
\setlength{\leftmargini}{0.5cm}


\begin{fact}[Principio di equipartizione]
Ogni grado di libert\`a contribuisce un addendo $\frac12 RT$ al calore specifico a volume costante.
\end{fact}
