\chapter{Introduzione al Calore}




\section{Modalit\`a di trasferimento di calore}
Il trasperimento di calore, cio\`e di energia derivante da una differenza di temperatura, avviene in tre modi: conduzione, covezione ed irraggiamento.

\subsection{Conduzione}
Parliamo di \textbf{conduzione} quando il tresferimento di calore avviene per contatto ma senza scambio di materia (attraverso una parete diatermica).\medskip

\noindent Empiricamente riscontriamo
\begin{fact}[Legge di Fourier]
Vale la relazione
\[\frac1A\frac{\delta Q}{\Delta t}=-\kappa\frac{\Delta T}{\Delta X},\]
dove $T$ \`e la temperatura, $X$ \`e la distanza tra i punti tra cui stiamo calcolando la differenza di temperatura, $A$ \`e l'area ortogonale alla direzione lungo la quale si propaga il calore e $\kappa$ \`e una costante detta \textbf{conducibilit\`a termica}.
\end{fact}

\noindent L'unit\`a di misura della conducibilit\`a termica \`e
\[[\kappa]=\frac W{mK}\approx \begin{cases}
10^2 &\text{metalli}\\
0.1 &\text{gas}
\end{cases}.\]
\noindent Possiamo precisare la legge di Fourier introducendo la
\textbf{corrente di calore} $\vec J_Q$. La legge assume la forma
\[\vec J_Q=-\kappa\vec \nabla T.\]
Concentrandosi su uno dei sistemi possiamo scrivere
\[\boxed{\delta Q= cm\delta T}\]
dove $m$ \`e la massa e $c$ \`e il \textbf{calore specifico}.\bigskip

\noindent Possiamo calcolare il calore totale che entra dentro una superficie per unit\`a di tempo come
\[\int_V c\pp tT\rho dV=\frac1{\Delta t}\int_{\del V} \delta Q=-\int_{\del V} \vec J_Q\cdot \vec d\Sigma\pasgnl={T.Div.}-\int_V \nabla \cdot \vec J_Q dV=\int_V\kappa\nabla^2 T dV.\]
Ricaviamo dunque
\[\boxed{\pp tT=\frac{\kappa}{\rho c}\nabla^2T}\]
Questa \`e la famosa \textit{equazione del calore}.

\subsection{Convezione}
Parliamo di \textbf{convezione} quando il trasferimento di calore avviene tramite lo spostamento di materia.\\ La formula rilevante in questo caso \`e
\[\frac1A\frac{\delta Q}{\Delta t}=h\Delta T,\]
dove $h$ \`e il \textbf{coefficiente convettivo}.

\subsection{Irraggiamento}
Parliamo di \textbf{irraggiamento} quando un corpo semplicemente emette energia come radiazione.\\ 
La formula rilevante in questo caso \`e
\[\frac1A\frac{\delta Q}{\Delta t}=\e \sigma(T^4-T_0^4),\]
dove $T_0$ \`e la temperatura dell'ambiente, $\sigma$ \`e una costante uguale per tutti i materiali e $\e$ dipende dai materiali.

