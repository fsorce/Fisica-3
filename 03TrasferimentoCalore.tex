\chapter{Trasferimento di calore}

\section{Modalit\`a di trasferimento di calore}
Il trasperimento di calore, cio\`e di energia derivante da una differenza di temperatura, avviene in tre modi: conduzione, covezione ed irraggiamento.

\subsection{Conduzione}
Parliamo di \textbf{conduzione} quando il tresferimento di calore avviene per contatto ma senza scambio di materia (attraverso una parete diatermica).\medskip

\noindent Empiricamente riscontriamo
\begin{fact}[Legge di Fourier]
Vale la relazione
\[\frac1A\frac{\delta Q}{\Delta t}=-\kappa\frac{\Delta T}{\Delta X},\]
dove $T$ \`e la temperatura, $X$ \`e la distanza tra i punti tra cui stiamo calcolando la differenza di temperatura, $A$ \`e l'area ortogonale alla direzione lungo la quale si propaga il calore e $\kappa$ \`e una costante detta \textbf{conducibilit\`a termica}.
\end{fact}

\noindent L'unit\`a di misura della conducibilit\`a termica \`e
\[[\kappa]=\frac W{mK}\approx \begin{cases}
10^2 &\text{metalli}\\
0.1 &\text{gas}
\end{cases}.\]
\noindent Possiamo precisare la legge di Fourier introducendo la
\textbf{corrente di calore} $\vec J_Q$. La legge assume la forma
\[\vec J_Q=-k\vec \nabla T.\]
Concentrandosi su uno dei sistemi possiamo scrivere
\[\boxed{\delta Q= cm\delta T}\]
dove $m$ \`e la massa e $c$ \`e il \textbf{calore specifico}.\bigskip

\noindent Possiamo calcolare il calore totale che entra dentro una superficie per unit\`a di tempo come
\[\int_V c\pp tT\rho dV=\frac1{\Delta t}\int_{\del V} \delta Q=-\int_{\del V} \vec J_Q\cdot \vec d\Sigma=-\int_V \nabla \cdot \vec J_Q dV=\int_Vk\nabla^2 T dV.\]
Ricaviamo dunque
\[\boxed{\pp tT=\frac{\kappa}{\rho c}\nabla^2T}\]
Questa \`e la famosa \textit{equazione del calore}.

\subsection{Convezione}
Parliamo di \textbf{convezione} quando il trasferimento di calore avviene tramite lo spostamento di materia.\\ La formula rilevante in questo caso \`e
\[\frac1A\frac{\delta Q}{\Delta t}=h\Delta T,\]
dove $h$ \`e il \textbf{coefficiente convettivo}.

\subsection{Irraggiamento}
Parliamo di \textbf{irraggiamento} quando un corpo semplicemente emette energia come radiazione.\\ 
La formula rilevante in questo caso \`e
\[\frac1A\frac{\delta Q}{\Delta t}=\e \sigma(T^4-T_0^4),\]
dove $T_0$ \`e la temperatura dell'ambiente, $\sigma$ \`e una costante uguale per tutti i materiali e $\e$ dipende dai materiali.

\section{Capacit\`a termica}
\begin{definition}[Capacit\`a termica]
Definiamo la \textbf{capacit\`a termica} come\footnote{Nota che NON \`e una derivata in quanto $Q$ non \`e una funzione di stato, quindi in particolare non \`e una funzione di $T$}
\[C=\lim_{\delta T\to 0}\frac{\delta Q}{\delta T}.\]
L'unit\`a di misura \`e $[C]=\mathrm{J}/\mathrm{K}$.\\
La \textbf{capacit\`a termica molare} \`e data da $c=C/n$.\\
Il \textbf{calore specifico} \`e dato da $C/m$.
\end{definition}

\begin{definition}[Termometro e Termostato]
Un \textbf{termostato} \`e un oggetto ideale con capacit\`a termica infinita\footnote{intuitivamente \`e un sistema grande a sufficienza in modo che anche se viene aggiunto calore, la temperatura non cambia.}.\\ 
Un \textbf{termometro} \`e un oggetto ideale con capacit\`a termica nulla\footnote{intuitivamente \`e un sistema piccolo a sufficienza in modo da poter trascurare gli scambi di calore.}.
\end{definition}

\begin{remark}
Possiamo scrivere la capacit\`a termica in termini di $U,\ V,\ p$ e $T$ come segue:
\[C=\frac{\delta Q}{\delta T}=\ppb TUV+\pa{\ppb VUT+p}\dd TV\]
\end{remark}
\begin{proof}
Sviluppando $dU$ troviamo
\[dU=\ppb TUVdT+\ppb VUTdV,\]
da cui
\[\delta Q=dU+pdV=\ppb TUVdT+\pa{\ppb VUT+p}dV.\]
Ora possiamo ``dividere" per $dT$ e trovare la tesi.
\end{proof}

\begin{definition}[Capacit\`a termica a volume/pressione costante]
Definiamo la \textbf{capacit\`a termica a volume} (risp. \textbf{pressione}) \textbf{costante} come le due seguenti quantit\`a
\begin{align*}
C_V=&\rbar{\frac{\delta Q}{\delta T}}_V=\ppb TUV\\
C_p=&\rbar{\frac{\delta Q}{\delta T}}_p=\ppb TUV +\pa{\ppb VUT+p}\ppb TVp=\ppb TUV +\pa{\ppb VUT+p}V\al
\end{align*}
\end{definition}

\begin{remark}[Disuguaglianza tra capacit\`a termiche]\label{DisguguaglianzaCapacitaTermiche}
Vale sempre $C_p>C_V$.
\end{remark}

\begin{remark}\label{DerivataEnergiaInternaRispettoAlVolume}
In un gas generale
\[\boxed{\ppb VUT=\frac{C_p-C_V}{V\al}-p}\]
\end{remark}

