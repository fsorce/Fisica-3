\chapter{Transizione di fase}
Generalmente transizioni di stato avvengono per $p$ e $T$ costanti, quindi la forma di energia pi\`u utile da considerare \`e l'energia libera di Gibbs
\[\Delta G=\Delta H-T\Delta S.\]
Poich\'e $dG=-SdT-Vdp$, si ha che se $p$ e $T$ restano costanti allora $\Delta G=0$, cio\`e \[T\Delta S=\Delta H.\]

\section{Equazione di stato dei gas reali}
Cosa pu\`o contribuire a negare l'approssimazione di gas ideale?
\begin{itemize}
\item \textbf{Interazione attrattiva tra particelle:} Una particella vicino al bordo \`e attratta dalle particelle pi\`u interne, dunque la pressione interna al gas \`e pi\`u grande di quella misurata 
\[p_{real}=p+a\frac {n^2}{V^2}.\]
\item \textbf{Volume occupato dalle particelle:} Le particelle in genere occupano un volume
\[V_{real}=V-bn\]
\end{itemize}
\begin{fact}[Legge di Van der Waals]
In prima approssimazione l'equazione di \textbf{Van del Waals}
\[\pa{p+a \frac{n^2}{V^2}}(V-bn)=nRT.\]
\end{fact}
\begin{remark}
Di solito $a$ si aggira tra $10^{-2}$ e $10$ $\displaystyle\frac{\ell^2\mathrm{atm}}{\mathrm{mol}^2}$, mentre $b$ si aggira tra $10^{-2}$ e $10^{-1}$ $\ell/\mathrm{mol}$.
\end{remark}

\noindent
Sotto una temperatura critica, le isoterme secondo l'equazione di Van der Waals diventano cubiche con un picco e una valle, in realt\`a quello che succederebbe nella realt\`a \`e che raggiungiamo le condizioni per \textit{transizioni di fase}.


\section{Transizione tra due fasi}
Chiamiamo le due fasi ``liquido" e ``vapore".\\
Siano $n_L$ le moli di liquido e $n_V$ le moli di vapore.
\begin{remark}[Energia libera di Gibbs molare]
Si ha che\footnote{nell'ultima uguaglianza abbiamo usato il fatto che l'energia \`e una grandezza estensiva.}
\[G=G_L(p_L,T_L,n_L)+G_V(p_V,T_V,n_V)\pasgnl={}n_Lg_L(p_L,T_L)+n_Vg_V(p_V,T_V)\]
dove $g$ \`e l'\textbf{energia libera di Gibbs molare}.
\end{remark}

\begin{remark}[Proporzione tra le fasi]
Per conservazione della materia $n_L+n_V=n$ \`e costante. Se $\al$ \`e la proporzione di liquido, cio\`e $n_L=\al n$ e $n_V=(1-\al)n$.
\end{remark}
Possiamo riscrivere l'energia libera di Gibbs in termini di $n$ ed $\al$:
\[G=n\al g_L(p_L,T_L)+n(1-\al)g_V(p_V,T_V).\]
Poich\'e consideriamo tutto in regime di quasi equilibrio $T_L=T_V$ e $p_L=p_V$\footnote{le pressioni sono le stesse perch\'e c'\`e equilibrio meccanico.}. Abbiamo dunque ricavato che $G$ dipende solo da $p,\ T$ e $\al$.

\begin{remark}[Condizione di equilibrio]
Per $p$ e $T$ costanti sappiamo che $G$ tende a diminuire (\ref{GibbsDiminuiscePerPressioneETemperaturaCostanti}), quindi siamo all'equilibrio se $G(\al)$ \`e minima, cio\`e
\[0=\pp\al G=ng_L+0-n g_V\implies g_L=g_V.\]
\end{remark}














*****************************


Modelliamo una contenitore chiuso ($U=cost$, $V=cost$). In questo caso

\begin{align*}
V=&n\al v_L+n(1-\al)v_V\\
U=&n\al u_L+n(1-\al)u_V\\
S=&n\al s_L+n(1-\al)s_V
\end{align*}


All'equilibrio $0=dV=dU=dS=dn$.

\begin{align*}
0=&dU+pdV-TdS=\\
=&n\al (du_L+pdv_L-Tds_L)+n(1-\al)(du_V+pdv_V-Tds_V)+\\
&+(d\al (u_L+pv_L-Ts_L)-d\al (u_V+pv_V-Ts_V))n
\end{align*}
Per essere all'equilibrio le tre parentesi sono nulle, dunque
\[u_L+pv_L-Ts_L=u_V+pv_V-Ts_V,\]
cio\`e evidenziando le moli
\[\under{=\frac{G_L}{n_L}}{\frac{U_L}{n_L}+p\frac{V_L}{n_L}-T\frac{S_L}{n_L}}=\under{=\frac{G_V}{n_V}}{\frac{U_V}{n_V}+p\frac{V_V}{n_V}-T\frac{S_V}{n_V}},\]
ovvero
\[g_L=g_V.\]



\section{Caso generale}
Consideriamo $N$ componenti\footnote{moralmente $N$ \`e il numero di sostanze diverse} e $F$ fasi. Sia $n^{(i)}_k$ il numero di moli della componente $i$ nella fase $k$.\medskip

\noindent
Osserviamo che
\[G=\sum_{k\in\cpa{1,\cdots, F}}G_k(T,p,(n_k^{(i)})_{i\in\cpa{1,\cdots, N}}),\]
cio\`e a priori $G$ dipende da $NF+2$ variabili.

\begin{proposition}[Regola delle fasi di Gibbs]\label{RegolaFasiGibbs}
Il numero di gradi di libert\`a \`e
\[\nu=2+N-F.\]
Questa \`e la \textbf{regola delle fasi (di Gibbs)}.
\end{proposition}
\begin{proof}
Consideriamo due fasi ($a$ e $b$) e la transizione dalla fase $a$ alla fase $b$:
\[\begin{cases}
n_a^{(i)}\to n_a^{(i)}-\delta n^{(i)}\\
n_b^{(i)}\to n_b^{(i)}-\delta n^{(i)}
\end{cases}.\]
Si ha che all'equilibrio
\[0=dG=dG_a+dG_b=\pp {n_a^{(i)}}{G_a}(-\delta n^{(i)})+\pp {n_b^{(i)}}{G_a}(\delta n^{(i)}),\]
ma l'energia libera di Gibbs \`e una grandezza estensiva, quindi vale la dipendeza lineare
\[\pp {n_k^{(i)}}{G_k}=\frac{G_k}{n_k^{(i)}},\]
segue dunque che
\[g_a^{(i)}=\frac{G_a}{n_a^{(i)}}=\frac{G_b}{n_b^{(i)}}=g_b^{(i)}.\]
Queste sono $F-1$ condizioni indipendenti per ogni componente.
\bigskip

\noindent Osserviamo inoltre che per ogni fase possiamo eliminare un grado di libert\`a considerando i rapporti tra le moli di componenti in quella fase***********.\bigskip

\noindent Tirando le somme si ha che i gradi di libert\`a sono
\[NF+2-(N(F-1)+F)=2+N-F.\]
\end{proof}



\begin{example}
Studiamo i valori di $N$, $F$ e $\nu$ per alcuni sistemi
\begin{itemize}
\item Fluido omogeneo: $N=1$, $F=1$, $\nu=2$
\item Fluido omogeneo dato da due gas: $N=2$, $F=1$, $\nu=3$
\item Acqua e vapore: $N=1$, $F=2$, $\nu=1$
\item Acqua, vapore e ghiaccio: $N=1$, $F=3$, $\nu=0$\footnote{non stupisce in quanto abbiamo tutte le fasi sono nel punto triplo.}
\end{itemize}
\end{example}


[ROBA SUI GRAFICI DEI PUNTI CRITICI]

\section{Calore latente}
Alla transizione di fase


\[p_{trans}(T),\quad V(p,T)=V(T)\]
\begin{align*}
n=&n_L+n_V\\
V=&n_Lv_L(T)+n_Vv_V(T)\\
U=&n_Lu_L(T)+n_Vu_V(T)
\end{align*}
Osserviamo che $dn_L=-dn_V$, quindi
\[\ppb VUT=\frac{u_V-u_L}{v_V-v_L}.\]
Per il primo principio
\[\delta Q=dU+pdV=dn_V(u_L-u_L+p(v_V-v_L))\]

\begin{definition}[Calore latente]
Definiamo il \textbf{calore latente (molare) di vaporizzazione} come
\[\la=\frac{\delta Q}{dn_V}=u_V-u_L+p(v_V-v_L).\]
\end{definition}

\begin{proposition}[Equazione di Clapeyron]\label{EquazioneClapeyron}
Sulla transizione di fase
\[\dd Tp=\frac\la{T(v_V-v_L)}\]
\end{proposition}
\begin{remark}[Equazione di Clausius-Clapeyron]
Se $v_V\gg v_L$ allora per gas ideali
\[\dd Tp=\frac\la{RT^2}p\leadsto p\propto e^{-\la/RT}.\]
\end{remark}