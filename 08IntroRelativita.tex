\chapter{Introduzione}
\section{I principi della relativit\`a}

\subsection{Principio di relativit\`a}

\begin{definition}[Sistema di riferimento inerziale]
Un sistema di riferimento \`e \textbf{inerziale} se valgono le leggi di Newton.
\end{definition}

\begin{fact}[Principio di Relativit\`a]
\textbf{Le leggi della fisica sono le stesse in qualsiasi sistema di riferimento in moto rettilineo uniforme.}
\end{fact}

Dati due sistemi di riferimento $S$ e $S'$ come in figura

[DISEGNO]

supponiamo che $S'$ si muova in moto rettilineo uniforme rispetto a $S$. Supponiamo inoltre che per $t=0$ si abbia $O=O'$.
\medskip

\noindent
Consideriamo ora un punto $P$. Nel sistema $S$ esso \`e espresso tramite le coordinate $(x,y,z)$, mentre in $S'$ le sue coordinate sono $(x',y',z')$. Osserviamo che al variare di $t$ si ha che la terna $(x',y',z')$ cambia.\\
Se i sistemi sono come in figura troviamo la forma canonica delle \textbf{trasformazioni di Galileo}:
\[\begin{cases}
x'=x-ut\\
y'=y\\
z'=z\\
t'=t
\end{cases}\]
Osserviamo che
\[\dd t{x'}=\dd t{}(x-ut)\pasgnl={moto rett. unif.}\dd tx-u,\]
dunque
\[\begin{cases}
v_x'=v_x-u\\
v_y'=v_y\\
v_z'=v_z
\end{cases}\]
e derivando di nuovo troviamo le trasformazioni anche per le accelerazioni:
\[\begin{cases}
a_x'=a_x\\
a_y'=a_y\\
a_z'=a_z
\end{cases}\]

\begin{example}[Le leggi della fisica non cambiano]
Consideriamo la legge di Hooke
\[m\dd[2] tx=-k(x-x_0).\]
Cambiando sistema di riferimento $x=x'+ut$, $x_0=x_0'+ut$, dunque
\[m\dd[2]t{x'}=m\dd[2] t{}(x'+ut)=m\dd[2] tx=-k(x-x_0)=-k(x'+ut-(x_0'+ut))=-k(x'-x_0'),\]
cio\`e la legge continua a valere sostituendo le grandezze di un sistema con le equivalenti nell'altro.
\end{example}

\subsection{Costanza della velocit\`a della luce}
\begin{fact}[Equazioni di Maxwell]
Valgono le seguenti equazioni
\begin{align*}
&div\vec E=\frac{\rho}{\e_0}
&&div\vec B=0\\
&rot\vec E=-\pp t{\vec B}
&&rot\vec B=\mu_0\pa{\vec J+\e_0\pp t{\vec E}}
\end{align*}
\end{fact}
\noindent
Osserviamo che
\begin{align*}
\pp t{}=&\pp {t}{t'}\dd{t'}{}+\pp t{x'}\pp {x'}{}+\pp t{y'}\pp {y'}{}+\pp t{z'}\pp {z'}{}=\pp{t'}{}-u\pp{x'}{}\\
\pp x{}=&\pp {x'}{}\\
\pp y{}=&\pp {y'}{}\\
\pp z{}=&\pp {z'}{}
\end{align*}

\begin{example}
Le equazioni di Maxwell e la relativit\`a Galileiana non sono compatibili
\end{example}
\begin{proof}
Cambiando sistema di riferimento alle equazioni di Maxwell (per esempio una di quelle con rotore) troviamo
\[\pp {y'}{E_{z'}}-\pp {z'}{E_{y'}}=-\pp {t'}{B_{x'}}+u\pp {x'}{B_{x'}}\]
ovvero, applicando l'equazione con la divergenza
\[\pp {y'}{E_{z}+uB_y}-\pp {z'}{E_{y}-uB_z}=-\pp {t'}{B_{x}}\]
possiamo dunque ipotizzare
\[\begin{cases}
E_z'=E_z+uBy\\
E'_y=E_y-uB_z\\
E'_x=E_x\\
\vec B'=\vec B
\end{cases}\]
Problema, esiste un'altra equazione di Maxwell
\[\pp y{B_z}-\pp z{B_y}=\frac1{c^2}\pp t{E_x}\]
che non coninua a valere con le sostituzioni sopra.
\end{proof}

\noindent Ricordiamo che dalle equazioni di Maxwell segue la legge
\[\pp[2]t{\vec E}=c^2\nabla^2\vec E.\]
Questa equazione sembra quella di un'onda, ma allora la relativit\`a magari funziona se teniamo conto degli stessi effetti che subiscono le onde sotto queste trasformazioni.\\
In particolare ipotizziamo l'esistenza di un mezzo attraverso il quale la luce si propaga: l'etere.

\begin{fact}[Esperimento di Michelson-Morley]
L'etere e la terra non sono indipendenti.
\end{fact}
\begin{proof}[Descrizione dell'esperimento]
Esperimento con interferometro e specchi.

[DISEGNO]

Supponiamo che il vento d'etere sia diretto lungo $BD$, cio\`e la terra si muove in quella direzione rispetto al riferimento dell'etere:
\begin{align*}
&t(BD)=t_1\\
&t(DB)=t_2\\
&t(BC)=t(CB)=t_3
\end{align*}
Se $L$ \`e la lunghezza di $BD$ e di $BC$, se $u$ \`e la velocit\`a dell'a terra rispetto all'etere allora
\[t_1=\frac L{c-u}\quad t_2=\frac L{c+u}.\]
Consideriamo ora il sistema di riferimento dove la velocit\`a della luce \`e $c$, cio\`e nel sistema solidale all'etere:\\
Per quanto riguarda il tratto $BD$ in questo sitema 
\[ct_1=L+ut_1,\quad ct_2=L-ut_2\quad\implies\quad t_1+t_2=\frac{2Lc}{c^2-u^2},\]
mentre sul tratto $BC$ si ha
\[(ct_3)^2=L^2+(ut_3)^2\quad\implies\quad t_3=\frac L{\sqrt{c^2-u^2}}\]
e il tempo che ci interessa \`e $2t_3$. Si ha dunque
\[2t_3=\frac{2L}{\sqrt{c^2-u^2}}\neq \frac{2Lc}{c^2-u^2}=t_1+t_2.\]
Ammettiamo allora di aver sbagliato qualche misura in modo tale che le distanze $BC$ e $BD$ non siano identiche. Si pu\`o ricavare
\[t_1+t_2-2t_3\approx \frac{2L_{BD}}c\pa{1+\frac{u^2}{c^2}}-\frac{2L_{BC}}c\pa{1+\frac{u^2}{2c^2}}.\]
Se ruotiamo l'esperimento, l'effetto \`e scambiare i valori di $L_{BD}$ e $L_{BC}$ ma in ogni caso non \`e stata misurata una differenza.
\end{proof}
\noindent
Cosa potrebbe star succedendo?
\setlength{\leftmargini}{0cm}
\begin{enumerate}
\item[$\boxed{Fizeau}$] Magari la terra trascina l'etere
\item[$\boxed{Fitzgerald}$] Magari le lunghezze parallele parallele alla direzione di modo si contraggono
\[L_{\parallel}=L_0\sqrt{1-\frac{u^2}{c^2}}.\]
Con questo cambiamento la differenza dei tempi trovata nell'esperimento di MIchelson-Morley effettivamente si annulla. 
\end{enumerate}
\setlength{\leftmargini}{0.5cm}

\section{Trasformazioni di Lorenz}
Scriviamo \[\gamma=\frac1{\sqrt{1-\frac{u^2}{c^2}}}\] e notiamo che la correzione di Fitzgerald corrisponde a dire $\gamma L_\parallel=L_0$. Deduciamo per il principio di relativit\`a che
\[x=\frac{x'}\gamma+ut,\qquad x'=\frac{x}\gamma-ut'\]
da cui ricaviamo le \textbf{trasformazioni di Lorenz}
\[\begin{cases}
x'=\gamma (x-ut)\\
y'=y\\
z'=z\\
t'=\gamma\pa{t-\frac{ux}{c^2}}
\end{cases}\]

\begin{remark}
$\gamma$ \`e sempre maggiore o uguale a $1$ in quanto $0\leq u\leq c$.
\end{remark}

\begin{remark}
Se $\frac uc\to 0$ allora $\gamma\to 1$ e quindi ritroviamo le trasformazioni di Galileo.
\end{remark}

\noindent 
\begin{proposition}[Trasformazioni di Lorenz per direzioni arbitrarie]\label{LorenzDirezioneArbitraria}
Per un moto rettilineo uniforme a velocit\`a $\vec u$ si ha
\[\begin{cases}
\vec x'=\vec x+(\gamma -1)\frac{\vec x\cdot \vec u}{u^2}\vec u-\gamma t \vec u\\
t'=\gamma\pa{t-\frac{\vec u\cdot \vec x}{c^2}}
\end{cases}\]
\end{proposition}
\begin{proof}
Basta notare che la componente parallela al moto \`e
\[\vec x_\parallel=\frac{\vec x\cdot \vec u}{u}\frac{\vec u}{u}\]
mentre quella perpendicolare \`e $\vec x_\perp=\vec x-\vec x_\parallel$.
\end{proof}

\begin{remark}[Forma pi\`u simmetrica delle trasformazioni di Lorenz]
Se definiamo $\beta=\frac uc$ allora osserviamo che
\[\begin{cases}
x'=\gamma (x-\beta ct)\\
y'=y\\
z'=z\\
ct'=\gamma\pa{ct-\beta x}
\end{cases}\]
\end{remark}

\begin{example}[Equazione delle onde]
Consideriamo l'equazione
\[\nabla^2\phi=\frac1{c^2}\pp[2]t\phi.\]
Si ha che questa equazione \`e invariante per le trasformazioni di Lorenz.
\end{example}
\begin{proof}
ESERCIZIO
\end{proof}


\begin{definition}[Intervallo invariante]
L'\textbf{intervallo invariante} \`e
\[(ct)^2-x^2-y^2-z^2.\]
\end{definition}
\noindent Questa quantit\`a \`e detta intervallo invariante perch\'e \`e invariante rispetto alle trasformazioni di Lorenz.

\begin{remark}
Se stiamo misurando della luce allora l'intervallo invariante vale $0$.
\end{remark}


\section{Dilatazione dei tempi e contrazione delle lunghezze}
\begin{definition}[Intervallo di tempo proprio]
Consideriamo due eventi che nel sistema di riferimento $S'$ hanno la stessa posizione. L'intervallo di tempo misurato da un orologio fermo rispetto a $S'$ tra questi \`e detto \textbf{intervallo di tempo proprio} tra i due
\[\Delta t'=\Delta t_0\]
\end{definition}

\noindent
Sia $\Delta t'=t_1'-t_2'$ l'intervallo di tempo proprio tra due eventi. Cambiando sistema di riferimento troviamo
\[\Delta t=t_1-t_2=\frac{\Delta t'+\cancelto0{\Delta x'} \frac{u}{c^2}}{\sqrt{1-\frac {u^2}{c^2}}}=\gamma \Delta t'\]
dunque l'intervallo di tempo misurato da un sistema in movimento rispetto alla posizione dei due eventi \`e maggiore rispetto a quello misurato da un sistema che li vede alla stessa posizione.\\
Questo fenomeno \`e detto \textbf{dilatazione dei tempi}.
\bigskip


\begin{definition}[Lunghezza propria]
Fissiamo un sistema di riferimento $S'$ e consideriamo una distanza $\Delta x'$ tra due punti fermi in questo sistema. Questa \`e detta la \textbf{lunghezza propria}.
\end{definition}
Sia $\Delta x'$ la distanza misurata tra due punti fermi rispetto a $S'$ in $S'$. Cambiando sistema di riferimento troviamo
\[\Delta x=x_1-x_2=\frac{\Delta x'+u\Delta t'}{\sqrt{1-\frac{u^2}{c^2}}}\]
Affinch\'e la misura di questa lunghezza abbia senso in $S$ dovremo avere $\Delta t=0$ (mentre $\Delta t'$ a priori pu\`o essere qualsiasi valore, tanto la distanza in $S'$ non dipende dal tempo). Consideriamo dunque la trasformazione inversa
\[\Delta x'=\frac{\Delta x+u\cancelto0{\Delta t}}{\sqrt{1-\frac{u^2}{c^2}}}=\gamma \Delta x\]
dunque $\Delta x=\gamma\ii \Delta x'$, cio\`e la lunghezza misurata in $S$ \`e pi\`u piccola rispetto a quella in $S'$.\\
Questo fenomeno \`e detto \textbf{contrazione delle lunghezze}.

\begin{example}[Relativit\`a della simultaneit\`a]
Se due osservatori sono in moto l'uno rispetto all'altro e uno dei due misura due eventi con posizioni diverse ma allo stesso istante allora il secondo osservatore vede i due eventi come non simultanei. Questo segue immediatamente da
\[\Delta t'=\gamma\pa{\Delta t-\frac{u\Delta x}{c^2}}=-\frac{\gamma u}{c^2} \Delta x\neq 0.\]
\end{example}



\section{Addizione delle velocit\`a}
\begin{proposition}[Formula del Boost]\label{FormulaBoost}
Se un oggetto si muove parrallelo al moto tra due sistemi di riferimento inerziali allora
\[v=\dfrac{v'+u}{\displaystyle 1+\frac{uv'}{c^2}}.\]
\end{proposition}
\begin{proof}
Consideriamo $x'(t')=v't'$ e portiamo nel sistema $S$
\[x=\gamma(x'+ut')=\gamma(v'+u)t'=\gamma^2(v'+u)\pa{t-\frac{ux}{c^2}}\]
dunque
\[x\pa{1+\frac u{c^2}\gamma^2(v'+u)}=\gamma^2(v'+u)t,\]
da cui la formula voluta
\[x=\frac{v'+u}{1+\frac{uv'}{c^2}}t.\]
\end{proof}

\begin{remark}
Se approssimiamo $\frac{u}{c}\to 0$ e $\frac{v'}{c}\to 0$ allora ritroviamo il Boost Galileiano.
\end{remark}

\begin{remark}
Se $v'=c$ oppure $u=c$ troviamo $v=c$, che \`e l'assioma sulla costanza della velocit\`a della luce.
\end{remark}

\begin{remark}
Se il moto non \`e allineato con quello dei sistemi troviamo
\[\begin{cases}
v_x=\dfrac{v_x'+u}{1+\frac{v_x' u}{c^2}}\\\\
v_y=\dfrac{v_y'}{\gamma\pa{1+\frac{v_x' u}{c^2}}}\\\\
v_z=\dfrac{v_z'}{\gamma\pa{1+\frac{v_x' u}{c^2}}}
\end{cases}\]
\end{remark}

\noindent La formula vettoriale per il boost di Lorenz \`e
\[\vec v=\frac1{1+\frac{\vec u\cdot \vec v'}{c^2}}\pa{\frac1\gamma \vec v'+\vec u+\pa{1+\frac1\gamma}\frac{\vec u\cdot \vec v'}{c^2}\vec u}\]
o equivalentemente
\[\vec v=\frac1{1+\frac{\vec u\cdot \vec v'}{c^2}}\pa{\vec v'+\vec u+\frac1{c^2}\frac{\gamma}{1+\gamma} \vec u\times (\vec u\times \vec v')}\]


\begin{remark}
Se interpretiamo $\gamma$ come una funzione $\gamma(u)=\pa{1+\frac{u^2}{c^2}}^{-\frac12}$ allora ricaviamo
\[\gamma(v)=\gamma(u)\gamma(v')\pa{1+\frac{\vec u\cdot \vec v'}{c^2}}\]
\end{remark}


\begin{example}
Consideriamo come cambia la velocit\`a della luce entro un mezzo con indice di rifrazione $n$ tra due sistemi in moto relativo a velocit\`a $u$:
\[v=\frac{c/n+u}{1+\frac{cu}{nc^2}},\]
da cui la differenza tra le velocit\`a \`e
\[\Delta v=v-\frac cn=\frac{u(1-n^{-2})}{1+\frac{uc}n},\]
che per $u\ll c$ si approssima a $u\pa{1-\frac1{n^2}}$.
\end{example}

\begin{definition}[Rapidit\`a]
Definiamo la \textbf{rapidit\`a} di un boost a velocit\`a $u$ come
\[\xi=\mathrm{arctanh}\pa{-\frac uc}\]
\end{definition}
\noindent
Ricordando la forma simmetrica delle trasformazioni di Lorenz
\[\begin{cases}
x'=\gamma (x-\beta ct)\\
y'=y\\
z'=z\\
ct'=\gamma\pa{ct-\beta x}
\end{cases},\]
il fatto che $\cosh^2\xi-\sinh^2\xi=1$ e che $(ct')^2-x'^2=(ct)^2-x^2$ si ha che
\[\begin{cases}
x'=\cosh\xi x+\sinh\xi ct\\
ct'=\cosh \xi +\sinh\xi x
\end{cases}\]
Segue che
\[\mat{ct'\\ x'}=\mat{\cosh \xi &\sinh \xi\\\sinh \xi & \cosh \xi}\mat{ct\\ x},\]
cio\`e il boost corrisponde ad una rotazione iperbolica.

\begin{remark}
Considerando una composizione di velocit\`a 
\[u=\dfrac{u_1+u_2}{1+\frac{u_1u_2}{c^2}}\]
allora le corrispondenti rapidit\`a si sommano, cio\`e
\[\xi=\xi_1+\xi_2.\]
\end{remark}

\section{Quadrivettori}
\begin{definition}[Quadrivettore]
Defniamo il \textbf{quadrivettore posizione} come
\[x^\mu=(ct,x,y,z).\]
In particolare $x^0=ct$, $x^1=x$, $x^2=y$ e $x^3=z$.
\end{definition}

Cerchiamo le trasformazioni lineari\footnote{Le cerchiamo lineari perch\'e vogliamo indipendenza dall'origine} che trasformano il quadrivettore preservando l'intervallo invariante.

\begin{definition}[Tensore metrico di Minkowski]
Definiamo il \textbf{Tensore metrico di Minkovski} tramite la matrice
\[\eta_{\mu\nu}=\mat{1&&&\\
&-1&&\\
&&-1&\\
&&&-1}\]
\end{definition}
\noindent
Osserviamo che $L$ preserva l'intervallo invariante se e solo se
\[\eta=L^\top\eta L.\]
\begin{remark}
Passando ai determinanti segue subito che $\det L=\pm 1$.
\end{remark}


\noindent Consideriamo l'equazione sopra in coordinate:
\[\sum_\nu\sum_\mu(L^\top)_{\al\mu}\eta_{\mu\nu}L_{\nu\beta}=\eta_{\al\beta}\]
nella convezione di Einstein possiamo evitare di scrivere i simboli di somma, dunque
\[L_{\mu\al}\eta_{\mu\nu}L_{\nu\beta}=\eta_{\al\beta}.\]
Osserviamo che queste equazioni non cambiano scambiando $\al$ e $\beta$.

Come convensione indici con lettere greche possono assumere valori tra $1$ e $4$, mentre indici latini solo tra $1$ e $3$.\\
Supponiamo che il moto avvenga lungo l'asse $x$ ($y'=y$, $z'=z$ e $x'$ non dipende da $y$ o $z$), cio\`e
\[L_{ij}=0 \quad i,j\in\cpa{1,2,3},\ i\neq j\]
Consideriamo ora vari casi:
\begin{itemize}
\item Se $\al=i$ e $\beta=j$ per $i\neq j$ allora
\[L_{\mu i}\eta_{\mu\nu}L_{\nu j}=\eta_{ij}=0\implies L_{02}=L_{03}=0\]
\item Se $\al=0$ e $\beta=j$ allora
\[L_{\mu0}\eta_{\mu\nu}L_{\nu j}=\eta_{0j}=0\implies L_{00}L_{0j}-L_{j0}L_{jj}.\]
Intuitivamente $L_{00}$ e $L_{jj}$ non sono nulli perch\'e altrimenti $x'$ non dipenderebbe da $x$ e similmente per le altre componenti, quindi abbiamo trovato
\[L_{j0}=L_{0j}\frac{L_{00}}{L_{jj}}\]
In particolare $L_{20}=L_{30}=0$ e $L_{10}=L_{01}\frac{L_{00}}{L_{11}}$.
\item Se $\al=\beta=0$ allora abbiamo
\[L_{00}^2-L_{10}^2=\eta_{00}=1\]
\item Se $\al=\beta=i$ allora
\[L_{0i}^2-L_{ii}^2=\eta_{ii}=-1,\]
dunque $L_{01}^2=L_{11}^2-1$ e $L_{22}^2=L_{33}^2=1$.
\end{itemize}
\noindent
Battezziamo $L_{11}=\gamma>0$ e notiamo che
\[\begin{cases}
L_{10}=L_{01}\frac{L_{00}}{\gamma}\\
L_{01}=\pm\sqrt{\gamma^2-1}\\
L^2_{00}=L^2_{10}+1
\end{cases}\]
da cui
\[L_{00}^2=L_{00}^2\pa{1-\frac1{\gamma^2}}+1\implies L_{00}=\pm\gamma\]
e $L_{10}=\pm L_{01}=\pm \sqrt{\gamma^2-1}$.
\medskip

\noindent Mettendo tutto insieme (quindi anche $\det L=1$) troviamo la seguente forma per $L$\footnote{abbiamo supposto $L_{00}>0$ perch\'e altrimenti futuro e passato si scambierebbero}:
\[L=\mat{\gamma &\pm\sqrt{\gamma^2-1}&\\\pm\sqrt{\gamma^2-1}&\gamma&\\&&1&\\&&&1}\]
Queste sono esattamente le trasformazioni di Lorenz, infatti se $\gamma=\pa{1-\frac{u^2}{c^2}}^{-\frac12}$ allora $\sqrt{\gamma^2-1}=\frac uc\gamma$.

\subsection{Diagramma di Minkowski}
[DISEGNINO]
\begin{definition}[Quadrivettore di tipo tempo/spazio luce]
Dato un quadrivettore $x^\mu$ definiamo la sua norma di Minkowski come \[s^2=x^\mu \eta_{\mu\nu}x^\nu=(x^0)^2-(x^1)^2-(x^2)^2-(x^3)^2.\]
Affermiamo che il quadrivettore \`e di tipo 
\begin{itemize}
\item \textbf{tempo} se $s^2>0$
\item \textbf{spazio} se $s^2<0$
\item \textbf{luce} se $s^2=0$
\end{itemize}
Se il quadrivettore \`e di tipo tempo allora esso appartiene al \textbf{futuro} se $x^0>0$ o al \textbf{passato} se $x^0<0$.
\end{definition}

\begin{remark}
Un quadrivettore ortogonale ad uno di tipo tempo \`e di tipo spazio, ma un quadrivettore ortogonale ad uno di tipo spazio non necessariamente \`e di tipo tempo.
\end{remark}

\begin{proposition}[La causalit\`a viene rispettata]
Consideriamo due eventi $A$ e $B$. Se $\Delta t>0$ e $\Delta t'<0$ allora i due eventi non possono essere l'uno la causa dell'altro, cio\`e il quadrivettore dato dalla loro differenza \`e di tipo spazio.
\end{proposition}
\begin{proof}
Applicando la trasformazione di Lorenz
\[\Delta t'=\gamma\pa{\Delta t-\frac u{c^2}\Delta x}\]
dunque se $\Delta t'<0$ e $\Delta t>0$ allora
\[-\gamma\Delta t>-\gamma\frac u{c^2}\Delta x\implies \Delta x>\frac{c^2}{u}\Delta t>c\Delta t\]
cio\`e la distanza tra gli eventi \`e maggiore rispetto alla distanza che la luce potrebbe percorrere in quell'intervallo di tempo, quindi i due eventi non possono essere l'uno la causa dell'altro.
\end{proof}

\noindent Per semplicit\`a ignoriamo le componenti $y$ e $z$. Definiamo una base ortonormale dello spazio di Minkowski:
\[\wt e_0=\mat{\cosh \xi\\\sinh\xi},\quad \wt e_1=\pm \mat{\sinh \xi\\\cosh\xi}.\]
Poich\'e $\cosh^2\xi-\sinh^2\xi=1$ per ogni $\xi$ effettivamente sono vettori di norma di Minkowski $1$, inoltre sono evidentemente ortogonali.
\bigskip

\noindent
Se $\wt e_0'$ e $\wt e_1'$ sono definiti come sopra ma a partire da una rapidit\`a $\xi'$ allora
\begin{align*}
\wt e_0\cdot \wt e_0'=&+\cosh(\xi-\xi')\\
\wt e_1\cdot \wt e_0'=&\pm \sinh(\xi-\xi')\\
\wt e_0\cdot \wt e_1'=&\mp\sinh(\xi-\xi')\\
\wt e_1\cdot \wt e_1'=&-\cosh(\xi-\xi')
\end{align*}

\noindent
Consideriamo ora un cambio di base
\[\wt a=x_0\wt e_0+x_1\wt e_1=x_0'\wt e_0'+x_1'\wt e_1'\]
Si pu\`o verificare che
\[\begin{cases}
\wt e_0=\cosh(\xi-\xi')\wt e_0'+\sinh(\xi-\xi')\wt e_1'\\
\wt e_1=\sinh(\xi-\xi')\wt e_0'+\cosh(\xi-\xi')\wt e_1'
\end{cases}\]
da cui ricaviamo
\[\begin{cases}
x_0'=x_0\cosh(\xi-\xi')+x_1\sinh(\xi-\xi')\\
x_1'=x_0\sinh(\xi-\xi')+x_1\cosh(\xi-\xi')
\end{cases}\]